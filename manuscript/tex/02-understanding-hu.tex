A lot of us think we are never happy. Often just stopping to look is
enough to notice that we are happy where we are. Stop and stay with it.

Understanding happiness.

Részekre osztjuk az ismeretet és fokozatos lépésekben beszélünk a
gyakorlásról, de a megértés egyszerre történik. Ami értékes lesz, az nem
a tények halmozása, hanem elengedés, a feszültség és szenvedés
feloldása.

A tapasztalatok és sikerek halmozásával egyre több külső dologtól függ a
nyugalmunk, és sosem érkezünk meg oda, ahol megállhatunk. Az ilyen
bolyongás kimerítő, tele vagyunk kétségekkel és hiányérzettel. Megállunk
és megértjük, hogy nem többre van szükségünk, hanem ennek az szükségnek
az elengedésére.

Azt számolva mit értünk el és hol állunk másokhoz képest, egyre
savanyúbb képet vágunk -- még ha sikeresek is vagyunk, ha ragaszkodunk
ehhez és azonosulunk a sikerrel, a külső függőséget és belső
hiányérzetet erősítjük. A hirdetések mindenesetre nagyon szeretnék, hogy
ebben higgyünk, hogy mindig a következő dologban lássuk magunkat.

A világi értékek és jutalmak alapetően mindig üresek lesznek. De észre
tudjuk venni, hogy meg tudunk állni. Elegendő a kevés is, nem igénylünk
sokat, és ebben a visszafogottságban olyan erőt és energiát találunk,
ami mindig kéznél van, és amit külső tényezők nem fenyegetnek.

A légzést figyeljük, és az érzékek működését. A szem formákat lát, és
megjelenik egy érzet. A fül hangokat hall, a test szilárdságot, hideget
és meleget érzékel. A kapcsolattal megjelenik az érzet. Ha van kapcsolat
az érzék és érzék-tárgy között, a folyamat nem rajtunk múlik, nincs
választásunk az érzet megjelenésében. Amikor a kapcsolat az érzék és
érzék-tárgy között megszakad, megszűnik az érzet, ami a tapasztalatunk
volt, az elmúlása sincs az irányításunk alatt.

Amikor ezt nem vesszük észre, úgy gondoljuk a miénk lesz az érzés,
valamit nyerünk majd belőle, ezért ragaszkodunk hozzá. A szükséget és
nyugtalanságot ez a helytelen megértés teszi lehetővé, és amikor
csalódunk az elvárásainkban, nem a hozzáállásunkat vizsgáljuk meg, hanem
úgy gondoljuk egy másik, új dolog majd nem így fog viselkedni.

Az érzékek kapcsolatát így figyelve, az érzések már nem vonzóak vagy
visszataszítóak, mert látjuk, hogy egyik sem lesz stabil és megbízható.
Ez nem tompaság -- a meditációban a tapasztalat felé fordulunk, nem
attól el -- éberek és érzékenyek maradunk a tapasztalatra, de az már nem
irányít minket, és higgadtak maradunk.

Nem hiszünk a történetekben arról, hogy kik vagyunk és mit kell
érezzünk. Magunkban vagyunk, a hiányt felváltja az elégedett nyugalom,
pedig nem kellet semmit megszereznünk hozzá.

Egy belégzés, egy kilégzés, figyeljük az elmét, és megismerjük, hogy
tudunk magunkban bízni, mert vissza tudjuk fogni magunkat.

Egy tiszta, jótékony, visszafogott gondolat elcsendesíti az elmét.
Éberek vagyunk az érzéseinkre, de azok nem irányítanak minket, mert ez a
visszafogottság ad magunknak időt az intelligens vizsgálatra. Megvan az
erőnk arra, hogy lemondjunk arról, amire valójában sosem volt igazán
szükségünk. Ezzel a nyugodt erővel pedig nekiláthatunk annak, amit tenni
akarunk.

A megértés igazságát abból ismerjük fel, hogy a saját tapasztalatunkra
épül, de az eredményt nem mi hoztuk létre, csupán megértettük a dolgok
igazságát, ami korábban is ott volt.

Hagyjuk magunkat elcsendesülni, és amikor elfelejtjük folyton ismételni
magunknak a belső történeteinket és a kritikáinkat, a tisztánlátás és
megismerés magától megjelenik. Nem mi hozzuk létre. Elég, ha félreállunk
az útból.

Így tanulunk a Buddha tanításáról. Segít az, hogy részekben tudunk róla
gondolkodni, és mind az ismeretünk, mind a kérdéseink így tisztábban
kifejezhetőek, de közben emlékezünk arra, hogy nem ez a tudás lesz
értékes, hanem az, hogy ezt a tapasztalataink megértésere tudjuk
irányítani.

Amit tapasztalunk, egy folyamat részeként tekintjük. Először a felszíni
tapasztalatra figyelünk fel -- főleg, ha fájdalmas, a kellemes
tapasztalatok után valahogy nem igénylünk annyi magyarázatot.

A folyamatok gyökereit keressük, az eseményeket okok és
következményekként vizsgáljuk. A tapasztalunk, ezt érezzük mint
következmény. Az ehhez való hozzáállásunk meghatározza hogyan
vizsonyulunk hozzá és mit cselekszünk, ez újabb okozó tényező lesz,
aminek következményeit -- legyenek kellemesek vagy fájdalmasak -- meg
fogjuk tapasztalni.

Alapvetően a saját hozzáállásunkon, szándékaink és tetteinken van a
hangsúly. Ebben van szabad választásunk, és képességünk, hogy
befolyásoljuk mi történik velünk. Nem tudjuk irányítani a világot magunk
körül, de a hozzáállásunk nyitja meg vagy zárja be az ajtókat a tettek
felé, amiket lehetségesnek látunk, és a tetteink hozzák létre azt a
helyzetet, ahol az életünket éljük.

A kezdő álláspont azt elismerni, hogy a feszültség, a szenvedés itt van.
Ezt ismeretként talán könnyen elfogadjuk, hogy igen, van a világban
feszültség és szenvedés, de amikor magunk tapasztaljuk, szeretünk mégis
inkább valami másra figyelni, vagy hajlamosak vagyunk inkább valakit
hibáztatni érte, csak ne kelljen nekünk elismerni és érdemben
foglalkozni vele.

Az utasítás itt az, hogy csak az fog előre vezetni, ha a szenvedés felé
fordulunk, és azt vizsgálva keressük a megértés módját. A Buddha
tanításában ez az Első Nemes Igazság -- van szenvedés, és a hozzáállás
ami megnemesít az, ha felé fordulunk és megértjük.

Mit értünk meg? Azt, hogy a szenvedés nem a semmiből jött létre, hanem
korábbi tényezők eredménye, és ezek elkerülhetetlen eredményét
tapasztaljuk. Ez nagy előre lépés, mert ha így tudjuk vizsgálni a
helyzetet, nem vagyunk tehetetlenek -- és talán nem értünk minden kicsi
tényezőt, de már az is megkönnyebbülés, hogy talán tudunk valamit
változtatni.

A Második Nemes igazság arra mutat rá, hogy a szenvedés okát magunkban
találjuk -- a saját kívánságunk, hogy a dolgok másképpen legyenek mint
ahogy természetüktől fogva vannak, a görcsös ragaszkodásunk ahhoz, ami
állandótlan, felbomlik, és nem megtartható. Az utasítás, a nemes
hozzáállás itt az, hogy ezt a szomjas kívánságot és ragaszkodást el kell
engednünk. A tapasztalataink, az életünk változékony dolgai nem
érdemesek a ragaszkodásra, ez mindig szenvedéshez vezet, mert a
természetük az állandótlanság. Amikor látjuk, hogy valójában nincs
bennük semmi, amit meg lehetne tartani, megértjük, hogy a ragaszkodással
hogyan okoztunk magunknak fájdalmat.

Az okozó tényező megszűnésével megszűnik az eredmény, a szenvedés is. A
Harmadik Nemes Igazság erre irányítja a figyelmünket -- van megoldás,
nem kötelező keserűségben és értelmetlen küszködésben élnünk. A tanács,
a nemes hozzáálás az, hogy ezt a kapcsolatot fel kell ismernünk a magunk
számára, és az elengedéssel lehetővé tenni a szenvedés megszűnését.

Még ha nem is tudjuk rögtön elengedni, már az is megkönnyebbülés, ha
látjuk, hogy az összefüggés igaz -- `ha elengedném, nem szenvednék
tőle'. Ez már a munka fele, eddig térkép nélkül bolyongtunk, de innen
már van út előre.

A Negyedik Nemes Igazság az út gyakorlását írja le. A Buddha nyolc
tényezőre bontotta, melyek magukban foglalják a mindennapi élet
helyzeteit és a meditáció fejlesztését is.

A Nyolcrétű Ösvény részei a (1) megértés, (2) szándék, (3) beszéd, (4)
tett, (5) megélhetés, (6) erőfeszítés, (7) éberség és (8) elmélyülés.
Amikor összhangban vannak az igazsággal, \emph{helyesnek} nevezzük őket:
Helyes Megértés, Helyes Szándék, és így tovább. A vizsgálódást segíti ez
a részekre bontás, könnyebb ilyen módon gondolkodnuk, viszont a tényezők
egymást erősítik és támogatják, és a gyakorlás egyesült egészként
valósul meg.

Az élet kavargó helyzeteiben nem állhatunk meg tényezőket számolni,
olyan eszköz hasznos, ami hordozható és könnyen elérhető. Mikor
olvasunk, töprengünk a jelentésen, van időnk körbejárni a szavakat, de
az éber figyelem, mint elvont ötlet nem sokat használ -- akkor értékes,
ha kéznél van a jelen pillanatban.
