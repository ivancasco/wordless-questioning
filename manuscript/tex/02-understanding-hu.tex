\chapter{Megértés}

Részekre osztjuk az ismeretet és fokozatos lépésekben beszélünk a
gyakorlásról, de a megértés egyszerre történik. Az észrevétel azonnal
történik, az `Aha!' pillanat, amikor eloszlik a köd. Kezdetben szükséges
némi információ, de emlékszünk arra, hogy a tanítás igazságát mindenki a
saját tapasztalatában éli meg. A meditáció gyakorlásában azok a tények
hasznosak számunkra, amik \emph{itt-és-most} tények, amit a jelen
pillanatban megismerünk.

Ha külső információra támaszkodunk, vagy folyton egy újabb élményre
várunk, sosem érkezünk meg oda, ahol megállhatunk. Ez a függőség
kimerítő, egyre több kétséget és mentális nyugtalanságot hoz létre.
Hiába van belőle több, egyre savanyúbb képet vágunk, és erősödik a belső
hiányérzet. Nem több információra van szükségünk, hanem ennek a
szükségnek az elengedésére, akkor tudunk nyugodtan megállni.

Az emlékek, érzékelés és elvárások egy tapintható erővel húznak-tolnak
minket, ami sosem pihen, míg az élmények maguk eltűnnek és a következőt
akarjuk. Hogyan lehetnek ezek olyan meggyőzőek, hogy folyton húznak
minket előre? \emph{Magunkat látjuk bennük.} Úgy látjuk, hogy ezek
vagyunk, ezek voltunk, ezek leszünk, és mivel szüntelenül változnak és
felbomlanak, egyre tovább a következőt várjuk.

Mi lenne, ha egy nap úgy ébrednénk fel, hogy semmire nem emlékszünk a
múltból? A tapasztalatunkat ilyen állapotban is megragadnánk, mint
\emph{én és enyém}. Annyira kiborító lehetne, hogy félelemtől bénulva
feküdnénk amíg meg nem tudunk ragadni valamilyen történetet, ami
megmagyarázza hol és kik vagyunk. Nem is kell elveszíteni a memóriánkat,
hogy ezt megfigyeljük: utazás közben lekésni egy fontos csatlakozást is
elég ijesztő lehet, hogy így érezzük magunkat, vagy más alkalommal
amikor nem tudjuk a helyzetünket irányítani.

A Buddha azt tanítja, hogy az elme továbbra is az `én és enyém' látást
alkot az érzékek tapasztalatából, amíg ennek a mulandóságát nem értjük
teljesen. Jaj, szegény elménk, miért nem lehet bölcsebb? Megbocsáthatunk
neki -- végül is, azzal foglaljuk le, hogy arról gondolkodunk hogyan
szerezzük meg amit akarunk, és a panaszkodással arról, amit nem kaptunk
meg, ahelyett, hogy a házi feladatunkon dolgoznánk és teljesen
megértenénk az érzéki tapasztalást.

Sokat gondolkodhatunk erről, de ha analitikusan próbáljuk megérteni,
csak fejfájáshoz jutunk. Ezt a megértést a testen keresztül kell
fejleszteni: egy jó kezdőpont az érzéki visszafogás gyakorlása, ez
egyszerűbb is. Összegyűjti a mentális erőnket, és vagy csendes nyugalom,
vagy figyelmes vizsgálódás követi, egy természetes ritmusnak
megfelelően.

Észrevesszük, hogy képesek vagyunk megállni, és ami most velünk van, az
is elegendő. Gondolj arra, mit használtál a mai nap, tárgyakra és
információra is tekintve. Lehet, hogy sok mindened van, de egy napra egy
kevés is elég. Nincs szükségünk annyi mindenre mint ahogy gondoljuk, és
lemondani róla nem veszteség, hanem egy teher alóli felszabadulás. Ebben
a visszafogottságban erőt és energiát találunk, amit korábban a
szétszórt vágy emésztett fel. A visszafogottság mindig kéznél van, és
nem veszélyeztetik külső tényezők.

Olyan ez, mikor megtanulsz kevesebbet pakolni a hátizsákodba egy túra
előtt. Némi tapasztalat után, nem is érted, miért kellett annyi mindent
cipelned korábban. Az erény olyan tettekből áll, amik boldogsághoz
vezetnek, és az erényt önmagunk felé is fordíthatjuk: a lemondás éppen
ilyen. A megértés információval látja el az erényt, és a boldogság, ami
ebből születik, ebbe a megértésbe vetett bizalmat erősíti.

A légzést figyeljük, és az érzékek működését. A szem formákat lát, és
megjelenik egy érzet. A fül hangokat hall, a test szilárdságot, hideget
és meleget érzékel. A kapcsolattal megjelenik az érzet. Ha van kapcsolat
az érzék és érzék-tárgy között, a folyamat nem rajtunk múlik, nincs
választásunk az érzet megjelenésében. Amikor a kapcsolat az érzék és
érzék-tárgy között megszakad, megszűnik az érzet. Ebből állt a
tapasztalatunk, ahogy a kapcsolattól függően keletkezik és elmúlik. A
mozdulataink felett van némi irányításunk, de ezen túl a tapasztalat
folyamatába nincs beleszólásunk, nem rólunk szól.

Amikor ezt nem vesszük észre, úgy gondoljuk a miénk lesz az érzés.
Valamit nyerünk majd belőle, ezért ragaszkodunk hozzá. Ez a helytelen
megértés húzódik a szükség és nyugtalanság érzése mögött. Amikor
csalódunk az elvárásainkban, vagy, azt tesszük fel, hogy mi rontottuk el
valahogy, vagy valaki más hibázott, vagy nem ez volt a megfelelő dolog.
Egy másik élményt kezdünk keresni, ami majd megfelelő lesz, és azt
reméljük az nem így fog lejátszódni.

Az érzékek kapcsolatát így figyelve, az érzések már nem vonzóak vagy
visszataszítóak, mert látjuk, hogy egyik sem lesz stabil és megbízható.
Ez nem tompaság -- a meditációban a tapasztalat felé fordulunk, nem
attól el. Ehelyett egy érzékeny, higgadt figyelmet fejlesztünk. Éberek
és érzékenyek maradunk a tapasztalatra, de az már nem irányít minket, és
nyugodtak maradunk.

Milyen hatással van ez a történetekre, amit magunknak mondunk? Egy
állandó narrátori beszéd megy, és mi vagyunk a mikrofon előtt. Azt
mondjuk magunknak, hogy ez az érzés jó volt vagy rossz, és mit kellene
erről gondolnunk. Egy központi eleme ennek a narrátori szövegnek a
\emph{mi érzéseink}, és hogyan legyen több a jó fajtából.

Mi történik, mikor mind a jók és a rosszak instabilak, megbízhatatlanok,
és a keletkezésük és elmúlásuk nincs az irányításunk alatt? A belső
értékeink átrendeződnek, a mulandóság alapján, a vágy helyett.

---

Egy belégzés, egy kilégzés, figyeljük az elmét, és megismerjük, hogy
tudunk magunkban bízni, mert vissza tudjuk fogni magunkat.

Egy tiszta, jótékony, visszafogott gondolat elcsendesíti az elmét.
Éberek vagyunk az érzéseinkre, de azok nem irányítanak minket, mert ez a
visszafogottság ad magunknak időt az intelligens vizsgálatra. Megvan az
erőnk arra, hogy lemondjunk arról, amire valójában sosem volt igazán
szükségünk. Ezzel a nyugodt erővel pedig nekiláthatunk annak, amit tenni
akarunk.

A megértés igazságát abból ismerjük fel, hogy a saját tapasztalatunkra
épül, de az eredményt nem mi hoztuk létre, csupán megértettük a dolgok
igazságát, ami korábban is ott volt.

Hagyjuk magunkat elcsendesülni, és amikor elfelejtjük folyton ismételni
magunknak a belső történeteinket és a kritikáinkat, a tisztánlátás és
megismerés magától megjelenik. Nem mi hozzuk létre. Elég, ha félreállunk
az útból.

Rendszerint akkor kezdünk csak figyelni, amikor észrevesszük, hogy
valami rossz, valami fáj, valami miatt szenvedünk. A kellemes
tapasztalatokra nem igénylünk annyi magyarázatot, ugye? Ezt a frusztrált
érzést, elégtelenséget és terjengő gondolkodást jelként használhatjuk a
magunk számára, hogy kezdjünk éberen vizsgálódni.

Gondolatban lépj egyet vissza, és figyeld a tapasztalatot mint egy
folyamatot, aminek van eleje, változáson megy keresztül, és megszűnik.
`A testben hol érzem ezt az érzést? Emlékszem, mikor kezdődött? Tudom
figyelni, ahogy változik? El tudom kapni, ahogy az érzés megszűnik?'

Nem tudjuk irányítani a világot magunk körül, de a hozzáállásunk
befolyásolja, mit látunk szabad választásként. A szemléletünk nyitja meg
vagy zárja be az ajtókat a tettek felé, amiket lehetségesnek látunk.
Ezek a tettek hozzák létre a helyzetet, amiben élünk, és ahogy látjuk
magunkat. Ha nem vizsgáljuk a tapasztalataink természetét, a jó
érzéseket jutalomnak és a rossz érzéseket büntetésnek tekintjük, és az
életünk értelme ezek körül fog forogni. A belső világunk mindig azokról
a kérdésekről fog szólni, hogy `Ki vagyok \ldots{} Hogyan tudom \ldots{}
Miért kell \ldots{} Mit kellene \ldots{}', és nem olyan terhes érzés ez,
amit jobb lenne elhagyni?

Emlékszel, milyen egy madzagra kötött súlyt körbe lengetni? Gyorsan megy
körbe-körbe, de mindig kötődik az énhez a középpontban. Egyes kérdések
csapdák, folyton körbe mennek anélkül, hogy szabaduláshoz vagy
megálláshoz vezetnének. A körbe lengetett súly megáll, amikor bele
fáradunk és vagy elengedjük a kötődést, vagy abbahagyjuk a lengetést.
Meg tudjuk ezt tenni a figyelmünkkel is?

A meditációban értelmezve, a vizsgálódás nem azt jeleni, hogy mindenféle
gondolkodás belátást fog eredményezni. A tapasztalatunkat ok-okozati
folyamatra bontjuk le, erre a Négy Nemes Igazságot használjuk
útmutatóként.

Ez egy olyan tapasztalattal kezdődik, amit személyesen könnyű
azonosítanunk: a szenvedés, feszültség, elégtelenség, vagy \emph{dukkha}
a Páli nyelvben. A gondolat iránya nem \emph{az én szenvedésem}, mint
személyes történet, hanem személytelen, természetes folyamatok.

A kezdő álláspont azt elismerni, hogy a feszültség, a szenvedés itt van.
Ez ismeretként triviális, hogy igen, van a világban feszültség és
szenvedés. De amikor magunk tapasztaljuk, szeretünk mégis inkább valami
másra figyelni, vagy hajlamosak vagyunk valakit hibáztatni érte. Bármit,
csak ne kelljen tudatosan elismerjük és érdemben foglalkoznunk vele.

Az utasítás itt az, hogy csak az fog előre vezetni, ha a szenvedés felé
fordulunk, és azt vizsgálva keressük a megértés módját. A Buddha
tanításában ez az Első Nemes Igazság -- van szenvedés, és a nemes
hozzáállás az, ha felé fordulunk és megértjük.

Mit értünk meg? Azt, hogy a szenvedés nem a semmiből jött létre, hanem
korábbi tényezők eredménye. Ha így tudjuk vizsgálni a helyzetet, nem
vagyunk tehetetlenek -- és talán nem értünk minden apró tényezőt, de már
az is megkönnyebbülés, hogy talán tudunk valamit változtatni.

A Második Nemes igazság arra mutat rá, hogy a szenvedés okát magunkban
találjuk. A saját kívánságunk, hogy a dolgok másképpen legyenek mint
ahogy természetüktől fogva vannak, a görcsös ragaszkodásunk ahhoz, ami
állandótlan, felbomlik, és nem megtartható. A szenvedés, a \emph{dukkha}
amit tapasztalunk, ettől a ragaszkodástól, szomjas vágytól függ. Az
utasítás, a nemes hozzáállás itt az, hogy ezt a szomjas vágyat és
ragaszkodást el kell engednünk, mert a mulandó élményekhez ragaszkodás
szenvedés.

Az okozó tényező megszűnésével megszűnik az eredmény, a szenvedés is. A
jó hír, hogy a szenvedés végét is magunkban találjuk.

Ebből a szemszögből, az elme hozza létre azt a fajta világot, amiben
élünk -- ha figyeljük, van esélyünk, hogy legalább ne rontsunk a
helyzeten. És ki tudja, akár még javíthatunk is rajta?

A Harmadik Nemes Igazság erre irányítja a figyelmünket -- van megoldás,
nem kötelező keserűségben és értelmetlen küszködésben élnünk. A tanács,
a nemes hozzáállás az, hogy gyakoroljunk és tapasztaljuk ezt meg a
magunk számára, a megértés és a ragaszkodás elengedésén keresztül,
lehetővé tegyük a szenvedés megszűnését.

Még ha nem is tudjuk rögtön elengedni, már az is megkönnyebbülés, ha
látjuk, hogy az összefüggés igaz -- `ha elengedném, nem szenvednék
tőle'. Ez már a munka fele. Eddig térkép nélkül bolyongtunk, de innen
már van út előre.

A Negyedik Nemes Igazság az út gyakorlását írja le. A Buddha nyolc
tényezőre bontotta, melyek magukban foglalják a mindennapi élet
helyzeteit és a meditáció fejlesztését is.

A Nyolcrétű Ösvény részei a (1) megértés, (2) szándék, (3) beszéd, (4)
tett, (5) megélhetés, (6) erőfeszítés, (7) éberség és (8) elmélyülés.
Amikor összhangban vannak az igazsággal, \emph{helyesnek} nevezzük őket:
Helyes Megértés, Helyes Szándék, és így tovább. A vizsgálódást segíti ez
a részekre bontás, könnyebb ilyen módon gondolkodnunk, viszont a
tényezők egymást erősítik és támogatják. A gyakorlás egyesült egészként
valósul meg.

Amikor leginkább szükségünk van rájuk, \emph{azonnal} kellenek, nem
állhatunk meg tényezőket számolni. Olyan eszköz hasznos, ami hordozható
és könnyen elérhető. Mikor olvasunk, töprengünk a jelentésen, van időnk
körbejárni a szavakat, ez a tanulás szakasza. Viszont az éber figyelem,
mint elvont ötlet nem sokat használ -- akkor értékes, ha kéznél van a
jelen pillanatban.

Mindig ide térünk vissza. Emlékszünk a múltra és tervezünk a jövőről, de
az emlékezés egy jelen tapasztalat, a tervezés egy jelen tapasztalat. A
meditáció gyakorlását nem a jövőért végezzük -- ha a megértést,
szabadságot, boldogságot, akadályok túllépését egy jövőbeli állapotként
látjuk, ezzel csak több a terhünk -- az elengedés a jelenben marad, ahol
az állapotok nélkülünk változnak.
