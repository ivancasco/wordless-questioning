\hypertarget{luxe9gzuxe9s-1}{%
\chapter{Légzés}\label{luxe9gzuxe9s-1}}

A lélegzet érzéseit figyeljük a testben, ez az éber figyelem összeszedi
és megnyugtatja az elmét. Figyelünk, mit érzünk a testben, amikor
lélegzetet veszünk, mit érzünk amikor kiengedjük a levegőt.

Belégzéskor, a hideg levegőt először az orrhegynél érezzük, megtölti a
tüdőt, miközben a mellkas kitágul, a hasizom kissé behúzódik. Nem kell
ezt irányítanunk. A test magától is tudja hogyan kell lélegezni,
számunkra most elég megfigyelnünk, mintha hullámokat figyelnénk ahogy
bejönnek a partra, majd visszahúzódnak.

Nem kell megmondanunk magunknak mit gondoljunk és mit érezzünk. Ha
tiszta gondolatokat akarunk, legjobb először csendben lenni és figyelni.
Csendben ülünk egy kis ideig, és mikor elhallgatunk, a tiszta gondolatok
maguktól fognak jönni, vagy az elme elégedett lesz a csenddel.

A tiszta, tudatos gondolatot a békés öröm követi, az elme elégedett és
boldog, nem érzi szükségét a sok belső párbeszédnek, elég öröme van
abban is, hogy a csended hallgatja.

A helyes hozzáállás óvatos egyensúlyt igényel. Megalapozzuk a tiszta
szándékot, hogy a meditációval maradunk és más ügyeket későbbre hagyunk,
de a túl sok erő merevvég és akadályozóvá válik.

A tiszta elme és jó elhatározás érzése nyugodt és hűvös, nyitott a
változásra. A akaraterővel küzdés érzése elfoglalt és forró, szűk a
látótere.

Emlékszem, egy időben úgy gondoltam, hogy meditáció közben azért
figyeljük a légzést, mert ebből új dolgokat kell tanuljunk. Nekiültem és
küszködtem vele, nem értettem hogy kellene ezt csinálni, folyton ezen
gondolkodtam és a légzésen változgattam, arra számítva, hogy végül
valahogy majd csak menni fog, és egyszer, a helyes módon lélegezve új
tényeket ismerek meg az elméről. Elég fájdalmas folyamat volt, és
jobbára eredménytelen.

A legjobb, amikor olyan dolgot kell megtanulnunk, amire nem is
vállalkoztunk. Nem tanulunk új tényeket a meditációból, mert olyan kevés
tényre van szükségünk, hogy valószínűleg már mindet ismerjük.

De nem állunk meg, hogy velük maradjunk, és kicsomagoljuk a jelentésüket
arra, hogy \emph{mit tegyünk}, \emph{milyen módon tegyük azt}, vagy más
szóval \emph{hogyan legyünk}. Egy sor tény, ha nem építjük be őket, nem
érnek le elég mélyre, hogy a szív és elme tapasztalatainak gyökereit
kezeljék, és nincsenek ránk hatással. A lélegzet figyelése megállít
minket, és megnyílik a figyelmünk, ami képes ezt elérni.

A Buddhának csak egy egyszerű üzenete van számunkra: ébredj fel, ne
ragaszkodj, nem kell szenvedned. Ezt csomagoljuk ki, egyre szélesebbre
tárva.

Szánj rá pár percet, hogy igazítsd ahogy ülsz, és megtaláld a
kiegyensúlyozott testtartást. A fontos szempont az, hogy a tartásod
egyenes legyen, stabil de nem feszes helyzetben, a fej legyen
egyensúlyban és ne dőljön előre, és a tartásod engedje a nyitott, könnyű
légzést.

Határozd el, hogy erre az időre félreteszed a mindennapi
tevékenységeket. Ha megszakít egy ilyen gondolat, válaszolj, 'Ez nem a
megfelelő idő, később vissza fogok erre térni amikor az idő alkalmas rá.
A jelen helyzetben, a testet és a lélegzetet figyelem.' Hosszabb, mint
az, hogy 'Csend legyen!', de barátságosabb magunkkal szemben. Segít, ha
a gondolatot szándékosan kifejezzük gondolatban. Ez megalapoz egy tiszta
szándékot az elmében, mint elpakolni az asztalról mielőtt dolgozni
kezdünk. Nem azért tesszük ezeket félre, mert nem fontosak, hanem mert
jól akarjuk ellátni a feladatainkat. Ha nem pihenünk, nem tudunk jól
dolgozni sem.

Vegyél egy mély lélegzetet, és figyeld, hogy érzel-e feszültséget,
valamit ami akadályozza vagy korlátozza a légzést. Ha könnyen és
nyitottan mozog, a testtartásod megfelelő. Nem kell különleges módon
ülnöd.

Figyelj a légzés testi érzéseire. Engedd, hogy a test szabályozza a
légzést, mi csak figyeljük és hagyjuk ellazulni, arra figyelve ami éppen
történik.

A jó testtartás és a nyugodt, könnyű légzés egy csendes és örömteli
érzés, mint leülni a parkban egy padra egy séta után. Semmi különös
dolgunk nincs, és ez az egyszerű, csendes ülés magában is öröm.

Belégzéskor, a levegőt először az orrhegynél érezzük. A hideg levegő
lefelé mozog a légcsövön, megtölti a tüdőt, és a mellkas kitágul. A has-
és rekeszizom ki- és be mozog, irányítva a levegő mozgását. A szívverés
halk ritmusa érezhető közben. Kilégzéskor, az izmok ellazulnak, a
mellkas összehúzódik, és a borda csontok közelebb zárulnak. A meleg
levegő felfelé áramlik a légcsövön, és elhagyja a testet az orron át.

Nem szükséges ezt gondolatban kifejezni, lazíts és nézd ahogy az érzések
megjelennek a testben. Eltart pár percig, amíg a test megállapodik. A
szívverés lecsillapodik, és a légzés egyenletes és könnyű lesz.

Hagyd, hogy a test maga szabályozza a légzést. Amikor egy véleménnyel
állunk hozzá, hogy a légzésünk rövid legyen, vagy hosszú, az merevvé és
erőltetetté válik. Fel akarjuk fedezni a tapasztalatainkat, nem
megszabni, mik legyenek azok.

A test jobban tudja hogyan kell lélegezni, mint mi. Nagyon jól tudja a
lélegzést végezni számunkra, ha hagyjuk neki. Ahelyett, hogy próbálnád
kitalálni, hogy helyesen lélegzel vagy sem, lépj egyet vissza és
fordítsd meg a figyelmet, hallgatózz ahelyett, hogy utasítasz. Belégzés,
kilégzés, mit érzel a testben?

Nincs semmilyen meghatározott dolog, amit tapasztalnod kell. Ehelyett a
szándék az, hogy legyen időd, és engedj teret annak, hogy a
tapasztalataiddal maradj.

Egyensúlyban önmaga középpontjában, ismerni a jelen pillanat
egyszerűségét. Ha úgy érzed, hogy valamit teljesítened vagy javítanod
kell, ez mindig egy hozzáadott dolog, valami amit mi hozunk létre. Mi
hozzuk létre az elvárást, hogy változtatnunk kell, ki kell valamit
javítanunk, irányítanunk kell. Vedd észre ezt a kényszert, és ismerd
fel, hogy el tudod engedni, nem kell azt tenned.

Ha sok kusza gondolat jár a fejedben, határozd el mit fogsz gondolni,
ahelyett, hogy hagynád az elmét körbe-körbe járni. Például használd a
BUD-DHO mantrát, ami azt jelenti, 'aki megismer'. A belégzéskor, gondold
BUD-, kilégzéskor, -DHO. Ha már erős lendületet gyűjtöttünk a
gondolkodással, és az nem hajlandó lecsillapodni, ez oldalkorlátot és
fekvő-rendőrt rak le, hogy az úton maradjunk és lassítsunk.

Belélegzünk, maradunk a jelen egyszerű tapasztalatával, és ennyi elég.

Kényszereket érzünk, vágyakat és aggodalmakat, úgy érezzük, 'erre
szükségem van', 'én ilyen vagyok', 'olyannak kellene lennem' -- ezeket
éberen szemlélni tudjuk. A légzéssel maradunk, és figyelmünket a
tapasztalat felé fordítjuk, ami éppen történik.

A testi éberség egy szilárd alap, megnyugtató és átrendezi mi az
értékes. Ha a tapasztalatod békés, boldog és elégedett, maradj vele.
Nincs abban semmi rossz. Ez egy olyan boldogság, ami nem kötődik a
ragaszkodáshoz, nem függ attól, hogy megszerezzünk vagy elérjünk
valamit. Ez a boldogság az érzékek elvonultságából ered, visszatérve az
egyszerűséghez, megismeri és együtt marad a jelennel. Az elme éber,
nyugodt és elégedett.

A meditáció a kavargó érzelmeket is a felszínre hozza, és ez jól van
így. Azt látjuk, amit eddig nem engedtük magunknak, hogy lássuk. Nincs
szükség arra, hogy válaszokat és megoldásokat keressünk, az érzelmeket
nem a személyes történetek szintjén vizsgáljuk, hanem alapvetőbb
szinten, mint az elme és szív állapotait. Azon a szinten nincsenek
történetek, az érzés, az elmeállapot nem mondja meg magáról kicsoda és
mit gondol rólunk, ezt mi adjuk hozzá.

A nagylelkűség ellazítja az elmét, a moralitás pedig stabil alapot ad.
Gondolhatunk jó tettekre, amit adtunk és kaptunk, emlékezhetünk azokra,
akikre jó példaként tisztelttel nézünk fel.

Ha azt veszed észre, hogy feszült vagy, szigorú és cinikus a hangulatod,
azt javaslom igazítsd a testtartásod, hogy kicsit lazább legyen,
csendben dörzsöld meg a füleid, vagy masszírozd meg az arc izmokat az
ujjaiddal, és emlékezz a nagylelkűségre. A kolostorban gyakran a világi
barátaink azok, akik eljönnek főzni és felajánlani a napi ebédet a
közösség számára. Nagy a sürgés-forgás amíg a konyhában vannak, de
amikor már végeztek, könnyebbek, lazítanak és mosolyognak.

Felidézni jó tetteinket, egyszerű kis dolgok esetében is, ellazítja az
eredményekre szomjas elmét. Képzeld el mit történne, ha valaki
százszorosan megadná neked amire szükséged van. Akkor hogy meditálnál?
Valószínűleg közel úgy mint most, csak lazábban. Add meg magadnak azt a
gazdag, tágas teret.

A nagylelkűség enged felismerni, hogy van elég terünk, nem kell
erőlködnünk, hogy mások elé jussunk, van jóság a világban és
felhagyhatunk a nagy sietséggel. Örömteli érzés felidézni a családunk,
rokonaink és barátaink nagylelkűségét is, de még amikor egy ismeretlen
embert látunk segíteni egy másik ismeretlennek, az is előcsal egy
mosolyt.

'Hogy tudom megcsinálni?' Közelítsd meg másként, és inkább azt kérdezd,
'Tudok rá figyelni?'

Ismerni csak azután fogod, hogy a kezdeti bizalom engedett gyakorolni,
és azt tudod majd ismeretként kifejezni, ami már mögötted van. Ami jön,
azt nem tudod precízen leírni, a gondolat erre nem elegendő.

A légzés érzete megállít. Visszakerülünk az elejére, ahol nem tudjuk mi
lesz. Egy üres és tágas helyre kerülünk így, ahol magunk vagyunk és van
időnk ott megállni.

A légzésre figyelve az érzékek befelé fordulnak. A szem látja a
színeket, de a látás befelé irányul, nem akar kívül színeket és formákat
keresni. A fül hallja a hangokat, de a hallás befelé fordul és nem
keres. A test érzi a hideget, meleget, a ruhák felületét és a csontok
merev súlyát. A légzés közben figyeljük ezt és hagyjuk a testet
lenyugodni, hagyjuk az elmét befelé fordulva elcsendesedni.

Mint egy tó, aminek nincsenek ki- és bevezető folyásai, határait körben
a völgy szabja meg, csupán egyetlen, a földből feltörő forrásból kap
hűvös, friss vizet. Amikor eső esik, némi víz kis erekben a tóba fog
folyni, de mivel nincs kifolyása, mind a tóban fog megállni és a völgy
határt szab neki. A tó vize nyugodt marad, és a forrás hűvös vize az
egész tóban el fog terjedni, annak minden részét áthatva.

Az érzés és tudat a testtől függ, nem tudunk hozzátenni vagy elvenni
belőle. Minden légzésben teljes, a testtel kezdődik és azzal lesz vége.
Ez a világ, ami érzésekből áll, ebben teljes -- minden ami vagyunk, vagy
amivé valaha válhatunk, ezen belül van.

Amikor szenvedünk, tudjuk, hogy van itt valami, amit nem értünk. Nem
értjük, hogy egy dolog hogyan jött létre a másikból, hogy egy dolog az
irányításunk alatt van, a másik pedig nincs.

Amikor nem látjuk, ugyanazt a mintát ismételjük, mint egy programot, és
ugyanazt a szenvedést újra és újra létrehozzuk. Panaszkodunk, hogy
'Miért van ez mindig így?' Ugyanazt a dolgot tesszük újra és újra, és
nem látjuk.

Közelebbről megnézve látjuk, hogy az egyik dolog a másiktól függ. Akkor
látjuk a lehetőséget, hogy szabadon abbahagyhatjuk. Így visszatérünk a
csendes elégedettséghez.

Amikor már egy ideje meditálva ülünk, gyakran elkezdjük bonyolítani a
dolgot. Honnan jön ez, hogy nem tudunk egy egyszerű dologgal együtt
maradni? Figyeld meg, ahogy az egyszerűbe vetett hit megváltozik,
elkezdünk valamilyen szempontról gondolkodni, és a kétség és
ön-kritizálás megállít mindent.

Komikus, hogy mennyire elkötelezetten tudjuk kritizálni magunkat, mintha
egy túlemelkedett élmény lenne az, hogy fájdalmat okozunk magunknak. De
úgy érezzük, erőlködnünk kell \emph{valamin}, meg kell törjük az egónkat
és el kell engedjünk mindent. Talán ez az egyetlen út amit ismerünk, nem
is tudjuk milyen lehet nem ilyennek lenni.

Az elején megvan az önmagunkkal szembeni kedves és rugalmas
hozzáállásunk, de a végén csak keménység és bírálat marad. A fiatal fa
friss és rugalmas, könnyen hajlik ahogy nő, de az öreg fa kemény és
száraz amikor elpusztul.

Térj vissza az elejére, ahol megvan a kezdővel szembeni kedvesség, ahol
még nem vártad el magadtól, hogy tudnod kell. Nem tudjuk, mi van itt,
amíg meg nem nézzük, hogy lássunk. Ez a látás és figyelem a friss
megismerés. Engedd magadnak, hogy mindig az elején legyél.
