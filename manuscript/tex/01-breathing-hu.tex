\hypertarget{luxe9gzuxe9s-1}{%
\chapter{Légzés}\label{luxe9gzuxe9s-1}}

'Hogy tudom megcsinálni?' Nem ez a feladatod. Kérdezd azt, 'Tudok rá
figyelni?'

Ismerni csak azután fogod, hogy a kezdeti bizalom engedett gyakorolni,
és azt tudod majd kifejezni ami már mögötted van. Ami jön, azt nem tudod
ismerettel leírni, a gondolat erre nem elegendő.

A légzés érzete megállít. Visszakerülünk az elejére, ahol nem tudjuk mi
lesz. Egy üres és tágas helyre kerülünk így, magunk vagyunk és van időnk
itt megállni.

Az üvegből folyó víz megáll, amikor az üveget egyenesen letesszük, és az
üvegben a víz a súlyánál fogva a helyére kerül. A mozgás látszik rajta
keresztül, de önmaga nem mozog.

A légzésre figyelve az érzékek befelé fordulnak. A szem látja a
színeket, de a látás befelé irányul, nem akar kívül színeket és formákat
keresni. A fül hallja a hangokat, de a hallás befelé fordul és nem
keres. A test érzi a hideget, meleget, a ruhák felületét és a csontok
merev súlyát, a légzés közben figyeljük ezt és hagyjuk a testet
lenyugodni, hagyjuk az elmét befelé fordulva elcsendesedni.

Ez az érzés világa ebben teljes, minden ami vagyunk, vagy amivé valaha
válhatunk, ezen belül van. Az érzés és tudat a testtől függ, nem tudunk
hozzátenni vagy elvenni belőle. Minden légzésben teljes, a testtel
kezdődik és azzal lesz vége.

Az érzés és tudat a testtől függ, nem tudunk hozzátenni vagy elvenni
belőle. Minden légzésben teljes, a testtel kezdődik és azzal lesz vége.
Ez világ, ami érzésekből áll, ebben teljes, minden ami vagyunk, vagy
amivé valaha válhatunk, ezen belül van.

Amikor szenvedünk, tudjuk, hogy van itt valami, amit nem értünk. Nem
értjük, hogy egy dolog hogyan jött létre a másikból, hogy egy dolog az
irányításunk alatt van, a másik pedig nincs.

Amikor nem látjuk, ugyanazt a mintát ismételjük mint egy program, és
ugyanazt a szenvedést újra és újra létrehozzuk. Panaszkodunk, hogy
'Miért van ez mindig így?' Ugyanazt a dolgot tesszük újra és újra, és
nem látjuk.

Közelebbről megnézve látjuk, hogy az egyik dolog a másiktól függ. Akkor
szabadon abbahagyhatjuk. Így visszatérünk a csendes elégedettséghez.
