\hypertarget{ciklusok-1}{%
\chapter{Ciklusok}\label{ciklusok-1}}

A meditáció a jelenbeli érzéseken, tapasztalatokon keresztüli
megismerést tanítja.

Az utasítások lépésekben épülnek fel, de nem ezek a lépések a meditáció
célja, hanem a jelen tapasztalat tiszta ismerete, a helyes nézőpont
visszaállítása. Kialakulhat bennünk az a benyomás, hogy mindig ugyanazt
a lépés sort kell teljesítenünk, és amikor az elme nem aszerint a
sorrend szerint fejlődik, csalódottak vagyunk.

Fordístd meg ezt a hozzáállást és kezd a tapasztalattal. Ha a
tapasztalat az alap, ahogy a dolgok vannak, az milyen megértést ad
nekünk? Amikor falat festünk, először megvizsgáljuk a falat,
kiválasztjuk hozzá a megfelelő festéket, és \emph{azután} követjük a
tanácsokat a dobozon. A rossz festék csak le fog válni, nem? Először az
elmét vizsgáljuk, hogy érezzük magunkat, milyen állapotban vagyunk, és
arra válaszolunk intelligensen.

A meditáció a jelenbeli érzéseken, tapasztalatokon keresztüli
megismerést tanítja.

A lépések az imitációs tanulás részei, egy példát követve figyeljük
önmagunkat és meglátjuk mi hogy működik. Amikor fájdalmat és szenvedést
érzünk, és vagy fel tudjuk oldani, vagy türelmesen kivárni, akkor
tudjuk, hogy ez nem csupán imitáció volt, és a gyakorlásba vetett
bizalmunk erősödni fog. Ebből tanultunk valamit, és nem ragaszkodunk az
első bemutató részleteihez.

It would be great if our meditation would develop in a linear
way\ldots{}
