\documentclass{article}
\usepackage{dia-page}
\geometry{paperwidth=125mm, paperheight=165mm, vmargin=1pt, hmargin=1pt, nohead, nofoot}

\begin{document}

\centering

\begin{tikzpicture}[font=\diaSmall, node distance=2mm]
  \node (associations) [tight ring]
  {\parbox{20mm}{\centering
      Támogató viszonyok
    }};

  \node [below=5mm of associations.south]
  {\parbox{30mm}{\centering\diaTiny
      Kedvező környezet,\\
      stabil alap a feladat elvégzéséhez
    }};

  \node (relations) [tight ring, left=15mm of associations]
  {\parbox{20mm}{\centering
      Morális megfontolások
    }};

  \node [below=5mm of relations.south]
  {\parbox{30mm}{\centering\diaTiny
      Feladatok és kapcsolatok:\\
      Hogyan kezdjem?\\
      Mik az akadályok?\\
      Kivel dolgozzak?
    }};

  \node (concentration) [tight ring, right=15mm of associations]
  {\parbox{20mm}{\centering
      Összpontosítás és Bölcsesség
    }};

  \node [below=5mm of concentration.south]
  {\parbox{30mm}{\centering\diaTiny
      A feladatra irányuló folytonos figyelem és vizsgálódás fejlesztése
    }};

  \draw [smallish arrow] (relations) to (associations);
  \draw [smallish arrow] (associations) to (concentration);


  \node (iddhipada) [below=28mm of associations.south, anchor=north]
  {\begin{minipage}{\linewidth}
      \centering\setlength{\parskip}{5pt}
      A Siker Négy Útja (\emph{iddhipāda})

      {\diaTiny Az elme tényezői, amik fejlesztik az összpontosítást\\
        és a cél elérése felé vezetnek\par}
    \end{minipage}};

  \node (chanda) [tight box, below right=7mm and 10mm of iddhipada.south west]
  {\parbox{25mm}{\centering
      Törekvés, Lelkesedés\\ (\emph{chanda})
    }};

  \node [right=7mm of chanda.north east, anchor=north west]
  {\parbox{65mm}{\raggedright\diaTiny
      Erős érdeklődés, elhivatott szeretet az ember munkája és annak célja irányában.
      Vágyni a cél teljesítésére, és megteremteni annak jó hatásait.
      Boldogság, öröm, elégedettség a cél elérésében.
    }};

  \node (viriya) [tight box, below=7mm of chanda]
  {\parbox{25mm}{\centering
      Energikus Erőfeszítés\\ (\emph{viriya})
    }};

  \node [right=7mm of viriya.north east, anchor=north west]
  {\parbox{65mm}{\raggedright\diaTiny
      Bátorság, erőfeszítés, kitartás,
      nem engedni, hogy az akadályok és nehézségek
      eltántorítsák vagy megfélemlítsék az embert.
      Az elme összhangban és stabilan dolgozik a cél megvalósításán.
    }};

  \node (citta) [tight box, below=7mm of viriya]
  {\parbox{25mm}{\centering
      Összpontosított Figyelem\\ (\emph{citta})
    }};

  \node [right=7mm of citta.north east, anchor=north west]
  {\parbox{65mm}{\raggedright\diaTiny
      Az elme elmerül a tevékenységben és összpontosít annak tárgyára;
      nem engedi el a figyelem tárgyát, és nem válik szétszórttá.
      Az ember lehet, hogy megfeledkezik a környezetéről és nem veszi észre az idő múlását.
    }};

  \node (vimamsa) [tight box, below=7mm of citta]
  {\parbox{25mm}{\centering
      Vizsgálat\\ (\emph{vimamsā})
    }};

  \node [right=7mm of vimamsa.north east, anchor=north west]
  {\parbox{65mm}{\raggedright\diaTiny
      Analízis és vizsgálat, ami az elmét összefogja, és követi a jelen feladat haladását.
      Kérdéseket feltenni a kiváltó tényezőkről és tervet formálni.
      Érvelni és elmélkedni; vizsgálni a saját tetteinkben fellelhető hibákat;
      kísérletezni és keresni annak módját, hogy javítsuk önmagunkat.
      Itt a bölcsesség képessége irányítja az összpontosítást.
    }};

\end{tikzpicture}%

\end{document}
