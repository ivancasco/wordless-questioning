\documentclass{article}
\usepackage{dia-page}
\geometry{paperwidth=120mm, paperheight=180mm, vmargin=1pt, hmargin=1pt, nohead, nofoot}

\begin{document}

\centering

\begin{tikzpicture}[font=\diaSmall, node distance=2mm]
  \node (realm) [tight ring]
  {\parbox{20mm}{\centering Létezés az érzékek világában}};

  \node (unwise) [tight box, below=22mm of realm]
  {\parbox{38mm}{\centering felületes, balga vizsgálódás\\ \emph{a-yoniso manasikāra}}};

  \node (reactions) [tight box, below=22mm of unwise]
  {\parbox{38mm}{\centering a mohóság, gyűlölet és zavar reakciói irányítják az életcélokat és tetteket}};

  \node (lack) [below left=5mm and 4mm of reactions.south, anchor=north east]
  {\parbox{38mm}{\centering\diaTiny
      Hiányos irányítás a mentális élet felett,
      a szomjas vágy táplálja a félelmet és feszültséget}};

  \node (taints) [tight box, below=23mm of reactions]
  {\parbox{38mm}{\centering kényszeres hajlamok (\emph{āsava}) erősödnek}};

  \node (suffering) [tight ring, below=21mm of taints]
  {\parbox{20mm}{\centering Szenvedéshez, bánathoz és elkeseredéshez vezet}};

  \node (group) [rectangle, left=10mm of unwise, minimum width=60mm, minimum height=15mm, draw] {};

  \node (fetters) [tight box, below right=2mm and 2mm of group.north west, minimum height=2\baselineskip]
  {\parbox{20mm}{\centering Tíz Béklyó}};

  \node (hindrances) [tight box, below left=2mm and 2mm of group.north east, anchor=north east, minimum height=2\baselineskip]
  {\parbox{25mm}{\centering Öt Akadály}};

  \node [above left=10mm and 3mm of fetters.north west, anchor=south west]
  {\parbox{26mm}{\raggedright\diaTiny\setlength{\parskip}{5pt}
      Béklyók kötik a születés-és-halálhoz,\\
      az első három tartja fenn a helytelen megértést

      (1)~személyes nézet,\\
      (2)~kétség,\\
      (3)~fogadalmak és rituálék megragadása
    }};

  \node [above left=10mm and 0mm of hindrances.north west, anchor=south west]
  {\parbox{26mm}{\raggedright\diaTiny\setlength{\parskip}{5pt}
      Akadályok megállítják a Helyes Erőfeszítést és Bölcs Vizsgálódást

      (1)~érzéki vágy,\\
      (2)~rosszindulat,\\
      (3)~tunyaság,\\
      (4)~nyugtalanság,\\
      (5)~kétség
    }};

  \node (thinking) [below=18mm of group]
  {\parbox{40mm}{\centering\diaTiny
      Így gondolkodik:\\
      Ki vagyok?
      Én valaki vagyok, akinek valamit tennie kell.
      Ez a nekem megfelelő dolog?
      Ezt így kell tenni, csak ez a helyes módja.
    }};

  \node (noble) [tight ring, below=13.5mm of thinking]
  {\parbox{20mm}{\centering Nem látja\\ a Négy Nemes Igazságot}};

  \node (reinforces) [below left=7mm and -5mm of noble]
  {\parbox{25mm}{\centering\diaTiny
      támogatja\\ a személyes nézetet,\\
      a ragaszkodást a születés-és-halálhoz
    }};

  \node (greed) [below=20mm of noble]
  {\parbox{30mm}{\centering\diaTiny
      mohóság: én ezt szeretem\\
      gyűlölet: én ezt utálom\\
      zavar: én ez vagyok,\\
      ez az enyém
    }};

  \draw [smallish arrow] (realm) to (unwise);
  \draw [smallish arrow] (unwise) to (reactions);
  \draw [smallish arrow] (reactions) to (taints);
  \draw [smallish arrow] (taints) to (suffering);

  \draw [smallish arrow] (taints) to (greed);
  \draw [smallish arrow] (greed) to (noble);
  \draw [smallish arrow] (noble) to (taints);

  \draw [line] (group) to (thinking);
  \draw [smallish arrow] (thinking) to (noble);

  \draw [smallish arrow, bend left=40] (group.north) to (unwise.north);
  \draw [smallish arrow, bend left=40] (unwise.south) to (group.south);

\end{tikzpicture}%

\end{document}
