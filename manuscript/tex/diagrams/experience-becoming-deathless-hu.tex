\documentclass{article}
\usepackage{dia-page}
\geometry{paperwidth=115mm, paperheight=145mm, vmargin=1pt, hmargin=1pt, nohead, nofoot}

% Experience, Becoming, and the Deathless

\begin{document}

\centering

\begin{tikzpicture}[font=\diaSmall, node distance=2mm]

  \node (conceiving)
  {\parbox{100mm}{\diaTiny\centering
      `Én ez vagyok' egy vélekedés (\emph{maññita}).\\
      ``A vélekedés kór, a vélekedés fekély, a vélekedés tövis.'' (MN 140)}};

  % === Grasping side

  \node (grasping) [tight box, below left=5mm and 3mm of conceiving.south]
  {\parbox{45mm}{\centering khandhákat és érzék-tapasztalatot\\ megragadja mint ami egy személy}};

  \node (person) [below=5mm of grasping]
  {\parbox{45mm}{\diaTiny\raggedright
      Személyes nézetből, az életet létesülésként tapasztalja.
      A születés, öregség és halál a személyre vonatkozik.\\[1em]
      A halál gondolata feszültséget, kétségbeesést,
      veszteség félelmét, jelentésbeli kétséget okoz.\\[1em]
      Az elme reagál, elterelve figyelmét a pozitív vagy negatív létesüléssel.
    }};

  \node (bhava) [tight box, below left=10mm and 3mm of person.south]
  {\parbox{18mm}{\centering pozitív\\ létesülés\\ (\emph{bhava})}};

  \node (vibhava) [tight box, below right=10mm and 3mm of person.south]
  {\parbox{18mm}{\centering negatív\\ létesülés\\ (\emph{vibhava})}};

  \node (bhava-notes) [below=5mm of bhava]
  {\parbox{28mm}{\diaTiny\raggedright
      Ha meg tudnám szerezni\ldots{}\\
      Mikor leszek olyan\ldots{}\\
      Ez ezt jelenti.\\
      Én ez vagyok.
    }};

  \node (vibhava-notes) [below=5mm of vibhava]
  {\parbox{30mm}{\diaTiny\raggedright
      Ha meg tudnék\\ szabadulni tőle\ldots{}\\
      Mikor nem leszek olyan\ldots{}\\
      Mi az értelme?\\
      Ki vagyok én?
    }};

  \node (becoming) [tight box, below=55mm of person]
  {\parbox{35mm}{\centering mindkettő megerősíti, hogy `én vagyok',\\ és a létesülés folytatódik}};

  \draw [line] (grasping) to (person);
  \draw [smallish arrow] (person) to (bhava);
  \draw [smallish arrow] (person) to (vibhava);
  \draw [line] (bhava) to (bhava-notes);
  \draw [line] (vibhava) to (vibhava-notes);
  \draw [smallish arrow] (bhava-notes) to (becoming);
  \draw [smallish arrow] (vibhava-notes) to (becoming);

  %\draw [line] (stopping) to (training);

  % === Non-Grasping side

  \node (non-grasping) [tight box, below right=5mm and 3mm of conceiving.south]
  {\parbox{45mm}{\centering khandhákat és érzék-tapasztalatot\\ úgy látja, ahogy valóban van}};

  \node (deathless) [below=5mm of non-grasping]
  {\parbox{45mm}{\diaTiny\raggedright
      Megragadás nélkül, nem gyárt személyes nézeteket:
      eltávolítva azt, amire a halál vonatkozik,
      a tapasztalat születés nélküli, öregség nélküli, haláltalan.\\[1em]
      ``Aki minden vélekedést elvetett, őt nevezik megbékélt bölcsnek. /
      Felismerte a haláltalant, azt testileg tapasztalva éli életét.''
    }};

  \node (awareness) [tight box, below=5mm of deathless]
  {\parbox{32mm}{\centering a tudatosság megállítja a kényszeres reakciókat}};

  \node (training) [below=5mm of awareness]
  {\parbox{42mm}{\diaTiny\centering Képzi magát az emelt erényben,\\ emelt elmében, emelt bölcsességben}};

  \node (transcendent) [tight box, below=5mm of training]
  {\parbox{32mm}{\centering önmeghaladó értékek}};

  \node (transcendent-notes) [below=5mm of transcendent]
  {\parbox{46mm}{\diaTiny\raggedright
      Nem kényszerül arra, hogy az életet személyes nézőpontból lássa.
      A személy szokásos koncepcióját
      magára veheti vagy leteheti ragaszkodás nélkül,
      ahogy a mindennapi helyzetek megkívánják.
    }};

  \node (non-becoming) [tight box, below=5mm of transcendent-notes]
  {\parbox{32mm}{\centering nincs további létesülés}};

  \draw [line] (non-grasping) to (deathless);
  \draw [smallish arrow] (deathless) to (awareness);
  \draw [line] (awareness) to (training);
  \draw [smallish arrow] (training) to (transcendent);
  \draw [line] (transcendent) to (transcendent-notes);
  \draw [smallish arrow] (transcendent-notes) to (non-becoming);

\end{tikzpicture}%

\end{document}
