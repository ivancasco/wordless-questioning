\documentclass{article}
\usepackage{dia-page}
\geometry{paperwidth=125mm, paperheight=151mm, vmargin=1pt, hmargin=1pt, nohead, nofoot}

\begin{document}

\centering

\begin{tikzpicture}[font=\diaSmall, node distance=2mm]
  \node (center-rect) [
    rectangle,
    minimum width=\linewidth-20mm,
    minimum height=0pt,
    inner sep=0pt,
    line width=0pt,
  ] {};

  \node (line-a) [left=9mm of center-rect.north] {};
  \node (line-b) [left=0pt of center-rect.west] {};

  \draw [dashed line] (line-a) to (line-b);

  \node [above=0mm of line-b, anchor=south west] {Belülre irányuló};

  \node [above=10mm of line-b, anchor=south west]
  {\parbox{33mm}{\raggedright\diaNarrow
      Szabadság a ragaszkodás formáitól.
      A problémákat belső vizsgálódás oldja meg.
      Megérteni az időtlen feltételeket amik felett nincs irányításunk.
    }};

  \node [below=0mm of line-b, anchor=north west] {Kívülre irányuló};

  \node [below=10mm of line-b, anchor=north west]
  {\parbox{33mm}{\raggedright\diaNarrow
      Megszerezni és megőrizni.
      A problémákat külső tényezők irányítása oldja meg.
      Hasznos módszerek alkalmazása, jótékony választásokat hozni,
      amennyire van lehetőségünk az irányításra.
    }};

  \node (tu0) [
    isosceles triangle,
    shape border rotate=270,
    above=2mm of center-rect,
    anchor=apex,
    minimum width=6cm,
    % isosceles triangle stretches,
    % minimum width=5cm,
    % minimum height=7cm,
    line width=0.4pt,
    draw,
  ] {};

  \foreach \idx/\h in {tu1/4.8, tu2/3.6, tu3/2.4, tu4/1.2}{
    \node (\idx) [
      isosceles triangle,
      shape border rotate=270,
      above=2mm of center-rect,
      anchor=apex,
      minimum width=\h cm,
    ] {};
    \draw (\idx.left corner) to (\idx.right corner);
  }

  \node (tb0) [
    isosceles triangle,
    shape border rotate=90,
    below=2mm of center-rect,
    anchor=apex,
    % isosceles triangle stretches,
    minimum width=6cm,
    line width=0.4pt,
    draw,
  ] {};

  \foreach \idx/\h in {tb1/4.8, tb2/3.6, tb3/2.4, tb4/1.2}{
    \node (\idx) [
      isosceles triangle,
      shape border rotate=90,
      below=2mm of center-rect,
      anchor=apex,
      minimum width=\h cm,
    ] {};
    \draw (\idx.left corner) to (\idx.right corner);
  }

  \node [below=7mm of tu0.north, anchor=mid] {Szabadság a Létesüléstől};

  \node [below right=7mm and 20mm of tu0.north, anchor=west, fill=white]
  {\parbox{39mm}{\raggedright\diaNarrow
      elengedni az `én vagyok' gondolatát,
      szembenézni a halandósággal,
      menedék a haláltalanban
    }};

  \node [below=7mm of tu1.north, anchor=mid] {Bizonytalanság};

  \node [below right=7mm and 20mm of tu1.north, anchor=west, fill=white]
  {\parbox{39mm}{\raggedright\diaNarrow
      elengedni a tulajdonok biztonságát,
      moralitás önmagunk felett,
      nyitottság az ismeretlenre
    }};

  \node [below=7mm of tu2.north, anchor=mid] {Egyetemesség};

  \node [below right=7mm and 20mm of tu2.north, anchor=west]
  {\parbox{39mm}{\raggedright\diaNarrow
      elengedni a kulturális identitást,\\
      nem `mi' és `ők',
      elhagyni a mohóságot és gyűlöletet,
      ön-elfogadás
    }};

  \node [below=7mm of tu3.north, anchor=mid] {Alázatosság};

  \node [below right=7mm and 20mm of tu3.north, anchor=west]
  {\parbox{39mm}{\raggedright\diaNarrow
      elengedni a társadalmi elismerést,
      elfogadni többféle véleményt,
      bizalom a Dhamma gyakorlásában
    }};

  \node [below=5mm of tu4.north, anchor=mid, fill=white] {Ön-meghaladás};

  \node [below right=4mm and 20mm of tu4.north, anchor=west]
  {\parbox{39mm}{\raggedright\diaNarrow
      elengedni a személyes eredményeket,
      látni a mulandóságot,
      nem keresni a tökéletességet
    }};

  \node [above=5mm of tb4.south, anchor=mid, fill=white] {Ön-megvalósítás};

  \node [above right=4mm and 20mm of tb4.south, anchor=west]
  {\parbox{39mm}{\raggedright\diaNarrow
      moralitás, kreativitás,\\ képességek elsajátítása, beteljesíteni a személyes lehetőségeket
    }};

  \node [above=7mm of tb3.south, anchor=mid] {Önbecsület};

  \node [above right=7mm and 20mm of tb3.south, anchor=west]
  {\parbox{35mm}{\raggedright\diaNarrow
      társadalmi elismerés,\\ mások általi tisztelet,\\ önbizalom
    }};

  \node [above=7mm of tb2.south, anchor=mid] {\parbox{35mm}{\centering Szeretet és\\ Odatartozás}};

  \node [above right=7mm and 20mm of tb2.south, anchor=west]
  {\parbox{35mm}{\raggedright\diaNarrow
      család, barátságok,\\ kapcsolat egy\\ bizalmas csoporttal
    }};

  \node [above=7mm of tb1.south, anchor=mid] {Biztonság};

  \node [above right=7mm and 20mm of tb1.south, anchor=west, fill=white]
  {\parbox{35mm}{\raggedright\diaNarrow
      testi biztonság, rend,\\ megélhetés, egészség
    }};

  \node [above=7mm of tb0.south, anchor=mid] {Túlélési Szükségletek};

  \node [above right=7mm and 20mm of tb0.south, anchor=west, fill=white]
  {\parbox{35mm}{\raggedright\diaNarrow
      étel, ruházat,\\ lakhely, gyógyszer
    }};

  \node (process) [
    circle,
    left=0pt of center-rect.north,
    xshift=0.75cm,
    minimum width=1.5cm,
    % Not drawing because this shape is for positioning the arrow.
    % color=red,
    % draw,
  ] {};

  \draw (process) node [xshift=-1pt] {
    \tikz [x=0.75cm,y=0.75cm,smallish arrow] \draw (0,0) arc (10:350:1 and -1);
  };

  \node [tight box, right=12mm of center-rect.north] {\parbox{38mm}{\raggedright\diaNarrow
      Gyakorolni a tiszta megértést,
      helyes erőfeszítés az adott helyzetben
      a \emph{dukkha} megszűnéséért.
    }};

\end{tikzpicture}%

\end{document}
