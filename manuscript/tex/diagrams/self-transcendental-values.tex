\documentclass{article}
\usepackage{dia-page}
\geometry{paperwidth=125mm, paperheight=151mm, vmargin=1pt, hmargin=1pt, nohead, nofoot}

\begin{document}

\centering

\begin{tikzpicture}[font=\diaSmall, node distance=2mm]
  \node (center-rect) [
    rectangle,
    minimum width=\linewidth-20mm,
    minimum height=0pt,
    inner sep=0pt,
    line width=0pt,
  ] {};

  \node (line-a) [left=9mm of center-rect.north] {};
  \node (line-b) [left=0pt of center-rect.west] {};

  \draw [dashed line] (line-a) to (line-b);

  \node [above=0mm of line-b, anchor=south west] {Internally focused};

  \node [above=10mm of line-b, anchor=south west]
  {\parbox{33mm}{\raggedright\diaNarrow
      Liberation from forms of attachment.
      Problems are solved by inner reflection.
      Understanding timeless conditions which one has no control over.
    }};

  \node [below=0mm of line-b, anchor=north west] {Externally focused};

  \node [below=10mm of line-b, anchor=north west]
  {\parbox{33mm}{\raggedright\diaNarrow
      Obtaining and safeguarding.
      Problems are solved by external factors.
      Application of skilful means, making wholesome choices, to the extent one has control.
    }};

  \node (tu0) [
    isosceles triangle,
    shape border rotate=270,
    above=2mm of center-rect,
    anchor=apex,
    minimum width=6cm,
    % isosceles triangle stretches,
    % minimum width=5cm,
    % minimum height=7cm,
    line width=0.4pt,
    draw,
  ] {};

  \foreach \idx/\h in {tu1/4.8, tu2/3.6, tu3/2.4, tu4/1.2}{
    \node (\idx) [
      isosceles triangle,
      shape border rotate=270,
      above=2mm of center-rect,
      anchor=apex,
      minimum width=\h cm,
    ] {};
    \draw (\idx.left corner) to (\idx.right corner);
  }

  \node (tb0) [
    isosceles triangle,
    shape border rotate=90,
    below=2mm of center-rect,
    anchor=apex,
    % isosceles triangle stretches,
    minimum width=6cm,
    line width=0.4pt,
    draw,
  ] {};

  \foreach \idx/\h in {tb1/4.8, tb2/3.6, tb3/2.4, tb4/1.2}{
    \node (\idx) [
      isosceles triangle,
      shape border rotate=90,
      below=2mm of center-rect,
      anchor=apex,
      minimum width=\h cm,
    ] {};
    \draw (\idx.left corner) to (\idx.right corner);
  }

  \node [below=7mm of tu0.north, anchor=mid] {Freedom from Becoming};

  \node [below right=7mm and 20mm of tu0.north, anchor=west, fill=white]
  {\parbox{35mm}{\raggedright\diaNarrow
        letting go of conceiving `I am',
        facing mortality,
        refuge in the deathless
    }};

  \node [below=7mm of tu1.north, anchor=mid] {Uncertainty};

  \node [below right=7mm and 20mm of tu1.north, anchor=west, fill=white]
  {\parbox{35mm}{\raggedright\diaNarrow
        letting go of safety in possessions,
        morality above oneself,
        openness to the unknown
    }};

  \node [below=7mm of tu2.north, anchor=mid] {Universality};

  \node [below right=7mm and 20mm of tu2.north, anchor=west]
  {\parbox{35mm}{\raggedright\diaNarrow
      letting go of cultural identity,
      not `we' and `them',
      giving up greed and hatred,
      self-acceptance
    }};

  \node [below=7mm of tu3.north, anchor=mid] {Humbleness};

  \node [below right=7mm and 20mm of tu3.north, anchor=west]
  {\parbox{35mm}{\raggedright\diaNarrow
      letting go of social recognition,
      accepting multiple opinions,
      trust in Dhamma practice
    }};

  \node [below=5mm of tu4.north, anchor=mid, fill=white] {Self-transcendence};

  \node [below right=4mm and 20mm of tu4.north, anchor=west]
  {\parbox{38mm}{\raggedright\diaNarrow
      letting go of personal achievements,
      recognizing impermanence,
      not seeking perfection
    }};

  \node [above=5mm of tb4.south, anchor=mid, fill=white] {Self-actualization};

  \node [above right=4mm and 20mm of tb4.south, anchor=west]
  {\parbox{35mm}{\raggedright\diaNarrow
      morality, creativity,\\ mastering skills, fulfilment of personal potential
    }};

  \node [above=7mm of tb3.south, anchor=mid] {Esteem};

  \node [above right=7mm and 20mm of tb3.south, anchor=west]
  {\parbox{35mm}{\raggedright\diaNarrow
      social recognition,\\ being respected,\\ confidence
    }};

  \node [above=7mm of tb2.south, anchor=mid] {Love and Belonging};

  \node [above right=7mm and 20mm of tb2.south, anchor=west]
  {\parbox{35mm}{\raggedright\diaNarrow
      family, friendships,\\ affiliation with a\\ trusted group
    }};

  \node [above=7mm of tb1.south, anchor=mid] {Safety};

  \node [above right=7mm and 20mm of tb1.south, anchor=west, fill=white]
  {\parbox{35mm}{\raggedright\diaNarrow
      physical safety, order,\\ livelihood, health
    }};

  \node [above=7mm of tb0.south, anchor=mid] {Survival Needs};

  \node [above right=7mm and 20mm of tb0.south, anchor=west, fill=white]
  {\parbox{35mm}{\raggedright\diaNarrow
      food, clothing,\\ shelter, medicine
    }};

  \node (process) [
    circle,
    left=0pt of center-rect.north,
    xshift=0.75cm,
    minimum width=1.5cm,
    % Not drawing because this shape is for positioning the arrow.
    % color=red,
    % draw,
  ] {};

  \draw (process) node [xshift=-1pt] {
    \tikz [x=0.75cm,y=0.75cm,smallish arrow] \draw (0,0) arc (10:350:1 and -1);
  };

  \node [tight box, right=12mm of center-rect.north] {\parbox{38mm}{\raggedright\diaNarrow
      Practicing clear comprehension,
      applying the right effort in a given situation
      for the ending of \emph{dukkha}.
    }};

\end{tikzpicture}%

\end{document}
