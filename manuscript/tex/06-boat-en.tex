\hypertarget{boat-1}{%
\chapter{Boat}\label{boat-1}}

Meditation techniques are useful, they let us learn by doing, but we
learn the techniques not for knowing the techniques, but for knowing the
mind. Too many techniques, or complicated steps and sequences however
are confusing, and lead to a sense of being lost.

Simplify it down to the essence. One breath, one BUD-DHO. On the
in-breath, think the first half of the mantra, BUD-, a pause in the
middle, and on the out-breath, -DHO. BUD-DHO. Done.

The essence is peace, and the understanding, which stops you.

The peace originates from the senses drawing in and turning inwards. The
seeking stops, because what is here is enough, and there is no need to
go anywhere. The quiet joy arises from the mind which understands that
there is no happiness in the world to be pursued. The values reorganize
themselves.

Where is the peace now? Where is the understanding now? There is nothing
to solve. BUD-DHO, ten breaths, and the world ends. It is enough, when
the question stops to mind. This pause is the listening silence, and the
answer is not necessary.

In an everyday situation, simplify it down, until it is clear to
recognize. You are completely exhausted, you have no energy, your mind
is spinning with the coming and going of the day, but the breath is
still available, the silence can be still felt.

One of the teaching images of the Buddha is bailing out the boat.
Imagine when a boat or ship is full of water, or full of goods it has to
carry, then the boat moves slowly. It is burdened, heavy and slow, all
the water and things are weighing it down.

We would like our boat to go fast, don't we? But at the same time we are
holding onto everything that is in it. We have to lighten the boat, give
up the self which is the heavy burden. We create 'me' and 'mine', we
create 'what I have to be', 'what I should be', 'what I am'. This is
what is holding to boat down, this is the weight.

Emptiness is when the boat is empty. It moves swiftly, because it is
light as a feather, not heavy with self, the story and drama of 'me' and
'mine'.

Often just stopping to look is enough to notice that we are happy where
we are. Stop and stay with it. A lot of us think we are never happy.

But when we are suffering, we know that there is something to learn,
something to understand, the attachment is hiding somewhere.

Letting go is not always cutting off. Sometimes we are telling ourselves
to let go, because we are afraid to open or commit ourselves, and
depriving ourselves pleases our self-critical mind. This ill-will and
anger toward ourselves is self-destructive.

See through the trick, the judgement that 'this is not ultimately
transcendental', which stops you from doing something good, how is that
wholesome?

Sometimes letting go means to let go of the fear to pull out all the
stops and do your best. Not because it is useful, not because somebody
expects it, but because you know it is the right thing to do, and you
love doing what you do.

One time I was walking in the countryside, and I had sheets of paper
maps with me. I had been walking for a few days, and I usually make
notes on the map, track which paths I followed, where did I find a good
camping spot, where did I receive alms-food in the village and so on, it
is a kind of log or journal. When I get back to the monastery, I scan
the maps and type up the notes.

This was a rainy and windy day, and I was walking in the middle of
nowhere on the muddy roads. I sat down to rest, and I thought, 'let's
mark up this last section of the route on the map.' I kept the map
sheets in a plastic folder. I take a look, and I can see today's map,
but yesterday's map is not there. \emph{I've lost yesterday's map.} With
all the notes.

I must have dropped it sometime earlier when pulling out today's map for
a look. It could be kilometres behind me, in the mud somewhere, or the
wind must have blown it into some corner. \emph{I've lost yesterday's
map.} I felt so shaken, it was comically absurd. I couldn't remember the
last time I was so disappointed.

The Sun was going to set soon, and I still had a lot of distance to
walk. The next morning I had to reach the next town, otherwise I can't
go alms-round and I don't eat that day. (The monk's rules don't allow us
to store food from one day to the next.)

So I couldn't easily turn around and start tracing my way back. I was
sitting there, thinking, 'I should let go. It's just some notes. This is
just a state of mind, a good monk should let go.'

But all that didn't sit right. I thought, 'What am I afraid of here? Why
is it wrong to like that piece of paper? Why is it OK to be critical,
but not OK to like myself? I love doing what I do, and I'm going back
for my map.'

I found it about 500m behind me. It was floating in a puddle, soaked,
but intact. I lifted it like it was an archaeological artefact.

That day I learnt more than I volunteered for. I rolled it up in a towel
and it dried eventually. It was almost dark by the time I found a place
to camp but everything was well. Next day I did get to the town in time
for alms-round, a man and two ladies offered me food for the day.

What lets the boat sail swiftly without obstruction?

Taking on everything is clearly not it, because the boat is going to
sink on the way. On one hand, keeping the wish-list short, on the other,
deciding the negative space of a work is just as useful, that is, what
is it that we are not going to include, what is not going to be our
problem to solve.

Setting boundaries to create structures in a way that allows us to
suffer less, and allow others to suffer less. If we take on everything
around us, the boat will sink even before leaving the port. That can't
help us, and can't help others either.

The Buddha taught the factors which lead to success. The energy to move
and do something toward a goal depends on the faith that it makes sense,
and the resolution to put effort into it. Doubt and criticism stops
everything. We don't have to know it will work, we will only know that
after we finished.

It is enough to consider that we investigated the situation sufficiently
to make a start. The plan will change anyway when we meet the
circumstances and recognize what needs to be solved.

If we consider the worst possible outcome that is reasonable to expect
and we prepare for it, that gives us enough resolution to start.
Investigating the circumstances, we resolve to do it. Determining the
tasks which need doing, putting in the effort to keep the momentum
going, keeping the sails in the wind.

There are times when investigation tells us to stop. Things change while
we are not looking, not waiting for us. Grateful for having been there,
there is a subtle art in gently closing the door behind us when leaving
a room, and moving on in silence.
