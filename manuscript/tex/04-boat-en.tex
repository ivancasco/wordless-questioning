\chapter{Boat}

\keywords{calming the mind, observing experience, sense-contact}

At the beginning of a meditation session, we calm the body and quiet the
thoughts by watching the breathing. We can't expect alertness and
balanced intelligence from an agitated, excited mind, and so
establishing at least some calmness is essential.

The calm mind is suitable for investigation. What can a happy person
learn from the teaching of the Buddha? What can an unhappy person learn?
Or, what about someone who just feels okay with nothing else special
going on?

We are observing our experience, the signs of impermanence, the
beginning and ending of feelings and thoughts, and how they appear,
change and disappear. Investigating for ourselves gives the words of the
teachings meaning and direct usefulness.

Our experiences manifest through the senses. Forms and colours are
perceived by the eye; sounds by the ear; smells by the nose; tastes by
the tongue; touch, hot and cold by the body; and the mind perceives
thoughts, memories, and other mental processes.

\keywords{three feelings, impermanence}

The experiences appear to us in three qualities, or feelings
(\emph{vedanā}): They may be pleasant, and we feel attracted to them;
they may be unpleasant, and we would rather distance ourselves from
them; or they may be neutral, and their presence doesn't bother us. An
aspect of the neutral feelings is that they, like the breath, can be
experienced as pleasant when we pay attention to them.

The appearance and cessation of the experience are not within our direct
control. The necessary condition for feeling is the contact between a
sense-base, a sense-object, and the attention which is directed there.
With this contact established, the sensation appears on its own. When
the sense-contact breaks, or our attention turns somewhere else, the
sensation disappears.

The happy person who is experiencing pleasant sensations can learn from
the experience. The attractive impression leads us to cling to the
pleasant sensation, when we forget that this dependent condition is
unreliable. The sensation doesn't belong to us. It is not possible to
keep, it has no deeper essence and is empty of self.

The unhappy person who is experiencing unpleasant, painful sensations,
can understand that this condition is not going to last. We can see that
it is superfluous to wind ourselves up with anger or hatred. When action
is required, we act, when waiting with patience is enough, we wait.

The person who feels they are living in a neutral, grey world, can avoid
to give themselves over to carelessness and foggy confusion. This
neutral condition is not going to be permanent either, and if we make
errors through lack of alertness, the result can be painful and
dangerous. It can be like running into a wall or falling into a hole in
the fog.

The impermanence and emptiness fundamentally changes our view, it
reorganizes our values.

The Buddha used the phrase, `all things converge on feeling.' The eye
sees forms, the ear hears sounds, the body feels touch, and so on. The
sense-base makes contact with the sense-object. If attention is present,
there is contact, and the result is the feeling.

Feeling draws our attention to it like a magnet. Remembering the
sequence in the suttas:

\begin{quote}
Rooted in desire, friend, are all things.\\
Born of attention, are all things.\\
Arising from contact, are all things.\\
Converging on feeling are all things.

\bigskip

\quoteRef{%

\href{https://suttacentral.net/an10.58}{AN 10.58}, Rooted

}
\end{quote}

\keywords{feeling as not-self, underlying tendencies}

This is the point where we complicate the matter. If we see it as a
transient, unreliable phenomena, we don't make a problem out of it. We
don't form attachments, craving doesn't have a basis to arise and we
don't become stressed. The Buddha compared feeling to the bubbles on the
surface of the water when it is raining heavily.\footnote{\href{https://suttacentral.net/sn22.95}{SN
  22.95}, A Lump of Foam} They appear quickly, and disappear. How could
there be anything in a bubble which we can hold onto?

But our ingrained habit is to assume that this feeling is `me', or
`mine'. From that, craving is born, either a desire to have more of it
or the desire to get rid of it. While we are spending our time reacting
to attraction and repulsion, the underlying compulsive tendencies
(sensual craving, hatred and delusion, i.e. the three \emph{āsavas}) are
fuelled and grow stronger in the mind.

There seems to be a lot of cleaning-up to do in the mind, but it's worth
it. Overcoming anger, for example, is an extremely productive way to
practice. It is an easy mental state to recognize and hence an easy
target to shoot at. Even small progress gives us an inner understanding
about ourselves and the way the Buddhist practice works. The effects of
anger are painful, it makes us sick, we lose our intelligence, and it is
destructive both to our personal and professional relationships. Greed
tends to be sticky, we know we shouldn't but we still want it; in
confusion we are lost; we fear getting close to fear; but it is easy to
not want to be angry. Being free from anger is a relief, and every step
of progress makes the next step easier. Once we cool down, what remains
is a sense of self-respect and the resolution to practice.

\keywords{fear and anxiety}

If we are in a dangerous or uncertain situation, naturally we begin
thinking about what should we do. Fear and anxiety are going to arise
because there is good reason for it. The emotion of fear carries the
information of possible danger, the emotion of anxiety implies an
uncertain outcome. Fear makes us cautious, and this is useful: I
wouldn't want to ride in a car with a driver who is not afraid of
crashing.

What should we expect from meditation? We might think that \emph{if we
were good meditators} we would be able to stop the fear and anxiety. If
we could just apply the right technique or remember the right words,
these annoying mind states would disappear. Notice in this motivation
the desire which strives for control. We are wishing for our situation
to be different from the way it is, to manipulate and end it.

And all the while, we are internally debating with ourselves, imagining
the situation to play out one way or another. The internal dialogue of
anxiety is a form of trying to control the events around us. We try
re-interpreting what we see in a way that fits our earlier view of the
situation. Once such a feeling has already appeared, though we can't
change or fix it, we are still part of the process. Our attitudes
influence the direction the feeling develops. Certainly, we can make it
worse. Alternatively, awareness of the mind state keeps it within safe
bounds, but gives it space to let it run its course and end.

When I am waiting for my luggage at the airport, I feel anxious -- did
they lose my luggage? I have done all I needed to do, and there is
nothing more I can do now. I feel anxiety because the situation is, in
reality, uncertain. As a practice, I recollect that I have enough space
to stay with this feeling. There is no need to hurry things up, the
anxiety can stay as long as it needs to be there.

We can't stop it, but we can stop making it worse. If there is a danger,
we do what is necessary. If there is no danger at the moment, but we
feel anxiety, we can understand that \emph{the anxiety is not the
danger}, and we can mindfully stay present without fear of the feeling.

\keywords{contemplating the body, slowing down thinking, BUD-DHO}

How does the feeling appear, when observed through the body? Where do we
feel it? When did it start? Is it changing? As a feeling in the body, is
it that bad? This type of investigation will not give us control, but it
develops an understanding that the feeling is not the danger, and we
don't have to continue the internal fight for control.

If the thoughts are not slowing down, we can occupy the thinking mind
with a thought which we determine, instead of allowing it to run in
every direction. A mantra, such as `BUD-DHO' can be used in this case.
It is a simple method to collect our scattered attention and make it fit
to work for our benefit.

If meditation feels too complicated, simplify it down to the essence.
Lots of complicated steps only increase the sense of unfamiliarity and
doubt.

One breath, one BUD-DHO. On the in-breath, we internally recite the
first half of the mantra, BUD-. The breath pauses in the middle for a
moment. On the out-breath we recite the other half, -DHO. BUD-DHO.

The essence is the understanding which stops you, and leaves peace where
`you' had been. The peace originates from the senses withdrawing, and
the flow of attention turning inwards. The seeking stops, because what
is here is enough, and there is no need to go anywhere.

\keywords{sadness at emptiness}

The first impression of this can be shocking. When we understand that
the wordly goals are empty and not as important as we thought, there can
be a feeling of subtle sadness, disorientation, we're not sure which way
to continue.

But this is just the result of our previous confusion, a bit like being
unsure about ourselves when waking up, a new world taking the place of
the dream. After the disorientation passes, a quiet joy arises in the
mind. The ongoing wakefulness recognizes the happiness in the present.
Our values reorganize themselves. We don't look for external strength
and security, because dependent conditions are uncertain,
unsatisfactory, and their pursuit is tedious without end.

Who is suffering? This experience, is it changing? Where is the peace
now? Where is the understanding now? Experience is not a problem to
solve. The awareness which knows it is right here, awareness stays with
the experience and comprehends it.

Turn attention to the moment before you ask the question: Who is asking
whom? This the trick of the narrating mind. It imagines there is
somebody to talk to, somebody to criticize or complain to. But the voice
speaking into the microphone, the questioner and the respondent are one
and the same, and between question and answer there is neither: only the
listening.

\keywords{stories of the world, BUD-DHO}

BUD-DHO, breathing in, breathing out: the stories of the world are not
interesting for us. When the questioning attention stops the words in
the mind, this is enough. Listening silence fills the pause, and the
answer is the present experience.

Meditation based on the breath and BUD-DHO is easy to adapt to informal
situations. In everyday situations, whether by using a mantra, or
wordlessly, simplify the practice until you can clearly recognize the
right attitude. The simple practice of watching the breath doesn't add
any more complications to the comings and goings of the world. We don't
have to solve it, it is enough to watch and listen.

\keywords{tudong story, self-criticism, self-support, aversion subverting Dhamma}

One time I was walking in the countryside, I had some printed maps as
sheets of paper with me. I had been walking for a few days at that
point, and I usually write notes on the map, tracking which paths I
followed, where did I find a good camping spot, where did I receive
alms-food in the village and so on. It is a kind of travel log or
journal. When I get back to the monastery, I scan the maps and type up
the notes.

This was a rainy and windy day, and I was walking in the middle of
nowhere on the muddy roads. I sat down to rest, and I thought, `let's
mark up this last section of the route on the map.' I kept the map
sheets in a plastic folder. I take a look, and I can see today's map,
but yesterday's map is not there. \emph{I've lost yesterday's map.} With
all the notes.

I must have dropped it sometime earlier when pulling out today's map for
a look. It could be kilometres behind me, in the mud somewhere, or the
wind must have blown it into some corner. I kept thinking, `I've lost
yesterday's map. I can't believe I've lost my map.' I felt so shaken, it
was comically absurd. I hadn't realized how much I treasured these
little notes, it felt like losing a part of my life. I couldn't remember
the last time I was so disappointed.

The Sun was going to set soon, and I still had a lot of distance to
walk. The next morning I had to reach the next town, otherwise I can't
go alms-round, which means I'm not going to eat that day. (The monk's
rules don't allow us to store food from one day to the next.)

So I couldn't easily turn around and start tracing my way back. I was
sitting there, thinking, `I should let go. It's just some notes. This is
just a state of mind, a good monk should let go.'

But all that didn't sit right. I thought, `What am I afraid of? Why is
it wrong to like that piece of paper? Why is it OK to criticise myself
and push toward the next goal, but not OK to be even a bit
self-supportive? I love doing what I do, and I'm going back for my map.'

I found it about 500 meters behind me. It was floating in a puddle,
soaked, but intact. I lifted it from the water so carefully like it was
an archaeological artefact. I rolled it up in a towel and it dried
eventually.

Meeting these obstacles are a fruitful practice, that day I learnt more
than I volunteered for. It was almost dark by the time I found a place
to camp but everything was well. Next day I did get to the town in time
for alms-round, a man and two ladies offered me food for the day.

The values we grow up with in Western culture make it readily acceptable
to think critical, judgmental thoughts about ourselves. When we say, `he
is his own worst critic', this sounds \emph{though but good}, this is
something we praise. Some of the Buddhist terminology fits right in with
this, `Give up your desires! You shouldn't have preferences! It is not
self! Let it go!' This mode of attention operates from self-aversion, it
subverts the Dhamma to beat ourselves with it, and although it's painful
to practice this way, we still think \emph{that's good}. Fortunately it
doesn't take any special skills to correct course in the right
direction, it's enough to stop going the wrong way.

\keywords{four roads to success, iddhipāda, changing plans}

The energy to move and do something toward a goal depends on the faith
that it makes sense, and the resolution to put effort into it. Doubt and
criticism stops everything. We don't have to know how it will work out
to the end, but if we considered the situation, we are ready ask, `What
is the smallest possible step I can do now?'

The Buddha described the mental tools for success in four categories,
called the basis for success (\emph{iddhipāda}): will, energetic effort,
focused attention and investigation.

You might have heard the saying, `Plans are worthless, but planning is
everything'.\footnote{U.S. President Dwight D. Eisenhower had used this
  phrase, which he credited to an unnamed soldier.} The plan changes
when we meet the actual circumstances, but when we are improvising a new
route, we are using the information we gathered while planning.

Investigating the circumstances, considering the worst possible outcome
that is reasonable to expect, and showing we can at least avoid that,
that's enough to resolve to start. Determining the tasks which need
doing, putting in the effort to keep the momentum going, keeping the
sails in the wind.

\keywords{meeting obstacles, the best time to learn}

In theory, to learn and practice sounds attractive, but what kind of
situations can we expect to learn from? Looking back, I remember periods
when everything was going well in life and things were under control. At
such pleasant times, at best I could repeat the old steps which had
always worked before. When I was feeling terrible, sorry for myself and
complaining, again I didn't learn much from that. And when I followed a
routine of trivial, comfortable but grey habits day after day, that
wasn't particularly insightful either.

Quiet and peaceful times are a blessing. I always appreciate a stable
routine which allows for long periods of concentrated work or dedicated
practice. That said, obstacles and conflicts are guaranteed to arise. We
don't have to worry that meditation is going to solve our every problem
and that we will have nothing left to do. Meditation is not
problem-solving. It is a practice of awareness which overcomes inner
obstacles and faces external problems as they arise. If we have
something important to do, it may help to clear our head first. But
merely sitting on a cushion as if we have transcended all problems, we
are practising ignorance the present, not the awareness of it.
Voluntarily facing obstacles and addressing them skilfully is a golden
chance to develop the mind beyond our preconceived limits. Acknowledging
the confused chaos is rich in the potential to realize practical
benefits.

We are not seeking the feeling themselves, not trying to create special
feelings by meditating, or seeking the ideal situation where everything
will pleasantly work for us. Pleasant, unpleasant, neutral feelings will
not, in themselves, give us right understanding if we follow their
influence and react mechanically. Awareness has to notice their
impermanence and uncertainty. With this, we can see what is wholesome
and what is unwholesome in the present situation.

\keywords{boat moving on the river, me and mine}

Is meditation practice easy or difficult? A useful image to think about
is how a boat moves down a river. When the boat is packed and burdened
with product-filled crates, it moves heavily and slowly. It is just
barely holding itself above the water.

We want our boat to go fast, don't we? But at the same time, we are
holding onto so much. We have to lighten our boat, and let go of the
heavy burden of the self. We create the burden of `me' and `mine'. We
create the impression of `I have been like this. I am like this. I
should be like this.' `That was mine. This is mine. This, I want to
keep. That, I have to get'. This is the weight that is holding our boat
down.

The sense of having enough opens up mental space for generosity.
Contentment is an ongoing part of practice. It is not a fixed state
attached to conditions. Wise action and learning flow with ease from
contentment: When I think, `I will be ready to do it when I have
\ldots{}', discontent occupies my thoughts and keeps interrupting my
focus on the current situation.

\keywords{wholesome thoughts, peace}

But when I think, `I am not good at it, but I have enough to start',
accepting my current limits gives me energy for action. And to my
surprise, I often end up doing more than what I thought I could.

Thinking has a bad reputation in meditation texts, but clear thoughts
create a condition for developing right attitude. Proliferative,
compulsive thinking is a painful experience, but wanting to stop all
thinking also misses the target.

Notice how wholesome thoughts are followed by contentment and peace.
Consciously recollecting one's moral actions establishes a sense of
stability and self-respect. We can trust ourselves to let go of the
superfluous because we feel we already have enough.

If we try to solve it in our head, the practice is going to get
complicated fast. In meditation, awareness through the body is a
reliable guide: Watching the feelings and mind states as they come and
go, we shift our view from being preoccupied with ourselves to a
different vantage point, seeing from a different perspective. We can
leave behind complicated questions because we no longer need the
answers.

\keywords{light boat, enjoyable learning}

What enables us to keep learning and developing? The journey is most
enjoyable when the horizon keeps expanding beyond our previous limits.
We expand the horizon not by travelling far, but by seeing with new
eyes. The desire to hold onto what we think we are creates our current
limits.

When it is empty of me and mine, the boat is light. It can cover great
distances without making drama and fuss. What happens, if we are sitting
in a boat, and somebody runs into us with their boat? We shout at them,
push them away with the oars, and complain about it for the rest of the
day. All this might be justified, but we ruined our day with our own bad
company. It's hard to see the wisdom in that. What happens if an empty
boat floats into our boat? Where did the earlier anger and negative
emotion come from?

We tend to manufacture stories about me and mine, whether based on real
or imagined events. If we take them seriously, and give them reality,
the stories start to control us, and we create problems which didn't
exist before.

Sometimes we sit on a meditation cushion and start playing out inner
arguments with puppets of the imagination. It's a serious business! We
have to win! Methodically thinking through a problem is a powerful tool,
but sympathy and kindness toward ourselves is necessary for a
constructive inner dialogue. Otherwise, when the self is talking with
itself, it finds itself in bad company.

\keywords{he who can laugh at himself}

It's surprising how we can wind ourselves up about a situation which
hasn't even happened yet. It helps to keep a pinch of humour in our
side-pocket in case of emergency seriousness. Recollecting a saying of
the Greek philosopher Epictetus, `He who can laugh at himself never runs
out of things to laugh at.'

\keywords{simplicity, letting go}

In the practice of meditation, we restore right view by returning to the
simplicity of the senses. If stories arise, we observe them from the
perspective of changing conditions. By investigating the senses, we take
a more fundamental level as our basis for attention. Pleasant feeling is
like this, as we are experiencing it. Unpleasant feeling is like this.
Neutral feeling is like this. They have a beginning and an end, they are
changing and empty.

In the practice, the value comes not in accumulating results in a hurry,
but in leaving space for letting go and patience. There are times for
action, but simple patience solves a surprising variety of difficulties.
The sense of being hurt, the feelings of urgency and importance come
from ourselves. Restraint is a safe perspective, guarding ourselves and
others. It allows our boat to move on in silence.
