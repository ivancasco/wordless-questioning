\chapter{Cycles}

\keywords{steps in practice, descriptions and awareness}

Meditation teaches understanding through attention to feelings and
experiences in the present. The instructions often describe a
step-by-step development, but in the present, only one moment is
available to us. Are we taking a step forward or backward? Either way,
the experience is one step at a time, the step where everything is
changing. We use words to describe experience, but awareness of that
experience is wordless. Mindful attention is actively looking inward as
if asking a question. Even the symbols of language are limiting. While
they consist of fixed representations, the experience is in motion.

Perfecting the steps of the instruction is not the purpose of
meditation. The purpose is clear knowing of present experience, this
restores right perspective. We can develop the impression that we must
always complete the same sequence of steps, and when the mind doesn't
develop according to that sequence, we feel disappointed.

\keywords{narrating mind, experience as the basis}

And worse, other people seem to be meditating peacefully! They must have
got it right! The practice brings self-doubt to the surface. We may
consider it from the other direction: Someone might compliment us, `You
looked so peaceful, I'm sure you know something!' But, knowing how much
distracted thinking was going on in our head, we can see how unreliable
these impression are.

The narrating mind keeps mechanically making comments which are devoid
of any depth or investigation. The value of taking this step back is in
getting a taste for how unreliable and uncertain your own thinking mind
is, even when you think `I'm sure about this!'

Turn that attitude around and start with experience. Begin with a
questioning, curious attention that wordlessly inquires about the
present. If we take our experience as the basis, the way this experience
is now, what understanding does that give us? We look at ourselves first
-- how are we feeling? what state are we in? -- and respond
intelligently to that with cultivating our meditation in the right
direction. When painting a wall, we look at the wall first, choose the
right paint for it, and \emph{then} follow the advice on the paint can.
The wrong kind of paint will peel off, won't it? There are times when we
need to calm down, other times we need to generate energy and effort, or
wait for an internal storm to pass.

The various steps of any meditation technique are a method of learning
through imitation. Following an example, we watch ourselves and see how
our minds work. When we feel suffering, either we are able to resolve it
or we wait it out with patient endurance until it ends. Once it passes,
we look back with a clear head, knowing what had happened and by what
cause. In this way, our confidence in the practice will grow. We have
learned something there, and we don't need to hold on to the details of
the introductory example.

\keywords{development in cycles}

This would be a simple task, if our meditation practice developed along
a straight line, in direct proportion to the minutes and hours spent
practising. We plan that we are going to sit down, perhaps a bit
distracted at the beginning, but after an hour, \emph{if we are good
meditators}, we are going to feel stillness and our mind is going to be
clear and focused. At least this is what we expect.

Later, as we recollect how the meditation went, we observe that this is
not what happens. Our experience doesn't develop in a linear way from
shallow to deep or from distracted to focused. We might think that this
is our fault, because we are not `good meditators', or because we are
not doing the steps `right'.

As soon as we try to follow a technique step-by-step, everything starts
happening differently from our expectations. We might think, `Am I not
trying hard enough?' We keep pushing and it keeps getting more painful.
This is the feeling of trying to fit an opinion onto the experience.

If we look back at how our experience changes over time, we see a
different pattern. Experiences arise, change, cease, and are followed by
other experiences. The mind develops in cycles like this, and these
cycles ignore our goals of wanting to develop our meditation like a rank
ladder. The commenting mind tries to fit a personal story onto the
experience, about us being somebody who is good or bad at meditation.

Instead, we can take our experience as ground truth and start from
there. What kind of experience is this? We can notice how attention
moves as a process in consciousness and how it progresses through
different cycles.

Initially, the mind is content to sit, resting with relaxed attention,
just like when sitting down on a bench after a walk: sitting and
breathing is peace complete. But thoughts start coming up and we follow
them. We stop and remain content with the stillness again, thinking
might even stop without us noticing that we are not thinking. But
attention starts moving and we notice ourselves thinking again.
Memories, desires and restlessness can come up and we notice we have to
work on these. Then, the mind is again content and returns to the
feeling of stillness.

\keywords{knowledge and knowing, naming process}

Some knowledge is necessary, but a little is enough. Remembering the
teachings of the Buddha is a treasure which doesn't run out. But the
insights and understanding doesn't become \emph{our knowledge}, as
something we own from that point on. We can't put the truth in a box and
store it for next time, instead, we continue recognizing it in the
present. When we create fixed ideas from what we think we already know,
the practice looses touch with reality. Every time we start again from
the beginning, and from there, we trust the present knowing.

Facts and statements are attractive to the thinking mind, we feel a kind
of security in reciting facts. We would like to say, `I had a good
meditation', `I had a bad meditation.' We want to create distinctions
and to name our experience.

This is the dissatisfied mind. It wants to become something, it wants to
arrive at a state and have a name. But there is nowhere it likes to
stop. It goes on and on until we notice that, in this constant running,
we are completely exhausted.

When it becomes apparent that we are the ones doing this, the naming
stops. It stops because seeing replaced not-seeing; knowing replaced
ignorance. Consciously seeing the naming is enough to stop the
compulsion from continuing.

In the present, everything is changing, nothing is static. Everything
moves, experience is turning and flowing. It doesn't pose for a photo
and wait for us to name it. In this change, the doubtful and anxious
questions, identity and goals dissolve and lose their meaning. Using the
phrase in the \emph{Mahāsatipaṭṭhāna Sutta}, `He dwells independent, not
clinging to anything in the world.'

This is enough, knowing the mind this way we stop and arrive at a place
where we can be grateful for being. Not a for anything in particular.
Being grateful that there is experience, knowing, clarity, and the
freedom which allows us to stop going towards more and more.

\keywords{ambiguity, limited symbols, knowing without naming}

In a balanced posture, the subtle feelings of the body are easy to
observe. With a curious attitude, we direct attention inward. We don't
know ahead of time what we are going to find.

These feelings are often not clear. We experience them but they don't
have clear boundaries. They don't have edges, or a definite shape. We
try to find words for them, but they don't fit. We are not sure what to
call them.

All symbols which could be used as a name are lacking. In our western
culture, we have strong trust in facts, and we like to return to that
security through terminology by giving things a name. We are not
familiar with the type of cognition that doesn't use names and fixed
symbols. Although the feelings and the experience as a whole are not
precisely defined, we know that this experience is present.

This way we can distinguish the naming process from the experience
itself. The subtle feelings in the body are nebulous, they don't have
clear boundaries. While breathing in and breathing out, we can
experience what the body as a whole feels like -- everywhere at once.
The whole body is breathing. There is feeling and experience, but there
are no names and clear boundaries.

We drop the naming process and recognize that we can know these feelings
as they are present. The knowing mind is glad to widen its scope and
include experience without filters. We can know what experience is like
without having to find a name for it.

\keywords{contemplating the mind, too much thinking}

Accompanying unwholesome mental states, we can notice sensations of
heat, restlessness, dissatisfaction and anxiety. We remember to turn
toward it with patience, and maintain endurance while this state goes
through its stages. This too will change, this too will end, and we can
wait for it. When we know where we are, in most cases this is enough.
Processes in the mind will change on their own. If we are not putting
fuel on the fire, it will burn up what it has and will go out on its
own.

Resolution and repetition are part of the practice, but in striving
towards a specific goal, the effort becomes bitter and tedious. `Is this
complete awakening yet? Or at least a bit? When is the meditation bell
going to ring?' Don't look for a state. The mind which tries to become
awakened is overcomplicating the situation.

Present experience is always simple, and mindful attention has a
capacity for wise understanding. In this practice, we keep returning to
it, this is what guides the effort.

We can't force this by will, and we can't guarantee what will happen: we
have to trust the process. What remains is a wholesome mind which
understands what is happening. We are not rushing with compulsion nor
forcing our way through things. After the difficulty, we have space for
gratitude to appear. We can appreciate the feeling of coolness and the
comfort of being at ease.

\keywords{integrated practice}

We look at our teachers as examples. They didn't meditate to achieve
some special state and then look for something else to do. Meditation
was not separate from, but was integrated into their lives. In examples
from the suttas, the traditional scriptures of Buddhism, the Venerable
Sāriputta's practice was to stay with the perception of
emptiness.\footnote{\href{https://suttacentral.net/mn151}{MN 151}, The
  Purification of Alms} The Buddha is portrayed as maintaining his mind
in concentration on the signless. This is how they continued to
meditate.
