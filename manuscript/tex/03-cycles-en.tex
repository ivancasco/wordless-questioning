\hypertarget{cycles-1}{%
\chapter{Cycles}\label{cycles-1}}

Meditation teaches understanding through the feelings and experiences in
the present. The instructions are given in steps, but in the present
only one moment is available to us. There is no step forward or
backward, only this one, where everything is changing. We use words to
describe experience, but awareness of that experience is wordless. The
symbols of language are even limiting, because those are fixed, while
the experience is in motion.

Perfecting the steps is not the purpose of meditation. The purpose is
clear knowing of the present experience, restoring right perspective. We
can develop an impression that we should always complete the same
sequence of steps, and when the mind doesn't develop according to that
sequence, we feel disappointed.

Turn that attitude around and start with experience. If our experience
is the base, the way it is, then what understanding does that give us?
When painting a wall, we look at the wall first, choose the right paint
for it, and \emph{then} follow the advice on the paint can. The wrong
paint will just peel off, won't it? We look at the mind first, how we
are feeling, what state we are in, and respond intelligently to that.

The steps are a method of learning through imitation, following an
example we watch ourselves and see how things work. When we feel pain
and suffering, and either we are able to resolve it or wait it out
patiently, we will know that was not just imitation, and our confidence
in the practice will grow. We have learnt something there, and we are
not going to hold on to the details of the first, introductory example.

It would be great if our meditation would develop in a straight line, in
direct proportion to the minutes and hours we have spent practicing. We
imagine that we are going to sit down, perhaps a bit distracted at the
beginning, and after an hour, \emph{if we are good meditators}, we are
going to feel stillness and our mind is going to be clear and focused.

Later, when we recollect how the meditation went, we observe that this
is not what happens. Our experience doesn't develop in a linear way from
shallow to deep, or from distracted to focused. We might think this is
our fault, because we are not good meditators, or we are not doing the
steps right.

As soon as we try to follow a plan step-by-step, everything is different
from our expectations. So maybe we are not trying hard enough? We keep
pushing and it only gets more painful. This is the feeling of trying to
fit an opinion onto the experience.

If we recollect how the experience changes over time, we see a different
pattern. An experiences arises, goes through change, ceases, and another
experience arises. The mind develops in cycles like this, and these
cycles ignore our goals, that we want to develop our meditation like a
rank ladder.

We can take experience as ground truth and start from there. What kind
of experience is this, here? We can notice how attention moves as a
process in consciosness, how it progresses through different cycles.

Initially, the mind is content to sit, and attention is relaxed, rests
in the sitting. Thoughts start coming up and we follow them. We stop and
stay content with the stillness again, thinking might even stop without
us noticing that we are not thinking. Attention starts moving and we
notice ourselves thinking again. Memories, desire, restlessness can come
up and we notice we have to work on this. Then the mind is again content
and returns to the feeling of stillness.

Some knowledge is necessary, but a little is enough. Remembering the
teachings of the Buddha is a treasure which doesn't run out. But this
knowledge doesn't become ours, we can't put it in a box and store it for
next time. Whatever we learnt, every time we start again from the
beginning, and from there, trust the present knowing.

Facts and statements are attractive to the thinking mind, we feel a kind
of security in reciting facts. We would like to make a statement that it
is a good meditation, or a bad meditation, we want to create a
distinction and name it.

This is the dissatisfied mind. It wants to become something, it wants to
arrive at a state and have a name. But there is nowhere it likes to
stop. It goes on and on, until we notice that this constant running is
crazy.

When it becomes visible in consciousness that we are the ones doing
this, the naming stops. It stops because ignorance, not seeing, was
replaced with seeing. Consciously seeing it is enough to break the
compulsion from continuing.

In the present everything is changing, nothing is static. Everything
moves, experience is turning and flowing, it doesn't pose for a photo
and wait for us to name it.

This is enough, knowing the mind this way we stop and arrive at a place
where we can be grateful for being. Not a for anything in particular,
just being grateful that there is experience, knowing, clarity, and the
freedom which allows us to stop going towards more and more.

In a balanced posture the subtle inner feelings of the body are easier
to observe. We direct attention inwards, in a curious manner. We don't
know in advance what we are going to find.

These feelings are often not clear. We experience them but they don't
have clear boundaries. They don't have edges, or a definite shape. We
try to find words for them, but they don't fit, we are not sure what to
call them.

All symbols which could be a name, are lacking. In our western culture
we are used to trust in facts, and we like to return to that security we
feel in names and terminology. We are not familiar with the cognition
that doesn't use names and fixed symbols. The feelings, the experience
itself is not clearly defined, just the fact that we know that this
experience is present.

This way we can distinguish the naming process from the experience
itself. The feelings in the body are especially nebulous, they don't
have clear boundaries. During breathing in and breathing out, we can
experience what does that feel like in the whole body, everywhere at
once. The whole body is breathing, there is feeling and experience, but
there are no names and clear boundaries.

We drop the naming process and recognize we can simply know these
feelings as they are present. The knowing mind is glad to include
experience without filters. We can know what experience is like, without
having to find a name for it.

In the sensation of unwholesome mental states we can notice a sort of
heat, restlessness, dissatisfaction and anxiety. We remember to turn
toward it with patience, and maintain endurance in the presence of the
feelings of that state. This too will change, this too will cease, and
we can wait for it. When we know where we are, in most cases this is
enough. Processes in the mind will change on their own. If we are not
putting fuel on the fire, it will burn up what it has and go out on its
own.

We can't force this by will, we have to trust the process. What remains
is a wholesome mind which understands what is happening. Not being
rushed by compulsion, not being forced. After the difficulty, we have
space for the gratefulness to appear, with the feeling of coolness and
comfort of being at ease.

We look at our teachers as examples. They didn't meditate to achieve a
special state and then look for something else to do. Meditation doesn't
separate out, but integrates into their life. In the examples of the
preserved traditional scriptures of Buddhism, the Venerable Sariputta
said his mind stays with the perception of emptiness, the Buddha said he
abides in the signless concentration, this way they continue to
meditate.
