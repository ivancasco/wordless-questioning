\clearpage

\section{All the Taints (excerpt)}

{\centering
\emph{\href{https://suttacentral.net/mn2/en/bodhi}{MN 2}, Sabbāsava Sutta}
\par}

{
\setlength{\parindent}{0pt}\setlength{\parskip}{5pt}
\fontsize{9.5}{14}\selectfont

\emph{(Abandoning the taints is for one who knows wise attention)}

Bhikkhus, I say that the destruction of the taints is for one who knows and
sees, not for one who does not know and see. Who knows and sees what? Wise
attention and unwise attention. When one attends unwisely, unarisen taints arise
and arisen taints increase. When one attends wisely, unarisen taints do not
arise and arisen taints are abandoned.

\emph{(The uninstructed person falls into the thicket of views)}

This is how he attends unwisely:

`Was I in the past? Was I not in the past?'

`What was I in the past? How was I in the past? Having been what, what did I become in the past?'

`Shall I be in the future? Shall I not be in the future?'

`What shall I be in the future? How shall I be in the future? Having been what, what shall I become in the future?'

Or else he is inwardly perplexed about the present thus:

`Am I? Am I not? What am I? How am I? Where has this being come from? Where will it go?'

When he attends unwisely in this way, one of six views arises in him.

The view `self exists for me'\\
arises in him as true and established;\\
`no self exists for me'\\
`I perceive self with self'\\
`I perceive not-self with self'\\
`I perceive self with not-self'\\
`It is this self of mine that speaks and feels and experiences here
and there the result of good and bad actions; but this self of mine is
permanent, everlasting, eternal, not subject to change, and it will
endure as long as eternity.'

This speculative view, bhikkhus, is called the thicket of views, the
wilderness of views, the contortion of views, the vacillation of views,
the fetter of views. Fettered by the fetter of views, the untaught
ordinary person is not freed from birth, ageing, and death, from sorrow,
lamentation, pain, grief, and despair; he is not freed from suffering, I
say.

\emph{(The noble disciple reflects wisely)}

He attends wisely:\\
`This is suffering';\\
`This is the origin of suffering';\\
`This is the cessation of suffering';\\
`This is the way leading to the cessation of suffering.'

When he attends wisely in this way, three fetters are abandoned in him:
personality view, doubt, and adherence to rules and observances. These
are called the taints that should be abandoned by seeing.

\enlargethispage*{2\baselineskip}

\emph{(Conclusion)}

Bhikkhus, when for a bhikkhu the taints that should be abandoned by seeing
\ldots{} restraining \ldots{} using \ldots{} enduring \ldots{} avoiding \ldots{}
removing \ldots{} developing, have been abandoned, then he is called a bhikkhu who dwells
restrained with the restraint of all the taints. He has severed craving, flung
off the fetters, and with the complete penetration of conceit he has made an end
of suffering.

\bigskip

{\raggedleft
\emph{(translated by Bhikkhu Bodhi)}
\par}

}
