\hypertarget{csuxf3nak-1}{%
\chapter{Csónak}\label{csuxf3nak-1}}

A meditáció elején a légzés figyelésével ellazítjuk a testet és
lecsillapítjuk a gondolatokat. Egy zaklatott, izgatott elmétől nem
várhatunk éber és kiegyensúlyozott intelligenciát, ezért legalább némi
nyugalom megalapozása lényeges.

Vizsgálódásra a nyugodt elme alkalmas. Mit tanulhat a boldog ember a
Buddha tanításából? Mit tanulhat a boldogtalan ember? Vagy, akivel semmi
különös nincs, éppen megvan ahogy van?

A tapasztalatunkat figyeljük, az állandótlanság jeleit, az érzések,
gondolatok kezdetét és végét, megjelenését, változását és megszűnését
figyeljük. A Négy Nemes Igazság megértése ezen a vizsgálódáson keresztül
kapcsolódik számunkra a valósághoz, a hallott tanítások így nyernek
értelmet.

Az érzékeken keresztül nyilvánulnak meg a tapasztalataink, az érzéknek
megfelelő formában. A szem formákat és színeket, a fül hangokat, az orr
szagokat, a nyelv ízeket, a test a tapintás, a hideg és meleg érzett
benyomásait fogja fel, az elme pedig a gondolatokat, emlékeket, mentális
folyamatokat jeleníti meg.

Ezek három minőségben jelennek meg -- lehetnek kellemesek, és vonzódunk
feléjük, lehetnek kellemetlenek vagy fájdalmasak, és inkább távolodnánk
tőlük, vagy lehetnek semlegesek, és a jelenlétük nem zavar minket, de ha
figyelünk rá kellemesek is lehetnek, mint a légzés.

A megjelenésük és elmúlásuk nincs közvetlen irányításunk alatt, a
szükséges feltétel mindössze annyi, hogy az érzék kapcsolatba kerüljön a
neki megfelelő érzék tárggyal, és a figyelmünk oda irányuljon. Az érzet
magától megjelenik. Amikor a érzék-kapcsolat megszakad, vagy a
figyelmünk másfelé fordul, az érzet megszűnik.

A boldog ember -- aki kellemes érzéseket tapasztal, ebből azt
tanulhatja, hogy ne higgyen a vonzó benyomásban, és ne ragaszkodjon a
kellemes érzéshez, mert ez a függő állapot megbízhatatlan. Nem a
sajátja, nem megtartható, nincs lényege és üres.

A boldogtalan ember -- aki kellemetlen, fájdalmas érzéseket tapasztal,
azt tanulhatja, hogy ez az állapot nem lesz maradandó, és ne keverje
magát haragba és gyűlöletbe emiatt, elegendő türelemmel várnia.

Aki úgy érzi, semleges és szürke világban él, azt tanulhatja, hogy ne
engedjen emiatt a figyelmetlenségnek és ködös zavarodottságnak, mert ez
a semleges állapot sem lesz állandó, és ha az éberség hiányában téves
nézetet követ, abból csalódás, fájdalom és szenvedés származik.

Az állandótlanság és üresség alapvetően megváltoztatja a nézőpontunkat,
átrendezi az értékeinket.

Ha sok minden jár a fejünkben és a gondolatok nem csillapodnak,
lefoglalhatjuk a gondolkodást egy előre eldöntött gondolattal, ahelyett,
hogy engednénk minden irányba rohanni. A BUD-DHÓ mantra hasznos
ilyenkor, összefogja a szétszórt figyelmet egy egyszerű módszerrel. A
théraváda buddhista szövegek nyelve a Páli, a \emph{buddhó} szó
jelentése 'aki felébredett'.

Leegyszerűsítjük a meditációt a lényegre, sok bonyolult lépéstől csak
növekszik az ismeretlenség és kétség érzése.

Egy lélegzet, egy BUD-DHÓ. Belégzés közben magunkban szavaljuk a mantra
első felét, BUD-, középen a lélegzet megáll egy pillanatra, és kilégzés
közben a másik felét szavaljuk, -DHÓ. BUD-DHÓ.

A lényeg a béke, és a megértés, ami megállít. A béke abból ered, hogy az
érzékek visszahúzódnak és befelé néznek. A keresés megáll, mert ami van
az elég, és sehova nem kell menni. A csendes öröm abból fakad, hogy az
elme megérti, hogy nincs boldogság amit a világban hajszolni kell. Az
értékek átrendeződnek, nem kívül keressük az erőt és boldogságot, mert
ez a függő kapcsolat mindig bizonytalan és kimerítő.

Hol van a béke most? Hol van a megértés most? Semmit nem kell megoldani.
BUD-DHO, néhány lélegzet, és a világ sztorijai már nem érdekesek
számunkra. Elég, ha a kérdés megállítja az elmét. Ez a megállás a
figyelő csend, a válasz pedig nem szükséges.

Hétköznapi helyzetben, egyszerűsítsd le, amíg mindig felismerhető. Akár
egy mantrával, akár szótlanul is. Teljesen fáradt vagy, semmi energiád,
kavarog a fejed a napi jövés-menéstől, de a légzés akkor is elérhető, a
csend ott is érezhető.

Milyen helyzetben várhatjuk azt, hogy tanulunk valamit, amit korábban
nem értettünk? Visszanézve arra emlékszem, hogy amikor minden jól ment
és irányítás alatt volt, legfeljebb a régit tudtam ismételni. Amikor
szörnyen éreztem magam és keseregve panaszkodtam, abból igazán nem
tanultam semmit, és amikor mindent szokás szerint követtem éppen úgy
mint tegnap, az sem volt kifejezetten hasznos.

Ezért nem magukat az érzéseket keressük, nem különleges érzéseket
próbálunk létrehozni a meditációval, nem az a helyzetet keressük, ahol
mindig minden kellemes. A kellemes, kellemetlen, semleges érzések
önmagukban nem adnak nekünk helyes megértést, mert csak követjük a
befolyásukat és gépiesen reagálunk rájuk -- az állandótlanságukat,
bizonytalanságukat kell az éberségnek észrevennie, akkor megértéssel
látjuk mi a jótékony, mi a kártékony a jelen helyzetben.

A gyakorlás könnyű vagy nehéz? Egy hasznos kép amire gondolhatunk, ahogy
egy csónak halad a folyón. Amikor a csónak tele van beszivárgott vízzel,
vagy áruval töltött ládákkal van megrakva, a csónak lassan halad. Le van
terhelve, nehéz és lassú, a beszivárgott víz és ládák mind lehúzzák.

Azt szeretnénk, hogy a csónakunk gyorsan haladjon, nem igaz? De
ugyanakkor ragaszkodunk mindenhez amivel megraktuk. Könnyítenünk kell a
csónakon, elengedni az ént, ami a legnehezebb súly. Mi hozzuk létre az
'én' és 'enyém' súlyait, mi hozzuk létre a benyomást, hogy 'ilyen
voltam, ilyen vagyok, ilyen kell legyek', 'ez az enyém volt, ez az
enyém, ezt meg kell szereznem'.

A csónak üres, amikor üres az éntől és enyémtől. Gyorsan mozog, mert
könnyű, mint a pihe, nem súlyos az éntől, az én és enyém sztorijai és
drámájától.

Mi történik, ha a csónakban ülünk, és valaki a csónakjával nekünk
ütközik? Mérgesen arrébb lökjük az evezővel, talán még kiabálunk is
vele. Viszont mi történik ha egy üres csónak ütközik nekünk? Honnan
eredt a korábbi harag és indulat?

Hajlamosak vagyunk az én és enyémről szóló történeteket gyártani, akár
valós, akár képzeletbeli események alapján. Ha komolyan vesszük ezeket,
és valóságot adunk nekik, a történetek kezdenek irányítani minket, és
korábban nem létező problémákat hozunk létre.

A meditáció gyakorlásában visszaálltjuk a helyes szemléletet azzal, hogy
visszatérünk az érzékek egyszerűségéhez, és a történeteket, ha vannak,
nem egy rögzült álláspontból nézzük. Az érzetek vizsgálatával egy
alapvetőbb valóságot veszünk alapul. A kellemes érzés ilyen, ahogy most
tapasztaljuk, a kellemetlen érzés ilyen, a semleges érzés ilyen, kezdete
van és vége, változik és üres.

A gyakorlásban nem a halmozás és sietség lesz értékes, hanem teret
hagyni az elengedésnek és türelemnek. Van amikor cselekedni kell, de sok
nehézséget megold az egyszerű türelem. A sértettség vagy sürgető
fontosság érzése magunkból ered, a visszafogottság a magunk és mások
számára is biztonságos nézőpont ilyenkor, és a csend elegendő.
