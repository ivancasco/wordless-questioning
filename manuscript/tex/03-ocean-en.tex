\hypertarget{ocean-1}{%
\chapter{Ocean}\label{ocean-1}}

Bodily awareness is a different form of cognition. It is not thinking,
not seeking solutions, not trying to answer questions, just an awareness
which knows.

Remember what it is like, when you are watching the ocean. The thinking
stops. The ocean is unfathomable, changing and moving, it cannot be
enclosed with fixed thoughts.

This open awareness solves problems not by providing answers, but by
giving us the perspective that this is happening in a wider space.
Happiness and suffering, success and disappointment is happening in
this, and by seeing this we know that this not what we are.

We can be grateful for just being with ourselves. For the solitude of
being with ourselves in the moment, for having the time to be, for
having the space to accept.

The present moment is always interesting. When you feel that it is not
interesting, try and inspect it closely -- as soon as we look, we start
seeing things we don't quite know. Seeing our own experience, we don't
quite know what is going on.

This is like watching the ocean. Do you remember what happens in the
mind when watching the ocean? Thinking cannot fix the ocean, so it gives
up and stops. Thinking cannot freeze the waves, and say 'this wave is
like this, that wave is like that'. By the time we think that,
everything has changed.

The ocean is moving. The waves are coming in, the waves are going out.
The analytical mind gives up. It gives up, lets go, and allows the open
attention to receive the whole picture at once.

With this attention we turn towards our experience. Receiving the whole
picture at once, not deciding what is going to happen, let it surprise
us.

Feelings are arising. Emotions come and go. Sense of certainty and doubt
alternate like waves moving in and going out. 'What should I do? Maybe
this is good. Maybe this is not good.' This comes and goes. The heart
has space for it, and that space remains empty.

Feelings of hot and cold, pressure, the movement of the air and the
rib-cage. This the experience from the inside. We have very little
control over it. Breathing in, just experience. Breathing out, just
experience. There is nothing we can fix, nothing we can change about it.

There don't have to be names and labels. The mind recognizes itself and
knows how to stay with that relaxed, wholesome attention, with the
coolness of being free to just watch.

Not understanding our experience, we make decisions about it. We decide
something is good, something is bad. At that point we create the world
and we have to become something. This we can watch, that we are creating
that.

It is not necessary to have a clear verbal theory which we produce.
Feelings are ambiguous. They don't have a clear edge. Thinking is not
always certain, not determined, and always changes.

When watching the ocean, the mind opens out, thinking stops. Nothing is
certain. Everything is changing.

We are interested in that, that we are not certain. We want to see
what's there, not only to recall a piece of knowledge from memory.
Knowledge is fixed and dead, it doesn't respond. The present is moving,
the knowing attention is alive.

The ocean is not something that you can judge, 'the waves should be like
this, the waves should be like that'. It is always moving, you can't fix
it. So the mind stops the judgement. Stops the analytical process and
just watches. This is the open awareness which allows the meditation to
stay with a sense of knowing, which doesn't have fixed rigidity of
judgement.

Breathing in, breathing out, it moves in its own way. There isn't a lot
of precise control that we have over it.

When the senses make contact with the world, the feeling is born. When
that contact ceases, the feeling ceases.

Feeling, thoughts, emotions are coming in, going out. We didn't create
them, we were not the ones in control of creating it. So feeling that
somehow this is mine, doesn't seem to be according to reality.

It would be not correct to say that we created it. It would be not
correct to say that we can control how it behaves. We had no choice, it
was already moving by the time we were conscious of it.

We have to remember endurance and patience. Emotions have their energy
of movement, if they had the condition to appear, we can't stop them. If
we get involved, we will be swept away. If we step back and watch, they
will move on.

Sometimes our emotions are like a screaming child. Telling them to stop
is useless, walking away and rejecting them is cruel. A child can't grow
up without kindness, we can't grow up without being kind to ourselves.
Kindness can stay with the unpleasant without rejecting it, while
compassion creates structures which allow beings to suffer less. We
contain our actions with the moral precepts and create a safe space for
ourselves to be, this way we don't create suffering for others and
ourselves.

We can open the mind and have the courage to give emotions a safe space
in the heart, allow them to stay as long as they need. Remember that
this is happening in a wider context. It is one picture, not beautiful,
not ugly, it is both things. It is not easy, not difficult, it is part
of the areas of experience.

The open attention can recognize that judgement, in the sense that it
should be different, that kind of judgement is out of place. The
attention which can see the body as nature, doesn't want to create a
judgement, it is just like this.

Like the waves of the ocean. If someone wants to create judgement of the
waves, they are going to have a difficult time.
