\chapter{Csend}

\section{Jelzés}

Ketten sétálunk a kolostor bejáratához vezető úton. Beszélgetünk erről
és arról, de amikor belépünk, észrevesszük a csendet és a párbeszédünk
abbamarad. A Dhamma terem a következő ajtón túl van, és nem akarunk
zavarni senkit, aki esetleg odabent meditál. Halkan bezárjuk magunk
mögött az ajtót. Mi az épület egy másik részébe tartunk, de a Dhamma
terem jelentősége nagyobb ennél a hétköznapi feladatnál.

A hallgatás csendje egy implicit kapcsolatot teremt a környezetünkkel.
Az előbbi esetben azzal a személlyel, aki lehet, hogy a Dhamma teremben
ül, de még ha látnánk is, hogy nincs ott senki, akkor is lehalkítanánk a
beszédünket vagy csendben maradnánk. Amikor belépünk, a csend jelzésként
szolgál arra, hogy figyeljünk. Teret adunk az önmagunkon túli
értékeknek, melyeket az szív és elme igazságainak szentelt Dhamma terem
jelképez.

Ebben a környezetben a csend jelzés, ami arra irányít, hogy emlékezzünk
arra, amit a világi értékeken túl van. Amikor egy templomba, kolostorba
vagy más megszentelt helyre lépünk be, zajos világi ügyeinken túlra
tekintünk, túl az önmagunkra irányuló megszokott elfoglaltságunkkal.

Elég tapasztalatunk van a zajos csacsogásban ahhoz, hogy tudjuk, a mély
megértés nem abban található. Ezért elcsendesülünk, hogy figyelmünket a
hallgatásnak adhassuk, hogy része legyünk a megértésnek, amit szavakkal
nem tudunk kifejezni. Csendben mozgunk, csendben hallgatunk, óvatosan
eltávolítjuk magunkat az útból, hogy meghallhassuk a hely üzenetét és
engedjük a cselekvést magáért beszélni. A csend jelenlétet ad, ami nem
elkülönít, hanem magába foglalja a teret és az ott élő más lényeket.
David Whyte szavaival,\footnote{The Winter of Listening by David Whyte}

\begin{quote}
Tartozhatsz mindenhez, ha egyszerűen hallgatsz.
\end{quote}

\section{Értékes}

A csend azt is kifejezi, mennyire értékeljük amit éppen teszünk.
Csendben lenni és fenntartani a figyelmünket kifejezi az éberséget és
tiszteletet a cselevés iránt. Ez egyaránt egy belső és külső jelzés:
Mások látják, hogy bármi is legyen amit csinálunk, csendre van
szükségünk. Mi is látjuk önmagunkat ahogy csendben vagyunk, szándékosan
visszafogjuk hatásunkat magunkra és a környezetünkre, ami azt
kommunikálja, hogy ahol vagyunk és amit teszünk nagyobb jelentőséggel
bír, mint magunkról csacsogni.

A nyugalom, megértés és csend közeli kapcsolatban állnak egymással.
Felhagyunk a beszéddel és óvatosan figyelünk, hogy vizsgálódjunk és
megértsünk egy jelenséget. A verbális csendet követően, az elme
folytatja, `Miért? Miért?', de amikor összeáll a kép az `Aha!'
pillanatban, az elme is megáll a belső párbeszédben és csendben vagyunk,
örömmel tölt el minket a megértés. Ebben az elégedett, nyugodt
hangulatban csendben maradunk, pillanatnyilag semmi másra nincs
szükségünk.

\begin{quote}
Mint egy tiszta vizű,\\
nyugodt és mély tó, olyan\\
a bölcs, aki miután hallott\\
a dharmákról, lecsendesült.

\emph{Dhammapada 82.}\footnote{Ford. Fórizs László}
\end{quote}

Ez nem jelenti, hogy a hang nem lehet kellemes. A zene terápiás hatásai
nyilvánvalóak, és segít ellazítani a feszült elmét. Lehet, hogy
\emph{nagyon jó zene} (a mi véleményünk szerint), de hányszor tudod
meghallgatni egymás után? Ugyanaz a dolog újra és újra, rövid idő alatt
kellemesből fájdalmasba fordul át. Voltál már úgy, hogy órákig zenét
hallgattál, azt gondolva, `szeretem ezt', de mégis megkönnyebbülve
érezted magad, mikor kikapcsoltad? `Jó zene, de már hiányzott a csend.'

A hangok bejövő jelek, amik stimulálják az idegrendszert, jó érzést
kelthet egy ideig, de attól még folyamatos stimuláció marad. A zajos
környezet rontja a figyelmünk és intelligenciánk minőségét, valószínűleg
emlékszel arra, milyen nehéz tisztán gondolkodni amikor a szomszéd
telkén építkezési munkák folynak. A személyes tapasztalaton túl, orvosi
tanulmányok is felmérték azt, hogy `a szellemi munkavégzés és látási /
hallási figyelem jelentősen csökken'\footnote{\href{https://www.ncbi.nlm.nih.gov/pmc/articles/PMC6901841/}{The
  Effect of Noise Exposure on Cognitive Performance and Brain Activity
  Patterns (2019)}} when being exposed to noise. mikor zajnak vagyunk
kitéve.

A mobil telefonoknak nem is kell zajt csapniuk, hogy `leszívják az
agyat': egy másik tanulmány azt találta, hogy `a saját telefonunk puszta
jelenléte csökkenti az elérhető szellemi kapacitást'.\footnote{\href{https://www.journals.uchicago.edu/doi/10.1086/691462}{Brain
  Drain: The Mere Presence of One's Own Smartphone Reduces Available
  Cognitive Capacity (2017)}} Nem meglepő, hogy a hagyományos belátás
meditációt tanító elvonulások igyekeznek csendes környezetet teremteni,
és arra kérik a résztvevőket, hogy ne hozzák be a telefonjukat a
meditációs terembe, vagy hagyják azt egy elzárt helyen az egész
elvonulás idejére. Adj egy kis szünetet az idegrendszernek és engedd
lecsillapodni, ne járjon úgy mint a varjú Szantóka haiku versében,
`Károg a varjú, csapkod a varjú, nincs hova megüljön.'\footnote{Grass
  and Tree Cairn, Taneda Santoka}

\section{Kántálás}

A kolostorban kántálást gyakorlunk a mindennapos közös meditációk előtt
vagy után. Először, amikor egyenként megérkezünk a Dhamma terembe,
csendben három leborulást végzünk a Buddha oltár irányába. A rangidős
szerzetes megcsengeti a harangot, ezzel jelzi a kántálás kezdetét.
Csendben várunk, miközben meggyújtja az oltáron a gyertyákat és füstölő
pálcákat, majd újra meghajlunk. A leborulás közben mindig csend van.

Együtt elkezdjük a kántálást, összehangolva a hangunkat: a halkat nem
lehet hallani, a hangos túl erős és hamisan kiválik az összhangból. (Egy
jó irányelv, ha nem hallod a saját hangod, akkor túl halkan kántálsz, ha
csak a saját hangodat hallod, túl hangos vagy.) A kántálások szövegei
felidézik a Buddhát és a tanításokat, ez a gyakorlat az elmét jótékony
gondolatok irányába vezérli. Az ilyen rendezett, szimbolikus ceremónia a
beszéd és test ritmusát használja, mint alkalmas eszközt arra, hogy
kitisztítsa az elmét a meditáció csendje előtt.

A pontos rutin kolostoronként változik. A Szumédháráma kolostorban
Portugáliában a reggeli mediáció 5 órakor kezdődik, mikor egy óra
meditációval kezdünk, ami alatt nincs beszéd vagy kántálás. Amikor
belépsz, csend van, egy órán át belső vizsgálódásra szánt megosztott
tér, amíg a rangidős szerzetes megcsengeti a harangot a meditáció végén,
amit 15-20 perc kántálás követ.

`Nem unalmas egy idő után?' Időnként egy iskolai program egy egész
osztály gyereket elhoz a kolostorba, hogy csendben meditáljanak (talán
azt remélve, hogy később csendesebbek lesznek), és ők valószínűleg agyon
unják magukat. Kezdettől fogva nem érdekelte őket hogy ott legyenek, de
a gyerekek okosak, és elviselik a felnőttek furcsa ötleteit.

Az unalom megváltozik amint ránézel. Amikor meditálni jössz, az
érdeklődés vezérel, hogy tanulj önmagadról és az elmédről, és
közelebbről megvizsgálva, az `unalmas' elég érdekessé válik. `Nem sok
minden történik, csak a lélegzés. Probléma ez nekem? \emph{Én} hozom
létre azt a problémát? Meg tudom állítani, hogy problémákat gyártsak
magamnak? A légzés tulajdonképpen egy árnyaltan gazdag, kellemes érzés.'

Miközben a légzésre való éberséget gyakoroljuk, megjelenik az öröm, ami
az érzékek visszafogottságából születik. Az elme ellazul, és a
gondolkodást engedhetjük megállni. Csendben vizsgáljuk a
tapasztalatunkat, nincs szükség azt kommentálni.

Az unalom a tényezők egy kombinációja: a vágy az izgalomra, a jelen
aktív elutasítása, és azt a hozzáállás, hogy már tudjuk amit kell, nincs
itt semmit új. Nem a helyzet magából eredő tulajdonsága, hanem a
képzetlen, nyugtalan elme szokása. A Buddha ahhoz hasonlította, mint
ahogy egy elefánt érzi magát, amikor az állatidomár első ízben megfékezi
a mozgásban úgy, hogy kiköti egy erős oszlophoz. Az elefánt alkalmatlan
a képzésre, amíg folyton arra vágyik, hogy a vadonban kóboroljon arra
amerre csak akar, de egy jó idomár fokozatosan megfékezi a
nyugtalanságát amíg meg nem tanul nyugton maradni.\footnote{\href{https://suttacentral.net/mn125/en/sujato}{MN
  125}} A szuttában, Dzsajaszéna herceg el sem hiszi, hogy a belső békét
lehetséges elérni az érzékek visszafogásán keresztül, hiszen a
palotában, ahol él, a figyelem elterelő szórakoztatás veszi őt körbe, és
nem tapasztalt még ilyen békét.

A meditációs terem ajtaja mindig nyitva van, bármikor felállhatsz és
kisétálhatsz. De azért vagy ott, mert korábban az elméd arra kóborolt
amerre csak akartad, de érezted, hogy végeredményben ez elégtelen volt,
és a képzetlen elme állandóan fájdalmas hibákba és gondokba kevert. Ha
ezer lépést teszel ezer irányba, csak elfáradsz, és mérgelődhetsz, hogy
miért nem jutottál sehova. Helyes dolog felismerni a szükséget, hogy
saját nyugtalan elménk idomárai legyünk, megtanuljuk mi a helyes irány,
és arra felé tegyünk lépéseket.

\begin{quote}
A nehezen megfékezhető,\\
csapongó, vágyűzött elme\\
ellenőrzése jó. A megfékezett\\
elme boldogságot hordoz.

Dhammapada 35.\footnote{Ford. Fórizs László}
\end{quote}

\section{Oltár}

Nem mindig volt Buddha oltárom a szobában vagy kunyhóban ahol éppen
szállásom volt a kolostorban. Azt gondoltam, hogy az intézményes
elvárásokhoz való igazodásban volt szerepük. Így többnyire figyelmen
kívül hagytam őket, és némileg nehezteltem a képekre és szobrokra, mert
úgy éreztem mások azt várják el tőlem, hogy tiszteljem azokat, és
rosszindulatúan nem akartam azt tenni, amit (úgy gondoltam) elvárnak
tőlem. A reakcióm olyan volt, mint az iskolás gyerekeké: elég okos
voltam ahhoz, hogy elviseljem a szimbólumokat, és elég arrogáns ahhoz,
hogy azt higgyem én már tudom mit jelentenek. Aki okosnak gondolja
magát, felületesen elutasít mindent, és unalmassá válik számára a világ.
Ez egy ön-butító kombináció, azt gondolni hogy \emph{én már tudom}
bezárja az elmét, így nem tudsz rájönni, hogy nem tudod. A brit
pszichológus Iain McGilchrist ahhoz hasonlítja ezt, mintha beragadtunk
volna egy tükrökből álló labirintusba: csak azt látod, amit te mondasz
magadnak, és sosem találod meg a kiutat.

Egy kis repedés jelenhetett meg azokon a tükrökön, mert észrevettem,
hogy tulajdonképpen senki nem gondolt rám ilyen ítélettel és elvárással.
Én magam hoztam létre a történet mindkét oldalát, és olyasmi miatt
emésztettem magam amit csak képzeltem.

Egy Buddha oltárt készíteni egy kis teret hoz létre a helyen ahol élünk,
egy emlékeztető arra, hogy álljunk meg a rohanásban és adjunk teret a
felébredésnek. A meditációs terem ugyan ezt az üzenetet adja át nekünk a
csenden keresztül. Egy oltár ajándék nekünk, magunktól. Nem azért van,
hogy mások elvárásainak feleljen, vagy akár a Buddhának. A történelmi
Buddha 2600 évvel ezelőtt elhunyt, és túl van azon, hogy tőlünk bármire
is szüksége legyen. Másoknak van elég dolguk amin aggódhatnak, és nem
gondolnak annyit ránk mint azt képzeljük.

Emlékszem arra gondoltam, `Miért nincs helyem a Buddha számára ott ahol
élek?' Neki is láttam levágni néhány fadeszkát és készítettem egy kis
polcot az oltárnak. A Buddha oltárak gyakran elég egyszerűek: egy vagy
több Buddha szobor, gyertyák, füstölő és virágok. A Buddha jelképezi a
felébredett tudatosságot az emberi formában, a gyertyák a bölcsességnek
felelnek meg, ami láthatóvá teszi a dolgokat (mint a fény a
sötétségben), a füstölő az erényt jelképezi,\footnote{Dhammapada 54: Se
  szantálfa, se tagara, se jázmin illata nem száll a szél ellenében, a
  jóság viszont szembeszáll a széllel, a jó áthatja az egész világot.
  (Ford. Fórizs László)} a virágok a boldogság örömét kölcsönzik,
elhervadásuk pedig emlékeztet minket az állandótlanságra.

Felajánlom ezeket a vizsgálódó szavakat azzal a szándékkal, hogy
bátorítsanak a gyakorlásban. A tanító a Buddha, a megvilágosító
magyarázatok forrása hozzá vezet vissza. Hálás vagyok azért, hogy a sok
évszázad kavargó káoszán keresztül a tanításait egyik generáció a másik
után tovább hordozta a mai napig. Remélem, hogy segítséget nyújtanak
majd abban, hogy az elménk zajos kavargását átalakítsuk megértő csenddé.
