\hypertarget{cycles-1}{%
\chapter{Cycles}\label{cycles-1}}

Meditation teaches understanding through the feelings and experiences in
the present.

The instructions are given in steps, but these steps are not the purpose
of meditation. The purpose is clear knowing of the present experience,
restoring right perspective. We can develop an impression that we should
always complete the same sequence of steps, and when the mind doesn't
develop according to that sequence, we feel disappointed.

Turn that attitude around and start with experience. If our experience
is the base, the way it is, then what understanding does that give us?
When painting a wall, we look at the wall first, choose the right paint
for it, and \emph{then} follow the advice on the paint can. The wrong
paint will just peel off, won't it? We look at the mind first, how we
are feeling, what state we are in, and respond intelligently to that.

The steps are a method of learning through imitation, following an
example we watch ourselves and see how things work. When we feel pain
and suffering, and either we are able to resolve it or wait it out
patiently, we will know that was not just imitation, and our confidence
in the practice will grow. We have learnt something there, and we are
not going to hold on to the details of the first tutorial.

It would be great if our meditation would develop in a linear way. We
imagine that we are going to sit down, perhaps a bit unfocused, and
after an hour, \emph{if we are good meditators}, we are going to feel
stillness and our mind is going to be clear and focused.

Later, when we recollect how the meditation went, we observe that this
is not what happens. Our experience doesn't develop in a linear way from
shallow to deep, or from distracted to focused. We think this is our
fault, because we are not good meditators, or we are not doing the steps
right.

As soon as we try to follow steps, everything is different from our
expectations. So maybe we are not trying hard enough? We keep pushing
and it only gets more painful. This is the feeling of trying to fit an
opinion onto the experience.

The mind develops in cycles, and these cycles ignore our goals to
develop a meditation career.

Take experience as ground truth and start from there. What is the
experience? We can watch how attention moves as an activity. It
progresses through cycles rather than linearly. First the mind is
content to sit and attention becomes clear and still, then thoughts
start coming up and we follow them, then we stop and stay content with
the stillness again, thinking might even stop without us noticing that
we are not thinking, and then attention starts moving and we notice
ourselves thinking again. Memories, desire, restlessness can come up and
we notice we have work to do so we deal with them, then the mind is
again content and returns to stay with the stillness.

Some knowledge is necessary, but a little is enough. Remembering the
teachings of the Budda is a treasure which doesn't run out. But this
knowledge doesn't become ours, and we can't save it. Whatever we learn,
every time we start again from the begnning, and trust the present
knowing from there.

The mind would like to say that it is a good meditation, or a bad
meditation. It wants to create a distinction and name it. This
preference creates dissatisfaction, and we try to improve our
experience. We decide how our experience should be, and strive towards
creating a state. There is an image, a concept, a named distinction
which we decided and from our current state of not being like that, we
want to change our state and become like that. This is the dissatisfied
mind. It wants to become something, it wants to arrive at a state and
have a name.

This would continue, this seeking for a distinction, this seeking for a
name, this seeking for a state, this would continue until we notice that
it is happening.

When it becomes visible in consciousness, that we are doing this, then
it stops. It stops because ignorance, not seeing, was replaced with
seeing. Seeing is enough to stop that compulsion from continuing.

This is enough, knowing the mind this way we arrive at a sense of
gratitude for being. Not a gratitude for anything in particular, just
that there is experience, there is knowing, there is clarity, and the
freedom which allows us to stop going towards more and more, or a
different thing.

In a balanced posture the feelings of the body are easy to observe. We
turn attention inwards, in a curious manner, wanting to discover.

These feelings are often not clear. We experience them but they don't
have clear boundaries. They don't have an edge, or a definite shape.
When we try to find names for it, we struggle, we are not sure what to
call it.

All the symbols which could be names are inadequate. In our culture we
are trained to trust knowledge, and we like to go back to that security
with names and terminology. We are not familiar with the cognition that
doesn't use names and symbols. The feelings, the experience itself is
not clearly defined, just the fact that we know that there is this
experience.

Seeing it this way we can distinguish the naming process. The experience
of the body is rather amorphous, it doesn't have edges. Breathing in,
the experience is everywhere at once. Breathing out, the experience is
everywhere at once. The whole body is breathing and there is feeling and
expereience, but there are no clear names.

The symbols constructed by the mind can only go so far. This is not our
fault, every symbol constructed comes with its own limit. Dedication to
finding them and trying to identify what kind of consciousness is this,
is a limited effort. It is bound by the limits of categories.

We see this limitation and we rather stop doing it. We will rather enjoy
staying with the knowing which includes experience without filtering.
Dropping the naming process, we recognize we can simply know these
feelings. We can know that there is experience, without having to find a
name for it. This is on a different level than the symbols themselves,
then the words and concepts.

Sitting still, breathing in, breathing out. It becomes easy to recognize
the mental states. Recognize the feeling that comes with unwholesome
mental states. A sense of heat, restlessness, dissatisfaction and
anxiety.

We know that there has to be patience, there has to be endurance with
that state. It will cease, it will change. We can wait for it. When we
know where we are, then we don't have to do much more. Conditions in the
mind will change on their own. If we are not putting more fuel on it,
the fire will burn up and cool down on its own.

The result will be a wholesome mind which understands. Not being
compelled, not being forced, we recognize it by the coolness and comfort
of being free.

The Buddha had given step-by-step instruction on how to develop
mindfulness of breathing. First, it starts with simply noticing the
breath, knowing the short and long breath. Then it guides us through
contemplating the body, the feelings which arise, the present state of
mind and the nature of phenomena, while staying grounded in the
breathing.

What is at the end of the instruction? Where does all this develop to?
The last step is relinquishment. 'I shall breathe in, contemplating
relinquishment. I shall breathe out, contemplating relinquishment.'
After mindfully knowing impermanence, dispassion and cessation, the
blessing is in letting go.

The great teachers are our examples. They didn't meditate to achieve a
special state and then do something else with their life. Meditation
doesn't separate out, but integrates into their life. The Venerable
Sariputta said he abides in the perception of emptiness, the Buddha said
he abides in the signless concentration. They contine to meditate.
