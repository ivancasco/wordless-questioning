\clearpage

\section{Az Összes Káros Folyamatról (részlet)}

{\centering
\emph{\href{https://a-buddha-ujja.hu/mn-2/hu/forizs-laszlo}{MN 2}, Sabbāsava Sutta}
\par}

{
\setlength{\parindent}{0pt}\setlength{\parskip}{5pt}
\fontsize{9.5}{14}\selectfont

\emph{(A káros folyamatok megszüntetése az alapos figyelmen alapszik)}

Bizony mondom, szerzetesek, a káros folyamatok megszüntetése csak azok számára
lehetséges, akikben megvan a tudás és látnak; akikben viszont nincs meg a tudás
és nem látnak, azoknak nem lehetséges. Mit kell tudni és látni, szerzetesek, a
káros folyamatok megszüntetéséhez? Ez az alapos, mindenen keresztüllátó elme és
a nem alapos, nem mindenen keresztüllátó elme különbsége. Amikor a szerzetes nem
elmélkedik elég alaposan, új káros folyamatok keletkeznek, a már létrejöttek
pedig felerősödnek. Amikor a szerzetes alaposan elmélkedik, nem keletkeznek új
káros folyamatok, a már létrejöttek pedig eltűnnek.

\emph{(Aki nem ismeri a Tant, a nézetek sűrű bozótjába gabalyodik)}

Nem elég alaposan így elmélkedik:

`Léteztem én a múltban? Nem léteztem a múltban?'

`Mi voltam a múltban? Hogy léteztem a múltban? Abból, ami voltam, mivé lettem a múltban?'

`Létezni fogok a jövőben? Nem fogok létezni a jövőben?'

`Mi leszek a jövőben? Hogy fogok létezni a jövőben? Abból, ami voltam, mivé leszek a jövőben?'

Vagy most, a jelenben, kételyekkel a bensejében:

`Létezem én? Vagy nem létezem? Mi vagyok? Hogyan létezem? Honnan jött e lény? És hová tart?'

Mivel ezeket nem fontolja meg alaposan, e hat nézet valamelyike alakul ki benne:

`Van átmanom'\footnote{átman: `önmaga', az (ami mindig) önmaga (marad).} -- úgy alakul ki benne ez a nézet, mintha igaz és megalapozott lenne.

Vagy:
`Nincs átmanom' \ldots{}\\
`Az átmannal fogom fel az átmant' \ldots{}\\
`Az átmannal fogom fel a nem-átmant' \ldots{}\\
`A nem-átmannal fogom fel az átmant' \ldots{}\\
`Ez az átmanom, ami beszél és érez, ami mindenütt megtapasztalja a jó és a rossz tettek gyümölcsét – nos, ez az átmanom állandó, stabil, örökkétartó, nincs alávetve a szüntelen átalakulásnak, örökké ugyanaz marad'.

Ezt nevezik, szerzetesek, a nézetekbe gabalyodásnak, a nézetek sűrű bozótjának, a nézetek erdejének, a nézetek zavarának, a nézetek kínlódásának, a nézetek béklyójának. A nézetek béklyóival megkötözött, tanításban nem részesült, evilági ember nem szabadul meg a születéstől, öregedéstől és haláltól, a bánattól, a jajveszékeléstől, a fájdalomtól, a csüggedéstől és a kétségbeeséstől. Bizony mondom, nem szabadul meg a szenvedéstől.

\emph{(A nemes szívű tanítvány bölcsen elmélkedik)}

Bölcsen így elmélkedik:\\
`Ez a szenvedés';\\
`Ez a szenvedés keletkezése';\\
`Ez a szenvedés megszűnése';\\
`Ez a szenvedés megszűnéséhez vezető út'.

Amikor bölcsen így elmélkedik, a három béklyó -- az önvaló nézete, a kétely és a
vallási szertartásokhoz ragaszkodás -- megszűnik benne.

\clearpage

\emph{(Konklúzió)}

Szerzetesek, arról a szerzetesről, aki belátással \ldots{} megfékezéssel
\ldots{} használattal \ldots{} béketűréssel \ldots{} elkerüléssel \ldots{}
eltávolítással \ldots{} gyakorlással, megszabadult a káros folyamatoktól, --
arról a szerzetesről mondják, szerzetesek, hogy megszabadult az összes káros
folyamattól, megszüntette a sóvárgást, kioldozta a ragaszkodás béklyóit, és
tökéletesen átlátva az önteltségen véget vetett a szenvedésnek.

\bigskip

{\raggedleft
\emph{(ford. Fórizs László)}
\par}

}
