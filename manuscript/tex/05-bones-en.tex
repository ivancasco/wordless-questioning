\chapter{Bones}

\keywords{contemplating the body, perception of self}

Even if we are sitting in good posture, sooner or later, something
somewhere is going to hurt. But we are not doing this to torture
ourselves. You might change the posture to standing, and allow the
joints and the muscles of the body to relax. Standing requires more
attention to maintain balance, and the bones -- the supporting
structures of the body -- are easier to feel.

The bones are fascinating because they make up the core, the rigid parts
of the body which determine its shape and what it can and cannot do. We
look in the mirror and think, `that's me'. How thoroughly did we examine
the image before seeing ourselves in it? A brief glimpse of its outline,
contour and colour will already trigger the perception of `me'. And
since the rate of change is slow, we rely on fewer and fewer key
features to recognize the image in the mirror.

Leaning closer, noticing more features, or seeing ourselves from an
unusual angle, we might need a minute to decide who we are looking at.
If the bones were slightly different, the body would be a different
shape. Such a change would change not only our appearance, but also how
we live.

\keywords{standing meditation, standing posture}

Stand in an upright but flexible manner and take some time to find your
balance. Place the feet at shoulder width, with the knees relaxed and
slightly bent, being engaged in holding the body. Don't let the knee
joints lock up straight. This stresses them and will cause your posture
to become rigid. Keep the feet parallel, with the toes pointing straight
ahead.

Sway the body left and right a bit and feel the centre of weight.
Develop a sense that you are holding the body upright, that you are
preventing it from falling. It is better to balance the weight of the
body toward the heels, rather than on the balls of the feet.

Hold the shoulders wide enough to open the chest for easy breathing, but
not so wide that they become tense. A comfortable place for the hands
can be for example on the thighs, at about over the place where the
pockets are on a pair of jeans.

Allow yourself some flexibility and make small adjustments to your
posture as your muscles get used to the situation. Feel out your balance
in standing and watch, take notice of the body as you hold it this way.
Gravity is pulling it down. There is pressure on the ground. Let your
hands rest in front of the abdomen, one palm comfortably on the other.

The eyes may be open or closed. If you feel drowsy, you might prefer
meditating with eyes open, but keep your gaze lowered, looking only a
couple of meters in front of you. If you look straight ahead, your
attention will be directed outwards, and various movements such as those
seen out a window will be distracting.

If you close your eyes but it feels strained and painful, pay attention
to where the eyes are focused when you close them. If they narrow in
close, as if focusing on something close behind the eyelids, this causes
the inner muscles of the eyes to strain and dry. This tension and
dryness can even cause the eyes to water up with tears.

On traditional Buddha statues, the eyes are depicted as being slightly
open. He is awake, not sleeping. Instead of shutting your eyelids tight,
practise relaxing them. Let them stay lowered without any pressure.
Though your eyelids are mostly shut, imagine looking at something far in
distance. This lets the eye muscles relax. To allow some light in, you
can try allowing a narrow slit to remain open. It can also help to
massage the inner eye muscles a bit with the tips of the thumbs in a
circular motion.

Breathe in, and watch how your posture changes with the movement of the
breath. The diaphragm muscle pulls in the air and the abdomen moves
forward to give way. The shoulder bones rise, the ribs in the chest open
outwards, and your body's centre of gravity shifts slightly.

Is there something limiting the breathing? Take care to stand upright
and don't let the shoulders hunch as this blocks the open breathing.

Take note of the balance of the head and try to find the position where
the head sits on top of the spine of its own weight, not leaning forward
or pulled backward. Instead of looking directly ahead, direct your gaze
slightly down in front of you, so as to mitigate your attention becoming
distracted by any comings and goings. Pulling in the chin, direct your
gaze a couple of meters in front of you on the floor. Allow the crown of
your head to rise up a bit, as if pushing the sky.

The position of the head controls the posture of the upper body to a
great degree. Through years of sitting on chairs, we have developed the
habit of pushing our heads forward which creates tension in the back
muscles along the spine. Although we can't control these muscles
consciously, we can experiment. What does it feel like to pull the head
back a bit? Can you feel out the balance where the muscles in the back
relax.

In a balanced posture, the vertebrae of the spine sit one on top of
another, like carefully positioned stones. With the spine aligned
upright, gravity is enough to keep it settled in place. Such a posture
gives us a pleasant feeling of light balance without forcing.

\keywords{attitude to the body, goodwill}

If our attitude is too analytical, it can conjure up grotesque and
shocking impressions of the body, but a sense of goodwill towards
ourselves can keep the meditation balanced and wholesome. The intellect
functions by building abstractions and separating out what it observes.
When viewing the body, it sees our various parts as abstract and
inanimate objects. Keep in mind that we are not practising
body-contemplation to create aversion or alienation to the body. We are
practising meditation with clear comprehension in order to see things in
their context, nothing is isolated in a vacuum. Our body-awareness
includes rather than excludes and must come from a welcoming attitude of
acceptance and goodwill.

This difference of attitude reminds me of a story about a dialogue
between Plato and Diogenes. Plato was giving a lecture to his students
at the Academy in Athens, where he defined men as `featherless bipeds'.
Diogenes happened to overhear this and, being fond of practical jokes,
he brought a plucked chicken to Plato and held it up in front of him
proclaiming, `Behold! I've brought you a man.' He must have thought that
this would show that a certain context was lacking in Plato's overly
intellectual definition. The group at the Academy added `\ldots{} with
broad flat nails' to the definition, in an attempt to satisfy their
scholastic sensibilities. This amended definition still, though, misses
Diogenes' point.

\keywords{experiencing the whole body}

We watch the sensations in the body as we breath in and as we breath
out. Turn attention inwards, mindful of the perception `the body is like
this'. The mind is not seeking, not going off around the world
somewhere. It does not need anything, what is here is enough. Awareness
sees the body, from the feet, to the legs, abdomen, chest, arms,
shoulders, the neck and the head. The body is one whole, one changing
perception, sensitive to the breathing.

Watching the body like this is like watching the rain. There is nothing
to do, nothing to decide. The rain just goes on without us having to get
involved.

\keywords{clear comprehension, awareness of the body, defusing anger and desire}

The Buddha compared the effect of awareness on the mind to rain: it
settles the dust and clears the air. Awareness of the mind stops
unwholesome mind states from arising, develops wholesome mind states,
this way purifying the heart. We may notice that our experience of the
world is not fixed: we are not isolated outside observers, looking onto
a world which is separate from us. We are part of the world we
experience, since we bring it about through our mode of attention.

When clear comprehension is established and you notice the mind becoming
more clear and stable, review what allowed this change? What did you do?
What did you \emph{not} do? You didn't have to fight or manipulate the
sense experience or the thoughts and emotions, since they change through
the change in the mode of attention. Our mode of attention creates the
frame of reference from which we experience the world of the senses
through perception and memory. This is a process that conditions an
attitude, like a function operating over time, which produces how we
recognize and interpret ourselves in the present.

Shifting our mode of attention can serve to stop providing unwholesome
mind states with more fuel. From the perspective of direct experience,
and in accord with the way things are, such unwholesome states are then
denied a basis or reference for their continuance. This is the long form
of the short saying that, awareness of the mind purifies the mind.

Staying with the awareness of body defuses both anger and desire. It
changes the frame of our attention and such mind states then fall flat
as though the carpet had been pulled out from under them. The busy,
thinking mind is like a noisy show, or the news in last year's paper.
The topic is no longer interesting, it has lost its urgency, it keeps
going around the same circles. Put the thinking down, like a weary hiker
their heavy backpack, and continue mindful awareness of the body.

\enlargethispage*{2\baselineskip}

Periodic distractions and daydreams can occur, but keep returning to the
breath and the physical sensations of standing. If while standing, you
begin story-telling or fantasizing until the bell rings, that's not
practising insight meditation\ldots{} it is practising waiting for the
bus.

\clearpage

\keywords{memory as self, narratives of self}

Investigate your state of mind as an experience. The perception of your
body and its feelings arise before we construct the perception of self
from it. What do we remember about ourselves? If we forget about the
narrative that someone told us yesterday, or if we recollect being with
friends years ago, do we perceive ourselves differently? This
interaction between our feelings and our mind states keeps changing.
Current perceptions keep changing, and recognizing awareness places
trust in a place which knows this change. This allows us to see from a
wider frame, where there is no fear of the change. Creating the
perception of our self is an ongoing process. We take an active part in
it through actively recalling and re-creating memories, our narration of
those memories, and what we choose to do as the next action in the
present.

\keywords{bones, parts of the body, judgements of appearance}

Observing the body and its parts, our minds stay with these changing
perceptions before the creation of a self. This process disarms the
self-judgement, fears and expectations that bog us down.

Notice the feeling of how the bones connect. There is this perception of
an inner structure which supports the body from the inside: rigid
pieces, connecting end to end, and stacked on top of each other. There
are sensations in the legs: rigid perceptions denoting the long leg
bones. There is pressure. The hip bone is resting on top of the legs and
the torso moves joined above all this. The rib-cage expands and
contracts with the breathing. The spine is holding the weight in a
curve. The head is sitting on top of the spine. The skull bones are
stretching the skin of the face.

Our body is made up of pieces. In some places, these pieces are hard and
rigid. In others, they are soft and flexible. The combination of these
is what gives our body its shape. When we look at a person, all we see
is hair of the head, hear of the body, nails, teeth and skin. And we
then construct a person from it all. We glance at a mirror for a
fraction of a second, recognize the general outline or notice some
particular feature, and think, `That's me. How do I look?'

In some situations, we can notice the gradual steps of how this
perception builds up as when we see someone walking in the fog. First,
we recognize the shape of a `person'. Then we detect `male' or `female'.
Maybe it is someone we know? Some detail triggers the final recognition
of our friend and their name. This entire process played out in the
realm of perception.

The habitual perception of the body - both of our own body and of other
people's bodies -- is that we see it as one unit, one thing. From that
perspective develops an obsession that there is some ideal way that it
should be. We imagine that the body has to be a certain shape, a certain
size, and so on.

These are worldly judgements, perceptions which our society has drilled
into us. Some cultures idealize a thin body, others a plump one, and
these cultural ideals keep changing from one generation to the next.
Advertisements and various messages from the media reinforce these
expectations and we dutifully believe in them. But when we look closer,
we see that such perceptions are twisted and not in accord with reality.

These social expectations create anxiety about our appearance and make
us concerned about what other people think about us. But how much are
\emph{we} concerned about the appearance of others? If I watch myself, I
don't think much about how other people look. But I can feel
self-conscious and imagine that \emph{they} must be thinking about
\emph{me}. When, in fact, they think about me as much as I do of them --
not much, if at all. They are occupied with getting on with their own
life, just as I am with mine.

Besides the pressure of our self-judgment, we also imagine how others
are judging us. And since we can't know and can't control what they
think, we internally ruminate in the mind about it, which creates an
illusion of such knowledge and control. When we play out these inner
dialogues, we enjoy the illusion of control. But we miss out on the
freedom of letting go of \emph{the need} for that control.

\keywords{body parts as not-self}

We can notice the conditioned nature of this anxiety when various parts
of the body become detached from their usual locations. We can be so
concerned about our hair, for example\ldots{} but only when it is on our
head. When it is cut off, we are not anxious about that pile of hair on
the floor. Similarly, when cutting our nails, what is that point when it
is no longer `me' and `mine'?

When we contemplate the body in this way, we see it not as one unit, but
as made up of pieces and parts which have their own nature and behave
accordingly, each part not the least concerned with our opinions or
those of others. Bones, skin, hair, teeth and nails: they are the way
they are.

The body is a blessing. This meditation is not meant to develop aversion
or negative emotions toward the body. Health is a blessing, it supports
us in everything we do. The Buddha called health the greatest treasure.

\keywords{stories as dreams, awareness of the body, grey and drifting states, gratitude}

We observe the breath, the parts of the body, and our present
experience. When we look, we find that they don't carry the stories of
`me' and `mine' with them. Since it is we who are creating these
stories, we can also stop creating them. When we do this, we can breathe
with relief that we are not chained to following them. Phenomena arise
through dependent conditions. When the conditions cease, the phenomena
cease. This is all that happens.

Awareness of the body loosens the grip of our desires and leads us to
recognize that we are fortunate to be here. We can always return to this
attention: one in-breath and out-breath is enough to remember arising
and ceasing. And with this, our doubts become like old stories. Over
time, we get tired of untangling all the threads which are so difficult
to follow. It is like interpreting someone else's dreams. What is real,
is always here in our present experience. What becomes important is not
who or what we are in the story, but whether we can give our attention
to where we are now.

Clear intention has an important role. When we don't set a clear
intention, we are drifting. Perhaps we don't particularly mind being
here and drifting like this, but the mind is grey with no life, almost
trying to hide itself and be invisible. We end up being grey and
invisible like that. Nothing wrong is happening, but there isn't any
brightness and joy in being here.

We don't stop often enough to notice when we are happy and peaceful.
When the mind is clear and calm, consciously decided to be there with
our experience, the natural feeling is a sense of gratitude for what is
here, and for the blessings we have received in our life.

Gratitude is not created by will. In this practice we are not creating
anything, we simply recognize what is here with clear intention. It is
not a matter of strength or ability as those are bound to time and
circumstance. But resolution and mindful attention are not bound to a
given circumstance. The result is a right perspective in which we can
see the right order of things and what to do with them -- or to stop,
give attention and breathe.
