\hypertarget{too-much-1}{%
\chapter{Too Much}\label{too-much-1}}

We sit down to meditate and turn attention inwards. What is the first
thing that we are aware of? We want something, or upset with somebody,
or feeling dull and sleepy, or trying to figure out what to do and why.
Other people seem to be sitting peacefully. How can they meditate at
all?

We meditate because we want to see, and often what we see is that we
better tidy up our mind because there is no space to even sit down in
there.

We often find that what we have to learn is something we never noticed
before. Something we have been doing for a long time, and only now we
start to see it.

The meditation object show us where our mind is. A quiet room with
nothing to do seems like the perfect place to spend on hour thinking
about our some drama or playing out dreams and fantasies, but when the
bell rings, do we feel it was worthwhile?

Rather, we feel a bit groggy and disoriented, like waking up from a
dream, and now we have something else in our schedule. Remembering the
value of the time to meditate lets us keep returning to the meditation
object. Remember there was a reason, some sense of lack or suffering why
you wanted to meditate.

We know that we are touching something real when it hurts. We have the
courage when we look, to develop a welcoming attitude toward difficult
feelings, "Oh great, I can finally see something! Finally getting to the
bottom of this old thing. It's not pretty but finally I know."

When we experience this, everything is going according to plan. They
were caused by our actions in the past, this is the result, and we are
doing the best possible thing by becoming aware of them.

The power of awareness is to break compulsions. When we can't see, the
mind runs itself in the old program and we only notice when the results
are painful. When we see it, then we have the freedom to stop.

Trying to find out who, what and when, is fruitless. Ruminating on the
drama only pulls on the hooks and they dig in deeper. These questions
don't have a way out. This is not the direction how to contemplate.

Then we look for the right recipe. "What meditation technique should I
practice? How can I reach a deeper level? There must be a teacher who
can tell me what to do." We can recognize these thoughts, and know that
we are in doubt. Stepping back and knowing the mind is experiencing
doubt.

And sometimes there is something we know we have to do. Asking for
forgiveness if we made a mistake, or forgiving a mistake and moving on.
Acknowledging a situation and resolving the built-up tension in a
patient, skilful manner.

Trusting the truth means the trust in cause and effect, that there is an
order, a natural law according to which things play out, a truth which
we didn't create but we can recognize and trust. Skilful actions result
in the best possible outcome, even when that means confronting what is
difficult, knowing that avoiding that difficulty will only let it
continue longer and create more pain.

But when it comes to the hindrances and defilements of the heart and
mind, we are often simply getting in the way with our restlessness to do
something.

Awareness breaks compulsions, patient endurance lets the fire burn
itself out. Often there is nothing to do. The results are felt, but that
is not in our control.

The simple practice of waiting for the fire to burn down is powerful.
"Should I say something? This time I really should tell them. I know I
am right." We would rather like some action and drama, but that
jumpiness created the fire in the first place.

Our intelligence is limited when we are excited or upset. If we can't
wait, if we are so compelled and driven that this can't wait until we
cool down, then what good can expect to come of it? Maybe we can't think
of anything good but we can remember to wait until we see better.

If we can wait, that often solves half the problem already, which was
us. Maybe there will be something to do, but often there won't be,
because the problem was entirely in our excitement or upset.

It makes a difference what we trust and what we believe, even when we
haven't yet confirmed it and can can't say for sure.

A sense of trust in the teachings and the good-will of others is
essential.

The Buddha said his teachings are only about two things: suffering and
the end of suffering. He had given his teachings motivated by
compassion, because he believed this was going to help those who wish to
understand.

He teaches that results appear from a cause, not without a cause. If
there is a cause for something to be there, it will be there. If there
isn't a cause for it to be there, it will not be there.

Whatever happens, if it is not in our control, it happened through
causes, and this was either a necessary outcome in some way, or we
caused it to ourselves in the past, but trust that with patience or
skilful actions it will work out in the best possible way, and this is
the best thing which we can be experiencing now.

The part which is ours, we can take care of. The rest will work out in
whatever way it must. We don't have to create a solution, that is not up
to us.

This rests on the foundation of virtue, a wholesome environment and a
trust in truth. We examine our conduct, abandon unwholesome habits and
develop wholesome ones. We examine our environment, avoid the company of
malicious people and cultivate the company of virtuous people. This
create a safe environment to live and for going through difficulties.

Ask yourself, "Can I find the space in my heart to be with this, as long
as it needs to be here?" You probably can, but you have to ask the
question.
