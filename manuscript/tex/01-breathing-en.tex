\hypertarget{breathing-1}{%
\chapter{Breathing}\label{breathing-1}}

We watch the sensations of the breathing in the body, this alert
attention collects the mind around a stable object. The physical
sensations in the body are easy to notice as we breathe in- and out.

Why are we exhausted with all the thinking? The mind is jumping from
thought to thought, but we don't know where we are going and so we never
arrive: restless desire is exhausting. Sense-restraint gathers energy
and directs it, instead of letting it flow in every direction. Directing
attention to a neutral, steady sensation slows down the thinking mind.

Sit in a balanced, upright posture. The bones in the body sit on top of
each other like a small tower of stones. When balanced carefully, its
weight at the centre, we don't have hold it by muscle power, gravity is
enough to keep the body upright.

Breathing in, the cold air touches the tip of the nose. Let the abdomen
muscles draw in the breath, instead of expanding the chest to gain
volume. The air moves through the lungs and the abdomen expands; this
way of breathing lowers the heart rate and reduces anxiety. Relaxing the
muscles, the air leaves through the nostrils. There is no need to
control it with precision, it is enough to suggest this rhythm. The body
knows how to breathe, we take a step back and observe, like watching
waves wash in onto the shore, and then recede.

We don't have to tell ourselves what to think and how to feel. If we
want clear thoughts, it is best to first be silent and listen. We sit
and rest for a little while. When we are silent, either clear thoughts
will come on their own, or the mind will be content to stay with the
silence.

Restraint and directed attention is necessary for clear, conscious
thought -- and it brings peaceful gladness with it. The mind is content
and happy, we don't need much internal discussion. Sitting and
breathing, listening to the silence is a blameless joy in itself.

The right attitude is a careful balance. We establish a clear intention
to stay with the meditation object and leave other matters until a later
time, while maintaining an open attitude. Pushing and forcing ourselves
is not sensitive enough to what is happening. Operating from will-power
becomes rigid and obstructive.

A clear mind and good aspiration feels settled and cool, open for
changes. A forceful struggle feels busy and hot, narrow in scope.

Are we learning new facts about the mind by observing the breathing? I
remember when I sat down and struggled to understand how to do it. I
kept thinking about how to change my breathing to improve my meditation,
expecting that one day I will somehow hit the correct buttons, and doing
it in the correct way I will start learning new information, new facts
about the mind. It was rather painful and entirely fruitless.

Analysing it for knowledge we miss what is happening. Think about a
conversation, when the other person keeps asking ``Why?'' after your
every sentence -- the conversation goes nowhere without listening.
Overthinking it, we are doing this to ourselves, and no wonder we want
to jump up and tell our commentating mind to stop and listen.

Frustration and disappointment are useful indicators to listen -- the
best is when we have to learn something we didn't expect to learn. Or
are we waiting for nature to change itself according to our
expectations? The facts that we need are little, but developing skill in
them requires much practice.

We don't stop to stay with them, and unpack their significance for
recognizing our situation, what is skilful to do there, and in what way
to do it. A list of facts, if not integrated, doesn't reach deep enough
to deal with root causes in the heart and mind, and have no effect on
us. Watching the breath stops us and opens the attention which can do
that. Perception and recognition can be gradual, like having to lean
close to something to see an essential feature, but every step is
interesting and leads onward.

The Buddha has a simple message for us: wake up, stop holding on, you
don't have to suffer. We keep unpacking this, unfolding it wider and
wider.

Adjust how you sit and find a balanced posture: an upright position, in
a stable but not tense posture, with the head balanced and not lulling
forward. The posture should allow open, easy breathing.

Determine that you are putting down everyday activities for this period.
If such a thought interrupts you, respond, `This is not the right time,
I will come back to that when the time is appropriate.' It's a bit
longer than `Go away!', but more friendly to ourselves. This establishes
a clear intention in the mind, like when clearing a desk before starting
work. We are putting them down not because they are not important, but
because if we don't clear our head, we can't do our work well.

Take a deep breath and watch if you feel tension, something obstructing
or limiting the breath. If it moves easy and open, then your posture is
suitable. You don't have to sit in a special way.

Pay attention to the physical sensation of breathing. Let the body
regulate the breath. We watch and let it relax, giving attention to what
is happening now.

Good posture and the calm, easy breathing is a quiet and pleasant
feeling, like sitting down on a park bench after a walk. There nothing
special to do, and this simple, quiet sitting is a joy in itself.

Breathing in, we first feel the air at the nostril. The cold air moves
down in the windpipe. Practice breathing with the diaphragm muscles at
the abdomen, and once you find the rhythm of it, let the body continue.
The abdomen moves outward, let this control the movement of the air
rather than controlling the chest. The chest opens to allow the air fill
the lungs, but we're not expanding the chest to a great volume. Sitting
still, we can feel the quiet rhythm of heartbeats. Breathing out, the
muscles relax, the warm air rises through the wind pipe, and exits the
body through the nose.

It is not necessary to express this in thought, relax and watch as the
feelings appear in the body. It can take a little while for the body to
settle. The beating of the heart will calm down, and the breathing will
become regular and light.

Allow the body to regulate the breathing on its own. When we approach it
with an opinion, that our breathing should be short or long, then it
becomes rigid and forceful. We want to discover our experiences, not
tell them what they should be.

The body knows how to breathe better than we do. It will breathe with an
even rhythm, if we let it. Rather than trying to figure out whether you
are breathing in the correct way or not, take a step back and turn the
attention around, listening instead of directing. Breathing in,
breathing out, what are you feeling in the body?

It is not one specific feeling which you have to experience. The
intention is rather to have the time and allow the space to be with your
experience.

Centred within itself, knowing the simplicity of the present moment. The
feeling that we have to complete, or fix something, is always an extra,
something which we create. We create this expectation that we have to
change, we have to fix, we have to control. Notice that compulsion and
recognize that you can let it go, you don't have to do that.

There may be a lot of tangled thinking in the mind. Determine what to
think, instead of letting the mind run in circles. For example, use the
mantra BUD-DHO, which means `the one who knows'. On the in-breath, think
BUD-, on the out-breath, -DHO. If the thinking doesn't slow down on its
own, this puts down a guard rail and speed bumps on the road, so that we
stay on track and slow down.

Breathing in, staying with the simple experience of the moment, this is
enough.

We feel compulsions, desires and anxieties, we feel `I need this', `I am
like this', `I should be like that'. They are something we can observe,
we don't need to get involved in the story. Staying with the breathing,
we can turn attention to the experience that is happening.

Awareness of the body is a solid base, calming and reorganizing what is
valuable. If your experience is peaceful, happy and content, stay with
that. There is nothing wrong with that. This happiness is not connected
to craving, not dependent on having to get or reach something. It arises
from seclusion of the senses, returning to simplicity, knowing and
staying with the present. The mind is alert, content, and satisfied.

Meditation can bring up turbulent emotions, and that is good. We are
seeing what we haven't allowed ourselves to see. Looking for answers or
solutions is not necessary at this point. We investigate the emotions on
a more fundamental level, as states of the mind and heart. On that level
there are no personal stories, the feeling or mind state doesn't
announce it's name and what it thinks of us, we are ones who make those
claims about it.

Generosity relaxes the mind, and morality establishes stability. We may
think of good actions, what we have given and received. We may recollect
people we look up to as good examples with respect.

If you find yourself in a tense, strict and cynical mood, try shifting
your posture to relax. We can get so serious about sitting on a cushion,
we become a living joke\ldots{} Rub your ears or massage the face
muscles using your fingers, this invigorates blood flow. Recollect
generosity. In the monastery, often the lay friends are coming to cook
and offer the midday meal for the community. They can be busy while in
the kitchen, but when finished, they are at ease, relaxed and smiling.

The mind can be anxious for results, and recollecting our good actions,
even simple and small ones, relaxes that tension. Imagine what would
happen, if someone magically gave you a hundred-times-fold of what you
need, such as winning both a spiritual and material lottery. How are you
going to meditate then? Probably much like now, but more relaxed. Grant
yourself that rich, wide space.

Generosity lets us recognize that we already have space, and don't have
to push to get ahead of others. Goodness is present in the world and we
can drop the big hurry. It feels joyful to recollect the generosity of
our family, relatives and friends, but even seeing a stranger help
another stranger brings us to smile.

`How can I do it?' Approach it differently, and ask instead, `Can I pay
attention to it?'

The sensation of breathing stops us. We are back at the beginning, when
we didn't know what is going to happen. We are at an empty and spacious
place this way, where we are by ourselves and we have time to stop
there.

The senses turn inward when watching the breathing. The eye sees
colours, but the attention of seeing turns inward, and not seeking
colours and forms outside. The ear hears sounds, but the hearing turns
inward and is not seeking. The body feels hot and cold, the surface of
clothes and the rigid weight of the bones. We watch this while breathing
and let the body calm down, let the mind turn inward and grow still.

Sense-restraint collects our energy and doesn't let it flow away in
every direction. Consider a lake which doesn't have inlets or outlets,
contained all around by the valley. Its single water-source is a fresh,
cool spring in the ground. When it rains, some water will flow into the
lake through small channels, but there being no outflow, it will all
settle in the lake contained by the valley. The water in the lake
remains still, and the cool water from the spring will spread and
permeate the entire lake.

Feelings and the mind are dependent on the body, we can't add to it or
take away from it. Experience is complete in every breath, it starts
with the body and is going to end with it. This world, made of feelings,
is complete in this -- it contains everything we are and everything we
can ever become.

When we suffer, we know that there is something we don't understand. We
don't understand how one thing is created by another, how one thing is
under our control, and another is not.

When we don't see, we repeat the same pattern like following a program,
and create the same suffering again and again. We complain, `why does it
always happen this way?' We keep doing the same thing, and not see it.

Looking closer, we see that one thing depends on another. Then we can
see the option, that we are free to stop doing it. We return to a quiet
contentment this way.

When we have been sitting in meditation for a while, we often start to
complicate it. Where does this restlessness come from? Can't we stay
with something simple? Notice how belief in the simple experience
changes. We start thinking about some point or question, and the doubt
and self-criticizing stops everything.

Isn't it comical? We can be so committed to our self-criticism, as
though it was a transcendental experience to cause ourselves pain. But
we feel we should be struggling with \emph{something}, we should crush
our ego and let go of everything. Perhaps this is the only way we know,
we never thought we could be different.

At the beginning we have a kind and flexible attitude to ourselves, but
there is only hardness and judgement at the end. The young tree is
pliant and fresh, it bends as it grows, but the old tree is hard and dry
when it dies.

Return to the beginning, when you had kindness and patience toward the
beginner. You didn't yet expect yourself to know what to do, and relied
on listening to see what happens. We don't know what is here until we
look and see. That seeing and watching is the fresh knowing. Allow
yourself to always be at the beginning.
