\hypertarget{breathing-1}{%
\chapter{Breathing}\label{breathing-1}}

I used to think that we are supposed to learn new things by observing
the breathing. I sat down and struggled to understand how to do it, and
kept thinking and changing how to breathe, expecting that eventually it
will somehow work, and some day, doing it in the right way I will start
learning new facts about the mind. It was rather painful and mostly
fruitless.

The best is when we have to learn something we didn't volunteer to
learn. We don't learn new facts from meditation, because there are so
few that we need, and we already know them. But we don't stop to stay
with them, and unpack their significance for \emph{what to do}, \emph{in
what manner to do it}, or in other words \emph{how to be}. A list of
facts, if not integrated, doesn't reach deep enough to deal with root
causes in the heart and mind, and have no effect on us. Watching the
breath stops us and opens the attention which can do that.

The Buddha only has a simple message for us: wake up, stop holding on,
you don't have to suffer. We keep unpacking that.

'How can I do it?' That's not the task for you. Rather ask, 'Can I pay
attention to it?'

You are only going to know it after the initial faith has allowed you to
practice, and you are going to be able to express what is already behind
you. That which is ahead, you can't describe it with knowledge, thinking
is not sufficient for that.

The sensation of breathing stops us. We are back at the beginning, when
we don't know what is going to happen. We are at an empty and spacious
place this way, where we are by ourselves and we have time to stop here.

The water pouring from the bottle stops when we put it down straight,
and the water in the bottle settles at the bottom because of its own
weight. We can see movement through it, but itself is not moving.

The senses turn inward when watching the breathing. The eye sees colors,
but the seeing is directed inward, it is not seeking color and forms
outside. The ear hears sounds, but the hearing turns inward and is not
seeking. The body feels hot and cold, the surface of clothes and the
rigid weight of the bones, we watch this while breathing and let the
body calm down, let the mind turn inward and grow still.

Feelings and the mind are dependent on the body, we can't add to it or
take away from it. It is complete in every breath, it start with the
body and is going to end with it. This world, made of feelings, is
complete in this, everything we are and everything we can ever become,
is contained in this.

When we suffer, we know that there is something we don't understand. We
don't understand how one thing was created by another, how one thing is
under our control, and another is not.

When we don't see, we repeat the same pattern like a program, and create
the same suffering again and again. We complain, 'why does is always
happen this way?' We keep doing the same thing, and not see it.

Looking closer, we see that one thing depends on another. Then we are
free to stop doing it. We return to a quiet contentment that way.

Take the time to adjust the body and find a good sitting or standing
posture. The important point is to be upright, in a stable but relaxed
posture, with the head balanced and not lulling forward, and to allow
the breathing to be open.

Determine that you are putting down everyday activities for this period.
'This is not the right time, I will come back to that at its proper
time.' Putting them down not because they are not important, but because
if we don't rest, we can't do our work well. It can help to think the
thought deliberately. This establishes a clear intention, and
communicates to the unconscious processes in the mind, like clearing
away clutter from a desk before starting work.

Take a deep breath to test how the body moves. If you don't feel
anything which is obstructing and limiting, if the breath is easy and
open, then that will do. You don't have to sit in a special way.

Pay attention to the sensation of breathing. Let the body regulate the
breath, we only watch and notice. Giving attention to what is happening.

Good posture and the happiness of breahing easily is a quiet and
pleasant experience, like sitting down on a park bench after a walk.
There nothing special to do, and the simple sitting is a joy in itself.

Breathing in, the breath touches the nostril. The cold air is coming
down into the lungs. The chest expands as the muscles pull in the fresh
air. Breathing out, the muscles relax, the bones of the ribcace contract
and the warm air rises through the wind pipe, and exits the body through
the nostril.

It takes a few minutes for the body to settle. The beating of the heart
will calm down, and the breathing will become even and regular.

Allow the body to regulate the breathing on its own. When we have an
opinion, that my breathing should be short, or it should be long, then
it becomes forceful. We want to discover our experience, not tell it
what it should be.

The body knows how to breathe better than we do. It can do breathing for
us very well, if we let it. Rather than trying to figure out if you are
doing it right, step back and turn the attention around, and watch the
feelings of the body move like clockwork. Breathing in, breathing out,
what are the feelings in the body?

There is no specific thing which you have to experience. It is rather to
have the time and allow the space to be with your experience.

Centered within itself, knowing the simplicity of the present moment. If
you feel that you have to complete, or change something, it is always an
extra, something which we create. We create this expectation that we
have to change, we have to fix, we have to control. Recognize that
compulsion and recognize that you can let it go, you don't have to do
that.

If there is a lot of thinking, determine what to think, instead of
letting the mind run in circles. For example, use the mantra BUD-DHO,
which means 'the knowing'. On the in-breath, think BUD-, on the
out-breath, -DHO. Thoughts can have a lot of momentum, and refuse to
quiet down, but this puts down a guard rail and speed bumps on the road,
to slow down and stay on track.

Breathing in, staying with the simple experience of the moment, and this
is enough.

The compulsions and desires which we feel that we need, or the anxieties
we feel we have to fulfil, they are something you can observe. Staying
with the breathing, you can turn attention to feeling that experience
that you have.

If your experience is peaceful, happy and content, stay with that. There
is nothing wrong with that. It is a happiness which is not connected to
craving, not dependent on having to get or gain something. It is a
happiness arising from seclusion of the senses, withdrawing to
simplicity, and knowing the present moment. That is an alert, content,
and satisfied state of mind.

If you find youself in a tense, strict and cynical mood, I recommend
shift your posture slightly, quietly rub your ears or massage the face
muscles, and recollect generosity. In the monastery, it is frequently
the lay friends who come to cook and offer the midday meal for the
community. They can be busy while in the kitchen, but they are always
relaxed and happy when finished.

Recollecting good actions we have done relaxes the mind which is thirsty
for results. Imagine someone gave you a hundred-times-fold of what you
need. How are you going to meditate then? Probably much like now, just
more relaxed. Grant yourself that richness.

Generosity lets us recognize that we have space, and don't have to push
get ahead of others, there is goodness in the world and we can drop the
big hurry. Recollecting the generosity of friends and relatives is also
joyful, but even seeing a stranger help another stranger brings us to
smile.

When we have been sitting for a while, we start wanting to complicate
it. Where does this come from, that we can't stay with something simple?
Notice how belief in the simple changes, there is some point we start
thinking about, the doubt and self-criticizing stops everything.

It is comical, how we can be so committed to our self-criticizm, as if
it was a transcendental experience to cause ourselves pain. But we feel
we should be struggling, we should crush our ego and let go of
everything.

Why is it OK to think hostile thoughts to ourselves, but not OK to be
supportive to ourselves and trust what is simple? The Buddha's message
is simple, but goes a long way. It takes a long time to unpack, and we
benefit even from a small bit.

There is kindness to ourselves and flexibility at the beginning, but
there is only hardness and judgement at the end. The young tree is
pliant and fresh, it bends easily as it grows, the old tree is hard and
dry when it dies.

Begin again, where there is kindness to the beginner, where you didn't
expect yourself to know. We don't know what is here until we watch and
see. That seeing and watching is the fresh knowing. Allow yourself to be
always at the beginning.
