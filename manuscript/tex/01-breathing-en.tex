\hypertarget{breathing-1}{%
\chapter{Breathing}\label{breathing-1}}

I remember, I used to think that during meditation we are observing the
breathing because we are supposed to learn new things from it. I sat
down and struggled to understand how to do it, and kept thinking and
changing how to breathe, expecting that eventually it will somehow work,
and some day, doing it in the right way I will start learning new facts
about the mind. It was rather painful and mostly fruitless.

The best is when we have to learn something we didn't volunteer to
learn. We don't learn new facts from meditation, because there are so
few that we need, that we probably already know them.

But we don't stop to stay with them, and unpack their significance for
\emph{what to do}, \emph{in what manner to do it}, or in other words
\emph{how to be}. A list of facts, if not integrated, doesn't reach deep
enough to deal with root causes in the heart and mind, and have no
effect on us. Watching the breath stops us and opens the attention which
can do that.

The Buddha only has a simple message for us: wake up, stop holding on,
you don't have to suffer. We keep unpacking this, opening it wider and
wider.

'How can I do it?' Approach it differently, and ask instead, 'Can I pay
attention to it?'

You are only going to know it after the initial trust has allowed you to
practice, and you are only going to be able to express with knowledge
what is already behind you. That which is ahead, you can't describe it
precisely, thinking is not sufficient for that.

The sensation of breathing stops us. We are back at the beginning, when
we don't know what is going to happen. We are at an empty and spacious
place this way, where we are by ourselves and we have time to stop
there.

The senses turn inward when watching the breathing. The eye sees
colours, but the seeing is directed inward, it is not seeking col or and
forms outside. The ear hears sounds, but the hearing turns inward and is
not seeking. The body feels hot and cold, the surface of clothes and the
rigid weight of the bones. We watch this while breathing and let the
body calm down, let the mind turn inward and grow still.

Consider a lake which doesn't have inlets or outlets, contained all
around by the valley, its only water-source being a fresh, cool spring
in the ground. When it rains, some water will flow into the lake through
small channels, but there being no outflow, it will all settle in the
lake contained by the valley. The water in the lake remains still, and
the cool water from the spring will spread and permeate the entire lake.

Feelings and the mind are dependent on the body, we can't add to it or
take away from it. It is complete in every breath, it start with the
body and is going to end with it. This world, made of feelings, is
complete in this -- everything we are and everything we can ever become,
is contained in this.

When we suffer, we know that there is something we don't understand. We
don't understand how one thing was created by another, how one thing is
under our control, and another is not.

When we don't see, we repeat the same pattern like following a program,
and create the same suffering again and again. We complain, 'why does is
always happen this way?' We keep doing the same thing, and not see it.

Looking closer, we see that one thing depends on another. Then we can
see the option, that we are free to stop doing it. We return to a quiet
contentment that way.

Take a few minutes to adjust how you sit and find a balanced posture.
The important point is to be upright, in a stable but not tense posture,
with the head balanced and not lulling forward, and that your posture
should allow open, easy breathing.

Determine that you are putting down everyday activities for this period.
If such a thought interrupts you, respond, 'This is not the right time,
I will come back to that when the time is appropriate. At this time, I
am watching the body the breathing.' It's a bit longer than 'Go away!',
but more friendly to ourselves. It helps to think the thought
deliberately. This establishes a clear intention in the mind, like
clearing a desk before starting work. We are putting them down not
because they are not important, but because we care about doing our
tasks well. If we don't rest, we can't do our work well.

Take a deep breath and watch if you feel tension, something obstructing
or limiting the breath. If it moves easy and open, then your posture is
suitable. You don't have to sit in a special way.

Pay attention to the physical sensation of breathing. Let the body
regulate the breath, we only watch and let it relax, giving attention to
what is happening now.

Good posture and the calm, easy breathing is a quiet and pleasant
feeling, like sitting down on a park bench after a walk. There nothing
special to do, and this simple, quiet sitting is a joy in itself.

Breathing in, we first feel the air at the nostril. The cold air moves
down in the windpipe, fill the lungs, and the chest expands. The abdomen
muscles and diaphragm move inward and outward, controlling the movement
of the air. The quiet rhythm of heartbeats can be felt. Breathing out,
the muscles relax, the chest contracts and the rib bones move closer.
The warm air rises through the wind pipe, and exits the body through the
nose.

It is not necessary to express this in thought, relax and watch as the
feelings appear in the body. It takes a few minutes for the body to
settle. The beating of the heart will calm down, and the breathing will
become regular and light.

Allow the body to regulate the breathing on its own. When we approach it
with an opinion, that our breathing should be short or long, then it
becomes rigid and forceful. We want to discover our experiences, not
tell them what they should be.

The body knows how to breathe better than we do. It can do breathing for
us very well, if we let it. Rather than trying to figure out whether you
are breathing correctly or not, take a step back and turn the attention
around, listening instead of directing. Breathing in, breathing out,
what are you feeling in the body?

There is no specific thing which you have to experience. The intention
is rather to have the time and allow the space to be with your
experience.

Centred within itself, knowing the simplicity of the present moment. If
you feel that you have to complete, or fix something, it is always an
extra, something which we create. We create this expectation that we
have to change, we have to fix, we have to control. Notice that
compulsion and recognize that you can let it go, you don't have to do
that.

If there is a lot of tangled thinking, determine what to think, instead
of letting the mind run in circles. For example, use the mantra BUD-DHO,
which means 'the one who knows'. On the in-breath, think BUD-, on the
out-breath, -DHO. If we have already built up a strong momentum in the
thinking, and it refuses to quiet down, this puts down a guard rail and
speed bumps on the road, so that we stay on track and slow down.

Breathing in, staying with the simple experience of the moment, and this
is enough.

We feel compulsions, desires and anxieties, we feel 'I need this', 'I am
like this', 'I should be like that' -- they are something we can
observe. Staying with the breathing, we can turn attention to the
experience that is happening.

Awareness of the body is a solid base, calming and reorganizing what is
valuable. If your experience is peaceful, happy and content, stay with
that. There is nothing wrong in that. It is a happiness which is not
connected to craving, not dependent on having to get or reach something.
It is a happiness arising from seclusion of the senses, returning to
simplicity, knowing and staying with the present. The mind is alert,
content, and satisfied.

If you find yourself in a tense, strict and cynical mood, I recommend
shift your posture slightly to relax, quietly rub your ears or massage
the face muscles using your fingers, and recollect generosity. In the
monastery, it is frequently the lay friends who come to cook and offer
the midday meal for the community. They can be busy while in the
kitchen, but when finished, they are at ease, relaxed and smiling.

Recollecting our good actions, even simple and small things, relaxes the
mind which is thirsty for results. Imagine what would happen, if someone
gave you a hundred-times-fold of what you need. How are you going to
meditate then? Probably much like now, just more relaxed. Grant yourself
that rich, wide space.

Generosity lets us recognize that we have space, and don't have to push
get ahead of others, there is goodness in the world and we can drop the
big hurry. It also feels joyful to recollect the generosity of our
family, relatives and friends, but even seeing a stranger help another
stranger brings us to smile.

When we have been sitting in meditation for a while, we often start to
complicate it. Where does this come from, that we can't stay with
something simple? Notice how belief in the simple changes, we start
thinking about some point, and the doubt and self-criticizing stops
everything.

It is comical, how we can be so committed to our self-criticism, as if
it was a transcendental experience to cause ourselves pain. But we feel
we should be struggling with \emph{something}, we should crush our ego
and let go of everything. Perhaps this is the only way we know, we don't
even know what it could be like to not be like this.

At the beginning we have the kind and flexible attitude to ourselves,
but there is only hardness and judgement at the end. The young tree is
pliant and fresh, it bends easily as it grows, but the old tree is hard
and dry when it dies.

Return to the beginning, where there is kindness to the beginner, where
you didn't yet expect yourself to know. We don't know what is here until
we look and see. That seeing and watching is the fresh knowing. Allow
yourself to be always at the beginning.
