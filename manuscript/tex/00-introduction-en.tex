\hypertarget{introduction}{%
\chapter{Introduction -- A Proper Purpose}\label{introduction}}

A sense of disturbing suffering, painful loss or sense of confusion
where nothing makes sense -- these feelings stop us in everything we
were doing, and start the circles of thinking about what to do and why
is it this way. A sense of meaninglessness and being locked into a
situation becomes a heavy burden with an overriding sense to free
ourselves from it. It is not clear how, but we start asking questions
and investigating the teachings. We don't know how to do it -- but the
aspiration to be well is there.

Human beings can overcome confusion and flourish in their lives.

The view and attitude we cultivate is that suffering doesn't come from
being alive, but from a distorted understanding of it. We crave and want
to hold onto experiences which are always going to change, and that
sounds like suffering, doesn't it? It sounds meaningless, limited and
locking us in.

Cultivating right view allows for space and freedom around the
limitations of life, and we don't need so much for living one day. When
you feel that you have what you need to live this day well, the
expression of living with that understanding is contentment, gratitude
for life and energy to continue the work of a human being.

We are flooded with good advice about `how to be happy', so what's the
problem? If all it took was good advice, we would be all enlightened
long ago. All the good things we should do, the kind of person we should
be -- one source says we should be tough and fearless, another says we
should have universal compassion. It is a special kind of suffering to
read it all.

Or should we have \emph{Nibbāna}? Is that the right thing? The meaning
of the word is \emph{going cool}, as a fire ceases to burn and goes
cool. A craving to have or attain it would be more fuel for the heat and
burning of becoming. But \emph{Nibbāna} is the coolness of ceasing to
burn, so we should become the non-becoming? The thinking goes,
`\emph{What?}' Here is the genius of the Buddha -- he is teaching that
thinking and becoming won't get us there. If we are creating it, adding
or removing it by thinking, it will be as limited as before. We are not
free by becoming the right thing, but by recognizing that we can give up
the compulsion to continue becoming.

When we are happy in our activities and life is full of meaning, we are
not looking for answers and solutions, because we don't see any need for
them. This peace is often disturbed though, and we start asking
questions.

The best questions come to us in the form of `How should I understand
this well?' Watch what happens when you ask this. Discard the opinions
which present themselves as answers, and keep returning to this open
attitude of knowing the present.

Both joy and sorrow are part of natural processes, but if we don't
understand them, we see one as a reward and the other as punishment,
life never seems to be fair and always out of our control.

Could it be something else the matter? At least we can imagine the
possibility that there is something we can learn. A turning point occurs
when we are not sure about our opinion and we stop to look at the
experience to investigate. Understanding how one thing leads to another
is liberating on a fundamental level, which applies either in times of
joy or sorrow. The experience of such an understanding is freedom in the
mind and contentment in the heart. Practising to live with this freedom
and contentment is a proper purpose of meditation.

Meditation, in general, may be simple relaxation. In the context of the
Buddha's teaching, and when we are going through suffering, meditation
refers to calming the mind and developing insight into the natural
processes of our experience.

It doesn't seem to ask too much, does it? Stopping the hurry for a bit
of clarity, building up some understanding by watching what is going on,
life might start to make sense and we can relax the tension a bit.

We start with assuming that we don't know. Consider how narrow our
attitude is when we start with `\emph{I've seen this, I know this}'.
Perhaps this is true. But I notice that when I try to use knowledge to
solve a problem, my attention revolves around memories, thoughts and
opinions. While I am caught up in the past, the present experience
escapes my attention.

The instruction of the Buddha is to establish a careful intention to
meditate, putting aside the matters of the world for the moment.
Watching experience directly in the body, feelings, mind, and natural
processes as they are.

Contemplating our experience through watching the senses, the
impermanence of experience changes our attitude and perspective of it.
This is about understanding a process, not about gaining knowledge --
there is still no fixed thought or opinion which will become \emph{our}
knowledge -- but our trust is now in the awareness which can be
conscious of, and stay with the changing present. Letting go of our
fixed positions becomes the way forward, seeing with new eyes.

The practice of this understanding feeds back into \emph{what} we do and
\emph{how} we do it. This time, we are not blocked by confusion and
resentment about where we are now, and instead accept our work to be the
best version of ourselves. Life may still not be fair and not entirely
under our control, but now we know a deliberate practice which makes the
difference between a mental breakdown and understanding the facts.

The obstacles of the mind and heart are there because there is something
we have not yet understood with conscious awareness. After the
difficulties clear up, there is space and time to stop for joy and
gratitude.

It feels hopeless sometimes, and the thinking mind seems to have no way
to stop. The fundamental principle is that watching the mind develops
the mind. A wakeful, conscious awareness stops the compulsive
tendencies, which have been unknowingly, unseeingly driving us to the
same painful situations.

The word `Buddha' means `one who knows, one who is awake'. The source of
contentment in activity is that we continue to trust and practice this
wakeful awareness. We don't know what is going to happen tomorrow, but
we can learn to trust our capacity to practise this wakeful awareness in
the present.

Over the years I received much needed help and support to practice from
my teachers and monastic friends. We may not believe in ourselves that
we can be happy, but our teachers believe in us and wish us well. May
all beings overcome confusion and flourish in their lives.

{%
  \raggedleft
  Gambhīro Bhikkhu
  \par%
}

