\chapter{Introduction}

A világ nem adott, a szenvedés nem kötelező Az elme metisztítása és
fejlesztése a hasznunkra van

a szemlélettel kezdjük, önvizsgálattal és meditációval fejlesztjük az
ülés előtt élünk, ülés közben élünk, ülés után élünk. az életben lesz a
gyümölcs, nem az ülés miatt ülünk mint egy béka

a tiszta elme más szemléletet ad, ami más világot ad és más módon
tapasztaljuk az életet

visszatérünk a szemlélethez, ahol a szavak már nincsenek, és kérdésünk
elcsendesült érti már mi a lényeg, és így cselekszik a mozgó világban

purity of the heart and purity of action and speech are not separable,
so we practise sitting and living breathing is not separable from living

the premise of his teaching is that (a) we have a problem to solve and
(b) he can teach a way that works for us, not only him (c) we can learn
and use that.

So if you don't have a problem to solve, if life is somehow not
confusing for you, then this book is not for you.

Otherwise, this is an introduction to become familiar with this.

Suffering is universal, our personal circumstances change It is my hope
this book offers a way of seeing and \emph{something to do} about
whatever is the per suffering. you will know the best thing for you, you
have the most information about your situation, this gives you a way to
investigate it

Tools of thought words which open new experience a way of asking the
right questions what are these? short list, for more, see appendix

Galantai: perspective cannot be paid for, but is the most important for
truth perspective determines what you are going to find in your world
the goals we feel are produced by perspective, right or wrong but the
results, happiness and suffering are according to reality goals from the
wrong perspective will be empty, or bear bitter and painful fruits. so
perspective should be according to reality to lead to happiness and
well-being understanding the natural truths of our minds leads to
happiness and a fulfilling and flourishing life

\begin{itemize}
\item
  restore view; life, the whole of it; a baby and the ash urn are same
  weight
\item
  contemplating impermanence, change and our death we come to love life
  through letting go of self
\item
  the path is to happiness
\item
  there is a relationship b/w virtue, focus and understanding
\item
  which leads towards a good life and happiness
\item
  the buddist position is to start by acknowledging at suffering
\item
  understand it to overcome it
\item
  full understanding is dispassionate, but accepting completely
\item
  expanding our perspective until it encompasses our own death
\item
  don't fabricate goals, don't expect the Buddha to be reponsible for
  your happiness, reliance is external and dependent
\item
  if you want to be happy, be, find it within, the island is within
\item
  then it can be integrated into life without us becoming dependent,
  this understanding is our task
\end{itemize}

Over the years, I have received much needed help and support to continue
practicing from my teachers and monastic friends. We may not believe in
ourselves that we can be happy, but our teachers believe in us and wish
us well. May all beings overcome confusion and flourish in their lives.

Gambhīro Bhikkhu
