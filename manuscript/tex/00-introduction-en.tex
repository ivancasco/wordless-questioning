\chapter{Introduction}

While I was studying at Budapest in 2005, I remember looking for books
which could help me get a perspective on my confused experiences, and
provide advice about what to do with them. There was no lack of
explanation and advice, but I felt missing a concrete direction,
`Interesting ideas, but what do I \emph{do} and how?' I believe that
good instruction should enable one to do more than before, shed light on
the `how' and `what', and even on the `why'.

The first one which gave me a tangible foothold was Ajahn Sumedho's
short book, \emph{The Four Noble Truths}.\footnote{\href{https://forestsangha.org/teachings/books/the-four-noble-truths?language=English}{The
  Four Noble Truths (forestsangha.org)}} It offered a practical method
of investigation, with examples of his own struggles. Later, when I was
staying at Amaravati Buddhist Monastery in England, his other book
\emph{Mindfulness: the Path to the Deathless}\footnote{\href{https://forestsangha.org/teachings/books/mindfulness-the-path-to-the-deathless?language=English}{Mindfulness:
  The Path to the Deathless (forestsangha.org)}} was also illuminating.
I mention them here because certain topics are covered in more detail in
the above two, and if you are reading this book, they could also be
helpful.

Here, I collected what I wish I had read sooner during the years since I
read \emph{The Four Noble Truths}. The right direction can remain
obscured until we learn the words for articulating the question. I find
the language of diagrams and illustrations communicate on another
channel, alongside the prose text. When I start drawing a diagram, I
discover relationships between terms which I didn't think of before.
When I read somebody else's diagram, it shows me how terms are connected
in a larger picture, and I ask how that maps onto my experience.

Illustrations for the meditation instructions were carefully drawn by
Filipe D. Martins. He is a long-time practitioner and so I knew he would
know inside-out what to draw. The Venerable Kovilo carried out a great
amount of editing work on the text, and converted my Hungarian-flavoured
English to native idioms and fluid phrases. I am grateful that they have
dedicated their time and energy, this guide is all the better for it. In
relation to any part which may be disorienting, unclear and confusing,
the responsibility and fault remains with me. Leave it behind and return
to the Buddha who is our incomparable teacher. He taught the timeless
truth, the complete ending of suffering because he had faith in our
ability to recognize~it.

Over the years, I have received much needed help and support to continue
practising from my teachers and monastic friends. We may feel inadequate
and not believe in ourselves that we can be happy, but our teachers
believe in us and wish us well. I offer these words in this spirit. I
wish that these reflections help you to overcome confusion and you may
flourish in your life.

\bigskip

\enlargethispage*{\baselineskip}

{\raggedleft
Gambhīro Bhikkhu\\
2021 November,\\
Sumedhārāma Buddhist Monastery,\\
Portugal
\par}
