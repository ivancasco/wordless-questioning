\hypertarget{elements-1}{%
\chapter{Elements}\label{elements-1}}

We experience the body sitting, and we tend to think 'I am sitting',
'This is something which I am doing'. But the experience, what is it?

It is made of the fundamental areas of experience, such as the pressure,
the feeling of hard pieces of the body connecting to each other,
maintaining the shape of the body. There is a sense of hot and cold. The
heat of the floor, cold on the exposed skin, the cold of the air as is
comes in, sense of warmth when the air goes out. Rigid bones moving, the
ribcage expands and contracts. The solid parts. The heated parts. The
moving parts. The flexibility. These make up the experience of the body.

Observing the material part of our experience this way removes 'us' from
the picture. We are not creating the solid, the heat. The body does
these things on its own. We are not creating the air, the movement or
the flexibility. We experience them but it would be hard to say that we
are responsible for creating them. They appear because they are the
nature of the body, and they disappear when the body changes.

We take credit for it after the fact, and perceive 'me' and 'mine' in
this. But where were 'we' involved in their arising? Can we stop their
ceasing in any way?

These are parts of experience, but when seeing the arising and ceasing,
it doesn't appear to us that this is 'me' or 'mine'. It becomes apparent
that that perception is superimposed, something we stamp onto it when
not paying attention.

This material experience, the solids, the heat, movement, flexibility,
this is within a space. There is direction around us, forward, backward,
up and down. There is space, as another fundamental area of expereience.

And there is a perceiving consciousness, because we are not statues.
Statues would also sit here, like a pile of stones or sand sculptures,
but the statue doesn't have an experience of what it's like to be a
statue. We have this quality which is about what it is like to sit here.
This is different from the material properties of the matter, and
different from space. It is another fundamental of our experience. This
creates the environment, the whole scenery which we call the world, or
life. This is the frame in which it happens.

We see our expereiences to change. There is creation, some of them are
happening to us, there is destruction, some of them are ceasing, some
pressure appears, some pain dissapears. We become conscious of some
experience, some idea pops into our head and disappears. The body
changes, even within half an hour of sitting. The body doesn't feel the
same way at the beginning of the meditation than at the end. Pressure
and tension accumulates, muscles get tired and sore.

In this we create ourselves. Especially when we are hurt, even a small
cut on the finger, immediately the expereience is definetly 'me' and
certainly 'mine'. These preferences initiate the 'me' being born.

The solidity is not concerned with this. The pressure, flexibility, the
space and consciousness, are not concerned with being something, being
'me' or being 'mine'. We know what these things are, but it is something
we put onto the fundamentals. This is the birth and death which we
create. Recognizing it is sufficient.

It is like recognizing that there is a game. We didn't know that there
was a game, but once we understand, we can stop following the rules, the
compulsive tendencies, 'you must do this, you must do that', and there
is a freedom from the game.

We stop playing the game that 'I have to be this, I have to be that'. We
go back to watching the body, experiencing knowing with awareness, and
the sense of freedom that we are not compelled one way or another. There
are things which we will do, there are choices which we will make, but
there is not going to be a hurry or rush, because we understand the
properties of the environment. We understand the properties of the wider
sphere that is around us.

Fear also drops with that knowing. When 'me' and 'mine' is something
constructed, it is no longer the most important thing, and our
priorities reorganize themselves. We can recognize it, but also know we
don't have to be afraid when things will change in a way we didn't want.

And they do change don't they? While we are not looking, something we
really like changes, and now we lost it, it is only a memory in the
past.

This is the game of the ego, the game of 'me' and 'mine'. When we are
not aware of it, we are elevated when winning, and miserable when
losing.

We are not going to build our life on elevated states of emotion, we
have more understanding than that. We go into long-term relationships
and carry responsibilities for our community knowing that it will be
hard work, but satisfying and worthwhile to have done something good.

Nonetheless, 'me' and 'mine' gives us a heartache. We can feel broken
and sick when something dear changes. Usually we don't prepare, and we
think the pain is just part of the package. Tragedy is sad by its
nature, but we also think we should be in pain and suffer. That is the
game we don't have to play, but we see very few examples of such
excellence.

Be conscious, be grateful for having the life that we have. We can give
up the craving, give up the dissatisfaction. This way we know that the
teaching is something we can use.

We are relieved to find ourselves in a wider space than before. Letting
go is the freedom which the heart recognizes and wants to return to. To
greater space where there is space for 'me', space for 'mine', but it is
only a small part of the picture now.
