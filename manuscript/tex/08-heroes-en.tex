\hypertarget{heroes-1}{%
\chapter{Heroes}\label{heroes-1}}

Thoughts don't have a good reputation in meditation handbooks. They are
seen as obstructions, some kind of fault, something annoying and
unwanted like flies. We come away with the impression that good
meditators don't think, and if we are thinking a lot, we can't be good.

Certainly, the distracted mind and obsessive thinking is a painful
expereience, but what about wholesome thoughts? Courage, resolution,
kindness and gratitude all express themselves in spontaneous thought.

Thoughts are embedded in a story, and the story is a description of the
way we see the world. Thoughts have to meet the unknown, become
conscious of it and benefit us with a wider perspective. In this way
they are like heroes.

Who are the heroes whose example you respect? A person who did something
surprisingly good, and you wanted to remember and do likewise? Such as
teachers, friends, historical figures.

Perhaps somebody popular, the author of a book? Do you know what they do
for a living? What is their life story, how do they live? It can be very
different from the show in the book. If their life turns out to be
virtuous and reliable, this will strengthen our trust in their work. It
it turns out to be full of scandals and suffering, we might want to keep
some distance from their ideas.

Our heroes represent our highest values. What are the thoughts which
represent our highest values?

"I am a human being who has the capacity for complete freedom from
greed, hatred and delusion."

"I have the courage and strength to be with my experience, whether that
is easy and happy, or hard and difficult. I have the space and courage
to be with that, and I am grateful for the life I have."

What is that, something unseen, which stops us from thinking that, can
we expand our view and see it? Thoughts meeting the unknown and trying
to make sense of it.

The way the psyche operates, its first response to an unknown, new
phenomena is a sense of alertness to possible danger and fear, to
determine how to respond. This is described as the orienting reflex, or
'What is it?' reflex. It may determine in a split-second the event to be
harmless, but before that there is fear.

In meditation we turn towards the unknown, and an initial sense of
confusion and fear is the appropriate response to that.

It is often the fear, sense of danger and suffering which turns on our
alertness, and we start to meditate to solve the problem. But the
meditation doesn't solve our problems, it teaches us not to fear the
problem. Our perspective widens to include the new and unknown. We
respond with courage and skilful action, which often have to be
improvised on the spot.

We can also find that courage by voluntarily meeting the unknown ahead
of time, when we are in a safe position to approach carefully.

Think of the worst possible outcome of a decision, which can be
reasonable to expect in an uncertain world, and consider it in detail.
Maybe the worst is not that bad, and anything better is only going be
beneficial. There are always sacrifices which have to be made. Early on,
we are lucky to choose what to sacrifice. Later, as life changes, it
will choose the sacrifices without asking us.

Thoughts are determined by what we believe about the world around us.
They way we see the world will determine the kind of thoughts our mind
creates. Or the reverse, we can't think or imagine something that isn't
in our world view. When that view expands, we are struck with new
thoughts which are astonishing to us.

We can discover what our unconscious beliefs are by observing what kind
of thoughts and fantasies play out in the mind. They will tell us the
stories which we believe, we think them because at that moment they made
sense to us. When we reflect on the desire, anger, or some obsessive
thought, it may not seem intelligent any more, but this revision of
values is the benefit of watching them.

The thoughts are embedded in a story about the world, which only loosely
connects to the physical environment. Rather, it describes the world of
possible actions, or how to navigate and find a path in life. We are
going to act out the views and thoughts we believe in, the results can
be miserable if our view was too wacky.

It is comical what kind of thoughts and stories we can be convinced to
be correct, so save a smile or two in the side pocket for the time when
we look back on the present. At the time we feel we must be right. If
our views can change like that, how about the current one? Remembering
that this is not sure either stops us from digging ourselves into a
hole.

Our thoughts and fatasies are sometimes horrible. They can be violent,
full of anger or jealousy. They are the product of the psyche, the
mechanics of the mind, they can really muddled and tangled up. This is
the underworld of the thinking mind -- chaotic, dark and destructive.

what has just flickered through your mind? as if thoughts were the
measure. horrible thoughts sometimes. you're not that.

We can step back and observe them. Keep other people safe from them, we
have the responsibility to keep them under check with self-restraint,
and with patience they will pass.

We are not these thoughts, the hero is the conscious awareness which
recognizes that they have no possible benefit. The mind will change and
there will be better thoughts, we can wait for them.

There is a hero's journey which describes how the story of the self
develops. In the Buddha's teaching we can see how this reflects the
effort of abandoning the unwholesome and developing the wholesome.

The Buddha teaches us about a truth which is greater than the stories of
the self. There is a story which is not about how the self develops, but
which lets go of the self.

When the Buddha taught groups of people, as we know from suttas, the
recorded texts, at the end there is often a summary of how did those
people receive the teaching. Were they delighted or upset, and how many
of them understood it. And there would be entire groups of people, who,
after the Buddha taught, would understand the truth at the same
occasion.

In one discourse, they can't have done a lot of studying. They can't
have all been in the same kind of emotional state, or they can't have
had the same kind of way of thinking. If you have fifty people in a
room, they are all different, and some of them will be really uncommon
types.

Understanding of the truth is not personality development, it is seeing
through the personality as a conditioned process arising and ceasing,
and not being blocked or compelled by it. The truth is not in what we
create. If we create something, that might be beautiful and interesting,
but it is going to end. The personality is not what we trust.

When this idea comes up in the mind, that `This is beyond me. I can't do
this. This is hopeless.' Then you can remember that this is not where
our refuge is. The Buddha is the awakening, the Dhamma is the truth, the
Sangha is the virtuous community. Our refuge is in the awakening, which
recognizes the truth and practices virtue in the world. This is what we
trust.

Always return to what is present experience. It is never complicated.
Present experience is always through the senses. Our world is a world of
the senses. Anything which you experience is through the body and its
mental impressions.

There is touch through the body, there is vision, hearing, smelling,
tasting, and the mental experiences. There is a physical and a mental
description of everything that we experience. That is all that the world
is.

We create stories throught the perception of time. We tell ourselves a
story about something or somebody who I am, who comes from yesterday,
but when we look at present expereience, the story breaks up and stops.

Watching experience in the body, it doesn't have a story. The body
doesn't tell you 'I am this, I am that.' 'I am going to be this, I am
going to be that.' The body doesn't tell you that. What it tells you,
every time something hurts, that it is not yours, it belongs to nature.

In the moment, present experience doesn't have a story. Where is your
story in the sense of touch? Or in the seeing, hearing, smelling,
tasting? We can't find it. Or in the mental experience? We can't find
it.

It is a relief not having to be the hero in a story, because then we are
not in a thriller, a drama, a comedy or tragedy.

The body and its senses are just nature. It was born, it grows, it gets
old, and it dies. This is what it knows. We catch ourselves sometimes,
taking it very seriously, and we look comically bitter as though it was
a job given by a film director\ldots{} so pull out those smiles you
saved in the pocket from earlier. Humour helps, it loosens the grip. We
step back and laugh how absurd this situation of being alive is.

When the stories are too complicated, return mindfulness to present
experience. Know what your experience is now. It gives us the
understanding that this, here is changing, we don't have a lot of
control over it, it is not sure, so don't hold on. We're not sure about
the rest of the story, but that's not going to be so important any more.

One time I was out on a wandering, walking on foot in the countryside. I
was planning to walk from the monastery to the property of a friend,
about 300km distance. I was on my own, stopping in the villages to go
alms-round and receive food for the day, and then moving on. The walk
was quite strenuous, and after 10 days I was already quite tired, but
that's all part of it. My tendency in these situations is to just brush
it off, telling myself to tough it out, don't complain, keep moving, you
can do it.

On day 11, I received alms-food from a man and three ladies, they were
very kind. I continued walking, and in the late afternoon I was walking
through a eucalyptus plantation, it was a dirt road with a lot of cut
branches lying around. At one step, a branch got caught in my sandals in
just the wrong way and peeled off some skin from the ball of the foot. I
bandaged it and the bleeding stopped, it was a minor injury, but right
on the ball of the foot, and I couldn't stand on it. There, walking was
over.

Fortunately I wasn't so hard-headed to not have a phone with me, and I
texted the monastery with what happened, where I was, and if they could
come and pick me up the next day sometime. I wasn't in a hurry any
more\ldots{}

In a few hours, friends who were staying at the monastery arrived, I was
glad to see them! Then I was thinking, isn't this better, this way the
moral is not about accomplishing a feat, but about being blessed with
good friends. The reverse would be sad in fact.

When the story is no longer about us and our achievements, what is left
is gratitude and kindness. Recollecting good actions from the past
brings back the faith in our own capacity for virtue, and when we look
around we find that we are not alone. Putting energy into cultivating
these face-to-face relationships is a deep source of happiness.
