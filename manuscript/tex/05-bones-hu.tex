\hypertarget{csontok-1}{%
\chapter{Csontok}\label{csontok-1}}

A meditációt álló testtartásban is gyakorolhatjuk. Az ülő helyzet után
engedi ellazítani a térd ízületeit és a test izmait. Több figyelmet
igényel az egyensúly megtartására, így a test tartóelemei jobban
érezhetőek, mint például a csontok szilárdsága.

A csontok kifejezetten érdekesek, mert a test központi, merev részeit
képezik, amik meghatározzák az alakját, mit képes és képtelen tenni.
Természetesnek vesszük a tükörképünket és egy viszonylag állandó képünk
van magunkról, de ha a csontok mások lennének, a testnek más alakja
lenne, és ez megváltoztatná nem csak a külsőnket, de azt is, hogyan
élünk.

Álló helyzetben, fordíts egy kis idő arra, hogy megtaláld a
kiegyensúlyozott tartást. Döntsd a tested jobbra-balra kissé, és érezd
ki a súlypontot. Fejelszd azt az érzést, hogy te tartod a tested
egyenesben, hogy ne dőljön el. Lazíts a térdeken és hagyd kissé
behajlani. Ne zárd az ízületeket egyenesre, az megfeszíti őket és a
tartásod merevvé válik.

Engedj magadnak rugalmasságot, kis változtatásokat a tartáson, ahogy az
izmok hozzászoknak a helyzethez. Érezd ki az állásban az egyensúlyt és
figyeld a testet ahogy így tartod. A gravitáció lefelé húzza, nyomást
hoz létre a földön. Hajtsd be a kezeket a has előtt, egyik tenyér a
másikon, kényelmes helyzetben.

Vegyél egy mély lélegzetet, és figyeld, hogyan változik a testtartás
légzés közben. A mellkasi bordák kifelé nyílnak, a váll megemelkedik, a
súlypont kicsit megváltozik. Van valami, ami korlátozza a légzést?
Ellenőrizd, hogy egyenesen állj, és a váll ne görnyedjen előre, ami
gátolja a nyitott légzést. Figyelj a fej egyensúlyára, találd meg a
pozíciót, amiben a fej a saját súlyával egyensúlyban ül a gerinc
tetején, nem dől előre vagy húzódik hátra. Jobb, ha a tekintet nem néz
egyenesen előre, hanem inkább kissé magunk elé, mert a jövés-menés és
mozgás elvonja a figyelmünket. Húzd be az állat könnyedén, ne ugorjon ki
előre, a tekintet magunk elé irányul, pár méterrel előre a földre, a
fejtető kicsit megemelkedik, mintha tartaná az eget.

A fej pozíciója nagy mértékben befolyásolja a felsőtest tartását. Mivel
sokat ülünk székeken, kialakul a szokásunk, hogy a fejet előre tolva
tartjuk. Ez megfeszíti az hátizmokat a gerinc mentén, de ezeket az
izmokat nem tudjuk tudatosan irányítani. Próbáld ki, milyen érzés kissé
visszahúzni a fejet. A hátizmok ellazulnak.

A kiegyensúlyozott testtartás úgy áll, mint az óvatosan egymásra
helyezett kövek. A csontok egymáson ülnek, mint a kövek. A gravitáció
elég, hogy megtartsa, erő nélkül is egyensúlyban van, és könnyű,
kellemes érzést ad.

---

Watching the body\ldots{}

A testi tudatosság megállítja a haragot, neheztelést és belső
vádaskodást. Gondolkodva csak jobban beássuk magunkat, de ezt ilyenkor
nem látjuk. Valahogy fontosnak érezzük, hogy mérgesek legyünk, pedig
csak az idő telik el vele fájdalmasan. Arra vágyunk, hogy vége legyen és
nyugodtan mehessenek a dolgok tovább.

Megfigyeljük a test részeit, és látjuk, hogy nem hordoznak magukkal
semmilyen sztorit. Így fellélegezhetünk, hogy nem vagyunk a sztorikhoz
láncolva, azokat mi hozzuk létre.

Ehhez a figyelemhez mindig vissza tudunk térni, egy belégzés és kilégzés
elég ahhoz, hogy emlékezzünk a keletekzésre és elmúlására, és a
problémáink olyanná válnak, mint a sztorik egy régi újságban. Ráununk
kibogozni a szálakat, mintha valaki más álmait kellene értelmeznünk. Ami
a valóság, az mindig itt van a jelen tapasztalatunkban. Nem az válik
fontossá, hogy mi a sztori, hanem az, hogy a figyelmünket annak tudjuk
szentelni, ahol most vagyunk.

A testi tudatosság enged a kívánságokból és rávezet arra, hogy
szerencsések vagyunk, hogy itt lehetünk.

Hova akarunk jutni? Elkezdhetjük most. Ha valóban érdemes dologról van
szó, szinte biztos, hogy nehéz is. Ha nehéz, szinte biztos, hogy nem
tudjuk mit kell tenni. A bizonytalanság a terv része kell legyen.

Elkezdeni viszont valószínűleg nem bonyolult. Megkérdezhetjük magunkat,
hogy ha lenne időgépünk, visszamennénk pár évvel, hogy már akkor
elkezdjük? Ha már pár évvel ezelőtt elkezdtük volna, örülnénk annak,
hogy mostanra már legalább van némi ismeretünk a helyzetről. A mostban a
jövőbeli önmagunknak tehetjük meg ezt a szívességet. Elkezdhetjük most,
és elképzelhetjük, hogy pár év múlva visszanézünk, és megköszönjük
magunknak, hogy elkezdtük eloszlatni a ködöt.

Amikor nincs tisztán elhatározott szándékunk, csak úgy sodródunk, és
\emph{nem kifejezetten zavar} minket, hogy itt vagyunk, de az elme
szürke és élettelen, szinte próbál elbújni és láthatatlanná válni. Az
eredmény, hogy valóban szürkévé és láthatatlanná válunk így. Semmi rossz
nem történik, de nincs semmi világosság sem abban, hogy ott vagyunk.

Visszanézve a jelenre a jövőbeli önmagunk szemével, visszajönnénk, mert
itt \emph{akarunk} lenni? Lehetünk cinikusak és gondolhatunk a
legrosszabbra, de meglepő módon, a válasz gyakran nem a helyzetet kezdi
el boncolgatni, hanem mint amikor új helyre utazunk, hálásak vagyunk,
hogy ilyen szerencsések vagyunk, hogy itt lehetünk ahol vagyunk. Van
amit még később meg kell tenni, de már azért is köszönetet tudunk
mondani, amit eddig megtapasztalhattunk.

Nem állunk meg elég gyakran, hogy észrevegyük mikor boldogok és
nyugodtak vagyunk. Amikor az elme tiszta és csendes, természetes módon
hálás azért ami itt van, és képes köszönetet mondani az áldásokért amit
életünkben kaptunk. A jelen jó, és akárhogy is alakul az életünk
hátralevő részében, meg tudjuk azt köszönni.

Nem létrehozunk valamit, hanem tiszta szándékkal felismerjük azt ami itt
van. Nem erő vagy képesség kérdése, ezek időhöz és körülményhez
kötöttek. Az elhatározás, a befelé irányuló flismerő figyelem nem egy
adott körülményhez kötött. Az eredménye a helyes szemlélet, amiben
látjuk a dolgok megfelelő helyét, és mit kell azokkal tenni -- vagy csak
megállni, figyelni és lélegezni.
