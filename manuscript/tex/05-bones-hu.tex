\chapter{Csontok}

Még ha jó testtartással is ülünk, előbb vagy utóbb valami valahol fájni
fog. Nem azért meditálunk, hogy kínozzuk magunkat, átválthatunk álló
helyzetbe és hagyjuk az ízületeket és izmokat ellazulni. Az álló helyet
több figyelmet igényel az egyensúly megtartására, így a csontok, a test
tartóelemei jobban érezhetőek.

A csontok a test központi, merev részeit képezik, amik meghatározzák az
alakját, mit képes és képtelen tenni. A tükörbe nézünk és azt gondoljuk,
`az én vagyok'. Mennyire vizsgáltuk meg a képet, mielőtt magunkat láttuk
benne? Egy rövid körvonal, kontraszt és szín máris kiváltja az `én' kép
megjelenését. Viszonylag állandó képünk van magunkról, a jelenben a több
évvel ezelőtti önmagunk képével azonosulunk, és mivel a külső
megjelenésünk változása évek alatt történik, egyre kevesebb és kevesebb
kulcs jellemzőre hagyatkozunk, hogy felismerjük a tükörben látható
képet.

Ha közelebb hajolunk és több jellemzőt látunk, vagy magunkat egy
szokatlan szögből mutatja egy fotó, egy percbe is beletelhet mire el
tudjuk dönteni, kit látunk. Mi határozza meg ez a képet? Ha a csontok
egy kissé mások lennének, a testnek más alakja lenne, és ez
megváltoztatná nem csak a külsőnket, de azt is, hogyan élünk.

Állj egyenes, de rugalmas helyzetben, fordíts egy kis idő arra, hogy
megtaláld a kiegyensúlyozott tartást. Döntsd a tested jobbra-balra
kissé, és érezd ki a súlypontot. Fejleszd azt az érzést, hogy te tartod
a tested egyenesben, hogy ne dőljön el. Lazíts a térdeken és hagyd kissé
behajlani. Ne zárd az ízületeket egyenesre, az megfeszíti őket és a
tartásod merevvé válik.

Engedj magadnak rugalmasságot, végezz kis változtatásokat a tartáson,
ahogy az izmok hozzászoknak a helyzethez. Érezd ki az állásban az
egyensúlyt és figyeld a testet ahogy így tartod. A gravitáció lefelé
húzza, nyomást hoz létre a földön. Hajtsd be a kezeket a has előtt,
egyik tenyér a másikon, kényelmes helyzetben.

A szemek lehetnek nyitva vagy csukva, ha álmos vagy jobb, ha nyitott
szemmel meditálsz. Tartsd a tekinteted a magad előtti pár méteren. Ha
egyenesen előre nézel, a figyelem iránya kifelé fordul, és az ablakokban
látható mozgás elvonja a figyelmed.

Ha becsukod a szemed, de erőltetettnek és fájdalmasnak érzed, figyeld
hova irányul a szem fókusza, amikor becsukod. Ha befelé húzódik, mintha
egy közeli tárgyra fókuszálna a szemhéjak mögött, a szemgolyó belső
oldalán lévő izmok erőlködnek, hogy befelé húzzák azokat, ez a
feszültség szárazságot és könnycsordulást is okozhat.

A hagyományos Buddha szobrokon a Buddha szeme kissé nyitva van -- ő
éber, nem alszik. Ne zárd a szemhéjakat szorosra, gyakorold ellazítani
azokat; engedd le őket nyomás nélkül. Képzeld azt, hogy a szemhéjak
mögött egy távoli dologra nézel, ez hagyja ellazulni az izmokat. Egy
keskeny sáv nyitva marad, némi fényt enged be. Az is segít, ha
megmasszírozod a belső oldali szemizmokat, a nagyujjak hegyével körkörös
mozgásban.

Vegyél egy lélegzetet, és figyeld, hogyan változik a testtartás légzés
közben. A rekeszizom behúzza a levegőt, a has előre mozdul, hogy helyet
adjon. A kulcscsontok megemelkednek, a mellkasi bordák kifelé nyílnak, a
test súlypontja kissé elmozdul.

Van valami, ami korlátozza a légzést? Ellenőrizd, hogy egyenesen állj, a
váll ne görnyedjen előre, ami gátolja a nyitott légzést.

Figyelj a fej egyensúlyára, találd meg a pozíciót, ahol a fej a saját
súlyával egyensúlyban ül a gerinc tetején, nem dől előre vagy húzódik
hátra. Irányítsd a tekintet kissé lefelé, ahelyett, hogy egyenesen előre
néznél, mert a jövés-menés elvonja a figyelmed. Húzd be az állat
könnyedén, ne engedd előre kitolódni, irányítsd a tekintetek pár
méterrel magad elé a földre. A fejtető kicsit megemelkedik, mintha
tartaná az eget.

A fej pozíciója nagy mértékben befolyásolja a felsőtest tartását. Mivel
sokat ülünk székeken, kialakul a szokásunk, hogy a fejet előre tolva
tartjuk, ez megfeszíti az hátizmokat a gerinc mentén. Ezeket az izmokat
nem tudjuk tudatosan irányítani, de kipróbálhatod, milyen érzés kissé
visszahúzni a fejet, érzed ki az egyensúlyt, ahol a hátizmok ellazulnak.

Kiegyensúlyozott testtartásban a csontok úgy ülnek egymáson, mint az
óvatosan egymásra helyezett kövek. A gravitáció elég, hogy stabilan
tartsa, és kellemes, könnyed egyensúlyt ad erőltetés nélkül.

Ha túl analitikus hozzáállással közelítjük meg a test vizsgálatát, ez
groteszk és sokkoló benyomásokat kavarhat fel. A magunk felé irányuló
jószándék tartja a meditációt egyensúlyban, a gyakorlás így valós
tényekre alapozott belátást eredményez, és hozzáállásunk jótékony marad.

Az intellektus működése elvont fogalmakat épít, figyelmének tárgyait
elkülöníti az egésztől, azokat absztrakt és élettelen tárgyként látja.
Nem azért gyakoroljuk a meditációt, hogy a testtel szemben ellenszenvet
vagy elidegenedést keltsünk, a meditáció tiszta megértése akkor működik
jól, amikor a dolgokat a környezetükkel együtt látja, semmi nem létezik
elszigetelt üres térben, a tudatosság befogad, nem kizár, és ehhez a
hozzáállásunk része kell legyen a nyitott elfogadás és jószándék.

Ez a hozzáállásbeli különbség egy történetre emlékeztet Plátóról és
Diogenészről. Plátó éppen egy előadást tartott a diákjainak Athénban az
Akadémián, és az `ember' fogalmát úgy határozta meg, mint `tollatlan
kétlábúak'. Úgy esett, hogy Diogenész hallotta ezt, és mivel kedvelte az
otromba vicceket, behozott az Akadémiára egy koppasztott csirkét, és
feltartotta Plátónak, ``Íme! Hoztam neked egy embert.'' Úgy gondolhatta,
ez bemutatja, hogy bizonyos környezeti jellemzők hiányoznak az ember
intellektuális meghatározásából, mint `tollatlan kétlábú'.

A csoport az Akadémián kiegészítette a meghatározást, ``\ldots{} lapos
és széles körmökkel'', ami kielégítette az intellektuális tárgyalásukat,
de még mindig nem érti mire mutatott rá Diogenész.

Figyeljük a testi érzéseket a belégzés és kilégzés közben. A figyelem
befelé fordul, éberen a benyomásra, hogy `a test ilyen'. Az elme nem
keres, nem a világban jár valahol, nem igényel semmit, ami itt van az
elég. Éberséggel látjuk a testet. A lábfejeket, a lábakat, a hasat, a
mellkast, a karokat, a vállakat, a nyakat és a fejet. Az test egy
egészet alkot, egy mozgó benyomás, érzékeny a lélegzetre.

A testet így figyelni olyan, mint az esőt nézni -- nincs tennivaló,
semmit nem kell eldönteni, az eső magától megy tovább anélkül, hogy be
kellene avatkoznunk.

A Buddha a éberség hatását az elmére az esőhöz hasonlította, ahogy
kimossa a port és megtisztítja a levegőt. Az elmére való éberség
megállítja a kártékony jellemzőket, fejleszti a jótékony jellemzőket, és
megtisztítja az elmét.

Aktív szerepünk van a világban, amit tapasztalunk. Nem tudjuk az
érzéseket kívánságunk szerint megjeleníteni vagy eltüntetni, mert az
érzések már egy késői szakasza a folyamatnak, viszont nem vagyunk
kizárt, külső szemlélők sem.

A múltbeli tettek (elme-, szó-, testbeli) tényezői csatlakoznak a jelen
eseményekhez és a figyelmünk módjához; ez alapján keletkeznek a
kellemes, kellemetlen vagy semleges érzések, és az ezekhez való
hozzáállásunk határozza meg, hogy magukkal sodornak minket, vagy
higgadtak maradunk.

Amikor megalapoztad a tiszta megértést, és észreveszed, hogy az elme
egyre tisztább és stabilabb, vedd szemügyre a folyamatot, mi tette
lehetővé ezt a változást? Mit kellett tenned? Nem kellett az érzéki
benyomásokat manipulálnod, vagy küzdened a gondolatokkal és érzelmekkel,
elegendő volt megváltoztatni a figyelem módját.

A figyelem módja hozza létre a referencia keretet, amiben
megtapasztaljuk az érzékek világát az észlelések felfogásán keresztül,
miközben emlékezettől függően felismerjük és értelmezzük önmagunkat a
jelenben. A figyelem változásán keresztül megváltoztatjuk milyen
keretben látjuk a tapasztalatunkat. Ezzel nem tápláljuk tovább a
kártékony tényezőket, és a közvetlen tapasztalat szemszögéből, annak
megfelelően ahogy a dolgok vannak, végük szakad, a folytatáshoz
szükséges referencia pont nélkül.

Röviden úgy mondjuk, az elmére való éberség megtisztítja az elmét.

A testi tudatosság felé fordulunk, ami kioltja mind a haragot és vágyat,
megváltoztatja a figyelmünk keretét, hasra esnek mintha kihúztuk volna
alóluk a szőnyeget. A zsúfolt, kritikus és haragos gondolatok olyanok,
mint egy zajos műsor, vagy a hírek a tavalyi újságban -- a témája már
nem érdekel minket, elvesztette a fontosságát, folyton ugyan úgy
körbe-körbe jár. Tedd le a gondolkodást, mint egy fáradt túrázó a nehéz
hátizsákot, és maradj a test éber figyelmével.

Időnként elvonja valami a figyelmünket, vagy elkezdünk álmodozni --
mindig térj vissza a légzéshez és az álló helyzet testi érzéseihez. Ha
csak állunk és történetekről, belső fantáziákról gondolkodunk, azzal nem
a belátás meditációt gyakoroljuk, hanem a buszra várakozást.

Vizsgáld az elme állapotát mint tapasztalatot, a test észlelése és az
érzései azelőtt jelennek meg, hogy az `én' észlelését felépítenénk
belőle. Mire emlékszünk önmagunkról? Ha elfelejtjük, amikor tegnap
valaki gorombán ránk kiabált, vagy emlékezünk, amikor barátainkkal
töltöttük az időt, ez megváltoztatja miként tekintünk önmagunkra?

Ez a kapcsolat folyton változik, a felfogott jelen képek folyton
változnak, az éber tudatosság a bizalmat abba helyezi, ami ismeri ezt a
változást. Az, ami ismeri a változást egy nagyobb képen belül látja azt,
és nem ragad meg a félelemben. Önmagunk képét egy aktív folyamat hozza
létre, részt veszünk benne az emlékeinkkel, azok újra-értelmezésével, és
amit a jelenben választunk megtenni.

A kényszeres ön-bírálat és elvárások olyan érzés, mintha mocsárba
lennénk ragadva, egyre csak taposunk, de sehova nem jutunk. Túljuthatunk
ezen a csapdán azzal, hogy a testet részeiben figyeljük, és a változó
észlelésekre összpontosítunk, mielőtt az `én' képe megjelenik.

Figyeld az érzést, ahogy a csontok kapcsolódnak egymáshoz az
ízületeknél. Megjelenik egy belső struktúra érzete, a struktúra ami a
testet belülről tartja. Merev darabok, egyik vég a másikhoz kapcsolódik,
egymásra rakva egyre magasabbra. A lábakban érezzük a merev csontokat,
ahogy tartják a testünket. Az érzés a gravitációról és nyomásról
árulkodik. A csípő csont a lábakon nyugszik, a felső test ehhez
kapcsolva mozog. A mellkas bordái szétnyílnak és összehúzódnak a
légzéssel. A gerinc ív alakban tartja a súlyt. A fej ott ül a gerinc
tetején. A koponya csontjai megfeszítik az arcbőrt.

Darabokból áll. Darabokból, amik egyes helyeken merevek, más helyeken
puhák és rugalmasak, ez adja meg a formáját. Amikor egy személyre
nézünk, nem látunk mást, mint a hajat a fejen, a szőrt a testen,
körmöket, fogakat és bőrt. Ebből építjük a személyt -- elég a másodperc
törtrészéig rápillantanunk, felismerni egy körvonalat vagy jellemzőt, és
azt gondoljuk, `Ez én vagyok, jól nézek ki?'

Egyes helyzetekben láthatjuk a fokozatos felismerést az általánostól az
egyéniig, például amikor valakit látunk a ködben sétálni. Először, egy
`személy' alakja, azután férfi vagy nő, lehet, hogy valaki akit
ismerünk, míg végül egy részlet kiváltja egy barátunk felismerését.
Mindez az észlelés képeinek világában játszódik le.

Szokásunk, hogy a saját testünket és másokét egy töretlen egységként
látjuk, egy dologként. Ebből a nézőpontból kialakul a megrögzött
gondolat, hogy van egy ideális formája és állapota. Elvárjuk, hogy a
testnek legyen bizonyos formája, magas vagy alacsony, és további
jellemzők.

Ezek világi bírálatok, a kultúra amiben felnövünk, és nap mint nap
élünk, alakítja ki ezeket a képeket. A hirdetések és a média üzenetei
megerősítik ezeket az elvárásokat és gondolat nélkül hiszünk bennük, de
amikor közelebbről szemügyre vesszük, azt látjuk, ez egy ferde tükör
képe, nem egyezik a valósággal.

Mi sokat gondolunk arra, hogy mások mit gondolnak rólunk, de \emph{mi
magunk} mennyit törődünk mások küllemével? Ha magamat figyelem, nem
foglalkozok sokat más emberek kinézetével, de én zavarban tudom érezni
magam, és azt képzelem \emph{ők} biztos \emph{rólam} gondolkodnak, még
ha idegenek is. Mikor, valójában, annyit gondolnak rám mint én rájuk --
alig, ha egyáltalán.

Az önbírálatunk nyomása mellett, azt képzeljük mások hogyan bírálnak
minket, és mivel nem tudhatjuk és nem irányíthatjuk mit gondolnak, az
elme belső párbeszédei megpróbálják megteremteni ezt a tudást és
irányítást, ami illúzió marad. Amikor lejátsszuk ezeket a belső
párbeszédeket, élvezzük ezt a megfoghatatlan irányítást, de lemaradunk
arról a szabadságról, ami az irányítás igényének elengedéséből születik.

Megfigyelhetjük hogyan szűnik meg ez a személyes aggodalom, amikor a
test egyes részei elválnak. Sokat foglalkoztathat minket a hajunk
például, de csak addig, amíg a fejünkön van. Amikor a fodrász levágja,
nem törődünk a padlón összegyűlt hajkupaccal. Hasonló módon, mikor a
körmünket vágjuk, mikor van az a pont, amikor már nem `én' és `enyém'?

Így vizsgáljuk a testet, mint ami darabokból áll össze, és látjuk, hogy
a test nem egy bontatlan egység; darabokból és részekből áll, amiknek
megvan a maguk természete, és aszerint viselkednek, nem hallgatnak a mi
vagy mások véleményére. Csontok, bőr, haj, fogak és körmök -- olyanok
amilyenek, a saját természetüknek megfelelően. Ezen szemlélődve nő a
bizalmunk, hogy jó szándékkal elfogadjuk őket.

A testünk egy áldás, nem azért gyakoroljuk a meditációt, hogy
ellenszenvet keltsünk felé. Az egészség egy áldás, támogat minket
mindenben, amit teszünk. A Buddha az egészséget a legnagyobb kincsnek
nevezte.

Figyeljük a légzést, a test részeit, a jelen tapasztalatunkat. Azt
találjuk, hogy nem hordozzák magukkal az `én' és `enyém' történeteit, és
szabad választásunk, hogy követjük őket vagy sem. A jelenségek függő
kapcsolatokon keresztül jönnek létre, a kapcsolat felbomlásával
megszűnnek. Ez minden ami történik.

A testi tudatosság enged a kívánságok igényén és rávezet arra, hogy
szerencsések vagyunk, hogy itt lehetünk. Ehhez a figyelemhez mindig
vissza tudunk térni, egy belégzés és kilégzés elég ahhoz, hogy
emlékezzünk a keletkezésre és elmúlására, és a kétségeink olyanná
válnak, mint a sztorik egy régi újságban. A szálakat nehéz követni és
fáradtságos kibogozni, mintha valaki más álmait kellene értelmeznünk.

Ami valós, az mindig itt van a jelen tapasztalatunkban. Nem az válik
fontossá, hogy mi a történet, hanem az, hogy a figyelmünket a jelennek
tudjuk szentelni, és otthon tudjuk érezni magunkat ott, ahol most
vagyunk.

A tiszta szándéknak fontos szerepe van. Amikor nincs tisztán
elhatározott szándékunk, csak úgy sodródunk, és nem kifejezetten zavar
minket, hogy itt vagyunk, de az elme szürke és élettelen, egy jövőbeli
időre vár, és addig próbál elbújni és láthatatlanná válni. Az eredmény,
hogy valóban szürkévé és láthatatlanná válunk. Semmi rossz nem történik,
de nincs jelenlétünk sem az öröm, sem a bánat elfogadására.

Nem állunk meg elég gyakran, hogy észrevegyük mikor boldogok és
nyugodtak vagyunk. Amikor az elme tiszta és csendes, természetes módon
hálás azért ami itt van, és az áldásokért amit életünkben kaptunk.

A hálát nem lehet akarattal erőltetni, nem létrehozunk valamit, hanem
tiszta szándékkal felismerjük azt ami itt van. Nem erő vagy képesség
kérdése, ezek időhöz és körülményhez kötöttek. Az elhatározás, a befelé
irányuló felismerő figyelem nem egy adott körülményhez kötött. Az
eredménye a helyes szemlélet, amiben látjuk a dolgok megfelelő helyét,
és mit kell azokkal tenni -- vagy csak megállni, figyelni és lélegezni.
