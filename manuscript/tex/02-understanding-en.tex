\hypertarget{understanding-1}{%
\chapter{Understanding}\label{understanding-1}}

We break knowledge into parts and we speak about practice in gradual
steps, but understanding happens at once. Perception is immediate, the
`Aha!' moment, when the fog clears. Some information is necessary to
start with, but we remember that the truth of the teaching `is to be
experienced for oneself'. In meditation practice, the facts which are
useful for us, are \emph{here-and-now} facts, the kind which we can know
in present experience.

If we depend on external information, or keep waiting for another
experience, we never arrive at the place where we can stop. This
dependence is exhausting, it keeps creating more doubt and mental
restlessness.

Memories, perceptions and expectations create a tangible force which
pushes and pulls on us, it never rests, while the experiences themselves
disappear and we want the next one. How can it be so convincing, that it
keeps pulling us forward? \emph{We see ourselves in them.} We see them
as what we are, what we were, what we are going to be, and since they
keep breaking up and changing, we continue on and on to the next one.

But what if we woke up one morning without any memory of our past? We
would still grasp our experience as \emph{me and mine}. It could be so
disturbing that we might feel paralysed with fear until we can grasp
some story that explains where and who we are. We don't even have to
lose our memory to watch this: missing a travel connection can be
frightening enough to feel that way, or any other time when we are not
in control.

The Buddha teaches that this is an ingrained tendency: the mind
continues to construct `me and mine' out of sense-experience until the
process of sense-experience is fully comprehended. Oh, our poor mind,
why can't it be wiser? We might forgive it though -- after all, we
occupy it with thinking about how to get what we want and complaining
about what we didn't get, instead of doing our homework to `fully
comprehend sense experience'.

We tend to expect we should figure this in our head, but it will give us
a headache. A much better place to start is sense-restraint, and more
simple as well. It collects our mental energies, and either calm
stillness or reflection will follow on from there in a natural rhythm.

We can notice that we are able to stop -- it is not more information,
possessions, experiences that we need, but the letting go of that need.
Consider what did you use today. You may own much, but for one day, a
little is enough. We don't need as much as we think, and relinquishing
is a relief from a burden, not a loss. In this restraint we find a
strength and energy, which before, was consumed by distracted desire.
Restraint is always at hand, and not threatened by external factors.

When you are not sure what to do with a distracted mind, occupied with
thinking about how to get one thing or another, remember the quiet
happiness of sense-restraint and relinquishment. A sense of relief
follows them, like putting down a burden.

This is like learning how to pack less in your backpack for a hike.
After some experience, you don't understand why you needed to carry so
much before. Virtue is action that leads to happiness, and
relinquishment is such a virtue. Understanding informs virtue, and the
happiness born from it supports the trust in that understanding.

We watch the breathing, and how the senses operate. The eye sees forms,
and a sensation appears. The ear hears sounds, the body feels solidity,
hot and cold. With contact a feeling appears. If there is contact
between the sense and the sense-object, the sensation is going to
appear, the process doesn't depend on us. When the contact between the
sense and sense-object is broken, the sensation ceases. This was our
experience, arising and ceasing dependent on contact. We may have a
measure of control over our movements, but other than that we can't
control the experience as a process.

When we don't notice this, we think that the feeling will be ours. We
believe we are going to get something from it, and so we cling to it.
This incorrect understanding underlies the need and anxiety we feel.
When we are disappointed in our expectations, our tendency is to assume
we didn't do something right, or this was not the right one. We go and
look for another experience, which will be right one, hoping that it is
not going to behave like this.

Observing sense-contact this way, the feelings are no longer attractive
or repulsive, instead we see that neither is going to be stable and
reliable. The feeling is not numbness -- in meditation we turn toward
experience, not away from it. Rather, there is a sense of sensitive
equanimity. We remain alert and sensitive to the experience, but it no
longer controls and disturbs us, we remain calm.

How does it affect the stories we tell ourselves? A narration is going
on an we are at the microphone. We tell ourselves whether the feeling
was good or bad, and how we should think about it. The central element
of this narration are \emph{our feelings}, and to have more of the good
ones. But what happens when both the good ones and bad ones are
unstable, unreliable, and not in our control to arise and cease? The
internal values are re-ordered, guided by impermanence, rather than
craving.

The painful experiences tend to be the first ones which we notice, and
then we start paying attention. We don't feel a great need to explain
pleasant experiences, do we? The sense of frustration,
unsatisfactoriness and proliferating thinking become a sign we can look
out for.

Take a step back and observe it as a process which has an origin, goes
through change, and then ceases. Where in the body do I feel this
feeling? Can I remember when it started? Can I see how it changes? Can I
catch it as the feeling ceases?

We can't control the world around us, but our attitude opens or closes
the doors to the actions we see as possible. These actions create the
situation where we live our lives and how we see ourselves. Without
reflecting on our experience, good feelings are a reward and bad
feelings are a punishment, and the meaning of our lives will revolve
around them. It will always be `Who am I \ldots{} How do I \ldots{} Why
do I \ldots{} What should I \ldots{}'.

Do you remember swinging a weight around on string? It goes fast, going
around in a circle, but always attached to the self at the centre. Some
questions are a trap, they continue without ever finding way to be free
or a place to stop. The swinging weight does stop, when we get tired of
it and either let go of the attachment, or stop swinging. Is that
something we can do with our attention as well?

In the context of meditation, reflection doesn't mean that all kinds of
thinking are productive for insight. We investigate experience using the
Four Noble Truths as a guide. This begins with an experience that is
personally easy to identify: suffering, stress, unsatisfactoriness, or
\emph{dukkha} in the Pali language -- but the direction of thought is
toward impersonal, natural processes.

The starting position is to recognize that the stress or suffering is
here. As information, this is trivial, that yes, there is stress and
suffering in the world. But when we ourselves experience it, we rather
like to pay attention to something else, or tend to rather blame
somebody else for it. Anything but avoiding to become conscious of it
and deal with it.

The instruction here is that the way forward is to turn toward the
suffering, and investigating it, we seek a way of understanding. This is
the First Noble Truth in the teaching of the Buddha -- there is
suffering, and the noble attitude is to turn toward it and understand
it.

What do we understand? That the suffering didn't come from nothing, it
is the result of earlier causes. Examining our situation this way, we
are not helpless -- we may not understand every little factor, but it is
already a relief that perhaps we are able to change something.

The Second Noble Truth points out that we find the cause of suffering in
ourselves. The wish that experiences were otherwise than their nature
dictates, our tight clinging to what is impermanent, breaks up, and not
possible to keep. The suffering, the \emph{dukkha} we experience depends
on that holding on. The instruction, the noble attitude here is to let
go of this thirsty craving and clinging, because clinging to transitory
experiences is suffering.

With the cessation of the factor which caused it, the result, the
suffering ceases as well. The good news is that the end of suffering is
also found in ourselves.

Here, the mind creates the kind of world we live in -- if we watch it,
we have a chance to at least not make it worse. And who knows, it might
get better?

The Third Noble Truth directs our attention toward this -- there is a
solution, we are not obliged to live in bitterness and meaningless
struggle. The advice, the noble attitude is to practise and experience
this for ourselves, through understanding and letting go of attachment,
allowing the suffering to cease.

Even if we can't let go right away, it is already a relief to see that
the connection is true -- `if I could let go, I wouldn't suffer from
it'. This is already half of the work. Until then we were wandering
without a map, but now there is a way forward.

The Fourth Noble Truth describes the practice of the way. The Buddha
divided it into eight factors, which incorporate the situations of
everyday life and the development of meditation.

The parts of the Eightfold Path are (1) understanding, (2) intention,
(3) speech, (4) action, (5) livelihood, (6) mindfulness and (8)
concentration. When they are aligned with the truth, we call then
\emph{right}: Right Understanding, Right Intention and so on. Breaking
it down to parts helps the investigation, it is easier for us to think
this way, but the factors strengthen and support each other. The
practice is realized as an integrated whole.

When we most need them, we them \emph{fast}, we can't stop to count
factors. The most useful tools are portable and accessible in the given
situation. When we read and ponder the meaning, we have time to turn the
words this way and that, this is a stage of studying. But mindful
attention as an abstract idea doesn't help much -- it is valuable when
practised, when it is at hand in the present moment.
