\hypertarget{csuxf3nak-1}{%
\chapter{Csónak}\label{csuxf3nak-1}}

Meditation techniques are useful, they let us learn by doing, but we
learn the techniques not for knowing the techniques, but for knowing the
mind. Too many techniques, or complicated steps and sequences however
are confusing, and lead to a sense of being lost.

Egyszerűsítsd le a lényegre. Egy lélegzet, egy BUD-DHO. Belégzésre
ismételd magadban a mantra felét, BUD-, középen a megállás, kilégzésre,
-DHO. BUD-DHO. Kész.

A lényeg a béke, és a megértés, ami megállít.

A béke abból ered, hogy az érzékek visszahúzódnak és befelé néznek. A
keresés megáll, mert ami van elég, és sehova nem kell menni. A csendes
öröm abból fakad, hogy az elme megérti, hogy nincs boldogság amit a
világban hajszolni kell. Az értékek átrendeződnek.

Hol van a béke most? Hol van a megértés most? Semmit nem kell megoldani.
BUD-DHO, tíz lélegzet, és vége a világnak. Elég, ha a kérdés megállítja
az elmét. Ez a megállás a figyelő csend, a válasz pedig nem szükséges.

Hétköznapi helyzetben, egyszerűsítsd le, amíg mindig felismerhető.
Teljesen fáradt vagy, semmi energiád, kavarog az agyad a napi
jövés-menéstől, de a légzés akkor is elérhető, a csend ott is érezhető.
