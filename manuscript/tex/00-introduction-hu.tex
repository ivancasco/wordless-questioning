\chapter{Bevezető}

\vspace{-\baselineskip}

Abban az időben, amikor Budapesten tanultam 2005-ben, emlékszem, hogy
kutattam olyan könyvek után, amik segíthetnének használható
perspektívába helyezni zavarodott tapasztalataimat. A magyarázatból és
tanácsból nem volt hiány, de úgy éreztem hiányzott a konkrét irány:
`Érdekes ötletek, de mit \emph{tegyek} és hogyan?' Úgy hiszem, hogy az
útmutatás akkor jó, ha segít abban, hogy többre legyél képes mint
korábban, fényt vet a `mit' és `hogyan'-ra, és talán még a `miért'-re
is.

Az első könyv, ami kapaszkodó pontot nyújtott, Ácsán Szumédhó rövid
könyve volt, \emph{The Four Noble Truths}\footnote{Magyar fordításban:
  \href{https://a-buddha-ujja.hu/books/csittaviveka/hu}{Csittavivéka - A
  csöndes tudat tanítása (a-buddha-ujja.hu)}} (A Négy Nemes Igazság). A
vizsgálódás egy gyakorlati módszerébe nyújtott bevezetőt, a saját
küszködéseinek példáival együtt. Később, amikor Angliában az Amaravati
buddhista kolostorban jártam, a másik könyvét is elolvastam:
\emph{Mindfulness: the Path to the Deathless}\footnote{\href{https://forestsangha.org/teachings/books/mindfulness-the-path-to-the-deathless?language=English}{Mindfulness:
  The Path to the Deathless (forestsangha.org)}} (Éberség: az Út a
Haláltalanhoz), ez is rávilágított sok új dologra.

Azért említem őket itt, mert bizonyos témákat részletesebben tárgyalnak,
és ha ezt a könyvet olvasod, talán ezek is hasznosnak bizonyulnak.

\enlargethispage*{2\baselineskip}

Itt azokat a tanácsokat és tanításokat gyűjtöttem össze, amikről azt
kívánom bárcsak korábban olvastam volna, vagy valaki elmondhatta volna,
az évek során mióta az első könyveket olvastam. A helyes válasz rejtve
marad, amíg meg nem tanuljuk feltenni a helyes kérdést.

Ábrákat és illusztrációkat is mellékeltem, amik a leírások szavai
mellett egy másik csatornán kommunikálnak. Amikor egy ábrát készítek,
olyan kapcsolatokat fedezek fel a kifejezések között, amire korábban nem
gondoltam. Amikor mások ábráit látom, azt mutatják nekem, hogyan
kapcsolódnak a kifejezések egy tágasabb összképet tekintve, és
megkérdezem magamtól, hol található ez a reprezentáció a saját
tapasztalatom térképén.

A meditációs testtartások illusztrációit Filipe D. Martins készítette. Ő
hosszú ideje gyakorol, és tudja mit kell rajzolni, a belső tapasztalatai
alapján. A Tiszt. Thānavaro Bhikkhu figyelmesen ellenőrizte a magyar
szöveget. Hálás vagyok, hogy idejüket és energiájukat ennek a feladatnak
szentelték; ez az útmutató sokat javult munkájuk eredményeképpen. A
könyv bármely része, ami összezavaró és nehezen érthető maradt, a saját
hibámból ered. Ha ezzel találkozol, hagyd magad mögött, és térj vissza a
Buddhához, tanulj a páratlan tanítónktól. Az időtlen igazságot
tanította, a szenvedés teljes megszűnését, mert hitt abban, hogy megvan
a képességünk felismerni azt.

Az évek során sok segítséget és támogatást kaptam első tanítóimtól, akik
a szüleim, illetve szerzetesi tanítóktól és barátoktól. Mi magunk lehet,
hogy alkalmatlannak érezzük magunkat, és nem hisszük, hogy valaha is
boldogok lehetünk, de tanítóink hisznek bennünk és sikert kívánnak
nekünk, hogy túljussunk a zavaron és virágozzon az életünk. Ebben a
szellemben ajánlom fel ezeket a szavakat.

\bigskip

\enlargethispage*{2\baselineskip}

{\raggedleft
Bhikkhu Gambhíró,
2021 November,\\
Sumedhārāma Buddhista Kolostor, Portugália
\par}
