\chapter{Ciklusok}

\keywords{lépések a gyakorlásban, leírások és tudatosság}

\noindent A meditáció a jelenbeli érzéseken, tapasztalatokon keresztüli
megismerést tanítja. Az utasítások lépésről-lépésre írják le a
fejlődést, de a jelenben csak egy pillanat elérhető számunkra. Előre
vagy hátra lépünk? Bármely irányban, a tapasztalat egyszerre egy lépés,
a lépés ahol minden változik. Szavakat használunk, hogy leírjuk a
tapasztalatot, de a tudatosság erre a tapasztalatra szótlan. Az éber
figyelem aktívan befelé néz, mintha önmagát kérdezné, de nem vár
választ. A nyelv szimbólumai is korlátozóak. Rögzített reprezentációkból
állnak, míg a tapasztalat mozgásban van.

A meditáció célja nem az utasítás lépéseinek tökéletesítése. A cél a
jelen tapasztalat tiszta ismerete, ami visszaállítja a helyes
nézőpontot. Kialakulhat bennünk az a benyomás, hogy mindig ugyanazt a
lépés sort kell teljesítenünk, és amikor az elme nem aszerint a sorrend
szerint fejlődik, csalódottak vagyunk.

\keywords{narrátor elme, tapasztalat mint alap, a megfelelő festéket választani}

És ami még rosszabb, úgy látszik mások békésen meditálnak, ők biztos jól
értik! A gyakorlás felszínre hozza az önkétséget. Vizsgálhatjuk ezt a
másik oldalról: Valaki talán dicsér minket, `Olyan békésnek tűntél, te
biztos tudsz valamit!' De mi tudjuk milyen szétszórt gondolatokkal volt
tele a fejünk, láthatjuk milyen megbízhatatlanok az ilyen benyomások.

A narrátor elme gépiesen folyton megjegyzéseket tesz, de nincs mögöttük
mélység vagy vizsgálódás. Érdemes egy lépés távolságból nézni ezt, és
megízlelni milyen megbízhatatlan és bizonytalan a saját gondolkodó
elménk, még ha azt is gondoljuk, `Ebben biztos vagyok!'

Fordítsd meg ezt a hozzáállást és kezdd a tapasztalattal. Egy kérdező,
kíváncsi figyelemmel indulj neki, ami szótlanul érdeklődik a jelenről.
Ha a tapasztalatunkat vesszük az alapnak, ahogy ez a tapasztalat most
van, az milyen megértést ad nekünk?

Először magunkat vesszük szemügyre -- Hogyan érezzük magunkat? Milyen
állapotban vagyunk? -- erre válaszolunk intelligensen, a meditációnk
megfelelő irányba való fejlesztésével. Amikor falat festünk, először a
falat vesszük szemügyre, kiválasztjuk az annak megfelelő festéket, és
\emph{azután} követjük az utasítást a dobozon. A rossz fajta festék le
fog peregni, nem igaz? Van amikor le kell higgadnunk, máskor energiát és
erőfeszítést kell bevetnünk, vagy várni, hogy a belső vihar tovább
álljon.

A meditáció különféle módszereinek lépései az imitáción keresztül való
tanulás része. Egy példát követve figyeljük önmagunkat és meglátjuk
hogyan működik az elménk. Amikor szenvedést érzünk, vagy fel tudjuk
oldani, vagy türelmesen kivárjuk, amíg véget ér. Később tiszta fejjel
visszanézünk, tudjuk mi történt minek a hatására, és a gyakorlásra
irányuló megértésünk nőni fog. Ebből tanultunk valamit, és nem
ragaszkodunk az első, bemutató példa részleteihez.

\keywords{fejlődés ciklusokban}

Ez egyszerű lenne, ha a meditációnk egyenes vonalban fejlődne, a
gyakorlással töltött percek és órák számával egyenes arányban. Azt
tervezzük, hogy le fogunk ülni, az elején kissé szétszórtan, de egy
órával később, \emph{ha jól tudunk meditálni}, nyugalmat fogunk érezni,
az elménk tiszta és összeszedett lesz. Legalábbis erre számítunk.

Később vissza emlékezünk mi történt a meditáció alatt, és azt látjuk,
hogy nem ez szokott történni. A tapasztalatunk nem egyenes vonalban
fejlődik a sekélytől a mélyig, vagy a szétszórttól az összeszedettig.
Azt gondolhatjuk, ez a mi hibánk, mert `nem vagyunk jók' a meditációban,
vagy `nem helyesen' követjük a lépéseket.

Amit megpróbálunk terv szerint, lépésről lépésre követni egy módszert,
minden az elvárásainktól eltérően történik. Arra gondolhatunk, `Talán
nem próbálom elég erősen?' Egyre jobban neki feszülünk, és egyre
fájdalmasabb lesz. Ilyen az az érzés, amikor egy véleményt rá akarunk
erőltetni a tapasztalatra.

Ha felidézzük, hogy a tapasztalatunk hogyan változik idő közben, más
mintát látunk. Egy tapasztalat megjelenik, változáson megy keresztül,
elmúlik, és egy újabb tapasztalat jelenik meg. Az elme ilyen ciklusokban
fejlődik, és ezek a ciklusok nem vesznek tudomást a céljainkról, hogy a
meditációnkat úgy akarjuk fejleszteni mint egy ranglétrát. A kommentáló
elme próbál egy személyes történetet illeszteni a tapasztalatra, arról,
hogy mi valaki olyan vagyunk, aki jó vagy rossz a meditációban.

Ehelyett vegyük a tapasztalatot alapigazságnak, és onnan induljunk.
Milyen fajta tapasztalat ez itt? Észrevehetjük, hogyan mozog a figyelem,
mint a tudatosság egy folyamata, hogyan megy keresztül különféle
ciklusokon.

Eleinte az elme elégedett az üléssel, ellazult figyelemmel pihen, mint
amikor egy séta után leülünk egy padra: ülni és lélegezni a béke
teljessége. De gondolatok jelennek meg és követjük őket. Megállunk, újra
nyugodtak és csendesek vagyunk, a gondolkodás lehet, hogy meg is áll
anélkül, hogy észrevennénk, hogy nem gondolkodunk. De a figyelem elkezd
mozogni, és megint észrevesszük magunkat, hogy gondolkodunk. Emlékek,
vágyak, nyugtalanság jelenik meg és észrevesszük, hogy ezen dolgoznunk
kell. Ezután az elme újra nyugodt, és visszatér a csendesség érzéséghez.

\keywords{tudás és megismerés, névadó folyamat, a tudás nem a miénk}

Némi ismeretre szükség van, de egy kevés is elég. A Buddha tanítására
emlékezni olyan kincs, ami nem fogy ki. A belátások és megértés viszont
nem válik a \emph{mi tudásunkká}, mintha az attól fogva a tulajdonunk
lenne. Az igazságot nem tehetjük egy dobozba, hogy eltárazzuk a
következő alkalomra, ehelyett folyamatosan felismerjük azt a jelenben.
Amikor rögzített elképzeléseket hozunk létre arról, amiről úgy gondoljuk
ezt már tudjuk, a gyakorlásunk elveszíti a kapcsolatot a valósággal.
Minden alkalommal újból az elejénél kezdjük, és onnan, bízunk a jelen
megismerésében.

A gondolkodó elme vonzódik a tényekhez és megállapításokhoz, egyfajta
biztonságot érzünk abban, ha tényeket tudunk felmondani. Szeretnénk
megállapítani, hogy `ez jó meditáció volt, ez rossz meditáció volt'.
Különbséget akarunk tenni és nevet adni a tapasztalatnak.

Ilyen az elégedetlen elme. Valamivé válni akar, meg akar érkezni egy
állapotba és nevet akar magának. De sehol nem szeret megállni. Megy
tovább és tovább, amíg csak észre nem vesszük, hogy a folytonos futásban
teljesen kimerültünk.

Amikor a tudatban láthatóvá válik, hogy mi magunk tesszük ezt, a
névkeresés megáll. Megáll, mert a látás felváltotta a nem-látást; a
tudás felváltotta a tudatlanságot. Tudatosan látni a névadó folyamatot
elegendő, hogy megtörje a kényszert a folytatásra.

A jelenben minden változik, semmi sem rögzített. Minden mozog, a
tapasztalat erre-arra fordul és folyik. Nem áll meg egy fotóra és vár
amíg nevet adunk neki. Ebben a változásban, a kétséggel és aggodalommal
teli kérdések, az önazonosság és a célok elvesztik a jelentésüket. A
\emph{Mahāsatipaṭṭhāna Szutta} kifejezését használva: `\emph{Szabadon
időzik, semmihez sem kötődve a világon}.'\footnote{\href{https://a-buddha-ujja.hu/mn-10/hu/toth-zsuzsanna}{MN
  10}, Az éberség megalapozásáról szóló tanítóbeszéd}

Ennyi elég, így ismerve az elmét megállunk és megérkezünk egy helyre,
ahol hálásak tudunk lenni a létezésért. Nem egy különös dolog miatt.
Hálásnak lenni, hogy van tapasztalat, megismerés, tisztánlátás, és a
szabadság, ami engedi, hogy megálljunk és nem kell több és több felé
mennünk.

\keywords{határozatlan vonalak, korlátozott szimbólumok, megismerés névadás nélkül, érzések határvonalak nélkül}

Kiegyensúlyozott testtartásban a test kifinomult belső érzéseit könnyebb
megfigyelni. Befelé irányítjuk a figyelmet, kíváncsi hozzáállással. Nem
tudjuk előre mit fogunk találni.

Megjelennek a test érzetei, illetve az érzések, amik a megismerésükhöz
társulnak. Megtapasztaljuk őket, de gyakran nincsenek tiszta
határvonalaik. Nincsenek éleik, vagy határozott formájuk. Próbálunk
szavakat találni rájuk, de ezek nem illeszkednek jól. Nem vagyunk
biztosak abban, hogy minek nevezzük őket.

\enlargethispage*{\baselineskip}

Minden szimbólum, amit névként használhatnánk, hiányos. A nyugati
kultúránkban erősen bízunk a tényekben, és szeretünk visszatérni ahhoz a
biztonsághoz, amit a nevekben és terminológiában érzünk. Nem ismerősek
számunkra azok a tudati folyamatok, amik nem használnak neveket és
rögzített szimbólumokat. A megfigyelt érzések, a tapasztalat maga nem
tisztán meghatározott, de mégis tudjuk, hogy jelen van ez a tapasztalat.

Így meg tudjuk különböztetni a nevet adó folyamatot magától a
tapasztalattól. A test kifinomult érzései ködszerűek, nincsenek éles
határaik. Belégzés és kilégzés közben, megtapasztalhatjuk milyen ez az
érzés az egész testben -- mindenhol egyszerre. Az egész test lélegzik.
Van érzés és tapasztalat, de nincsenek nevek és éles határok.

A nevet adó folyamatot elhagyjuk, és észre vesszük, hogy képesek vagyunk
ismerni ezeket az érzéseket, ahogy jelen vannak. A megismerő elme örömét
találja abban, hogy szűrők nélkül szélesebb körben fogja be a
tapasztalatot. Tudjuk, hogy milyen a tapasztalat, anélkül, hogy nevet
kellene találnunk rá.

\keywords{az elme vizsgálata, túl sok gondolkodás}

A kártékony elmeállapotok érzetében észrevehetünk egyfajta hőséget,
nyugtalanságot, elégedetlenséget és szorongást. Emlékezünk, hogy
türelemmel forduljuk felé, és fenntartsuk a kitartást az állapot
érzéseinek jelenlétében. Ez is meg fog változni, ez is el fog múlni, és
meg tudjuk ezt várni. Amikor tudjuk hol állunk, a legtöbb esetben ennyi
elég. Az elme folyamatai maguktól meg fognak változni. Ha nem teszünk
tüzelőt a tűzre, az el fogja égetni amije van és magától kialszik.

Az elhatározás és ismétlés része a gyakorlásnak, de egy kifejezett cél
felé törekedésben az erőfeszítés keserűvé és fárasztóvá válik. `Ez már a
teljes felébredés? Vagy legalább egy része? Mikor fog már szólni a
meditációs harang?' Ne~egy állapotot keress. Az elme, ami felébredetté
akar válni, túlbonyolítja a helyzetet.

A jelen tapasztalat mindig egyszerű, az éber figyelemnek megvan a
képessége a bölcs megértésre. A gyakorlásban folyton ehhez térünk
vissza, ez irányítja az erőfeszítést.

Ezt nem erőltethetjük akarattal, és nem garantálhatjuk mi fog történni:
bíznunk kell a folyamatban. Ami marad, az a jótékony elme ami érti mi
történik. Nem sürget minket a kényszer, és nem kell végig erőltetnünk
magunkat a dolgokon. A nehézség után van terünk ahhoz, hogy megjelenjen
a hála, és értékelni tudjuk a könnyedség hűs, kényelmes érzését.

\keywords{beépített gyakorlás}

A tanítóinkra nézünk fel példaként. Nem azért meditáltak, hogy elérjenek
egy különleges állapotot és azután keressenek valami más tennivalót. A
meditáció nem elkülönült, hanem beépült az életükbe. A \emph{szutták}, a
buddhista hagyomány megőrzött szövegeinek példáiban, a Tiszteletreméltó
Száriputta az üresség szemléletét gyakorolta,\footnote{\href{https://a-buddha-ujja.hu/mn-151/hu/fenyvesi-robert}{MN
  151}, Az alamizsna megtisztítása} míg a Buddha a jeltelenre irányuló
koncentrációt tartotta fenn. Így folytatták a meditációt.
