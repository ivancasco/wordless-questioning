add: overcome struggle, flow, recover

collectedness \emph{is} pleasant.

dropping the inner critic \emph{is} a relief.

\hypertarget{ciklusok-1}{%
\chapter{Ciklusok}\label{ciklusok-1}}

A meditáció a jelenbeli érzéseken, tapasztalatokon keresztüli
megismerést tanítja. Az utasítások lépésekben épülnek fel, de a jelenben
csak egy pillanat elérhető számunkra. Nincs előre vagy hátra lépés, csak
ez az egy, ahol minden változik. Szavakat használunk, hogy leírjuk a
tapasztalatot, de a tudatosság erre a tapasztalatra szótlan. A nyelv
szimbólumai még korlátoznak is, mert azok rögzítettek, míg a tapasztalat
mozgásban van.

Nem a lépések tökéletesítése a meditáció célja, hanem a jelen
tapasztalat tiszta ismerete, a helyes nézőpont visszaállítása.
Kialakulhat bennünk az a benyomás, hogy mindig ugyanazt a lépés sort
kell teljesítenünk, és amikor az elme nem aszerint a sorrend szerint
fejlődik, csalódottak vagyunk.

Fordístd meg ezt a hozzáállást és kezd a tapasztalattal. Ha a
tapasztalat az alap, ahogy a dolgok vannak, az milyen megértést ad
nekünk? Amikor falat festünk, először megvizsgáljuk a falat,
kiválasztjuk hozzá a megfelelő festéket, és \emph{azután} követjük a
tanácsokat a dobozon. A rossz festék csak le fog válni, nem? Először az
elmét vizsgáljuk, hogy érezzük magunkat, milyen állapotban vagyunk, és
arra válaszolunk intelligensen.

A lépések az imitációs tanulás részei, egy példát követve figyeljük
önmagunkat és meglátjuk mi hogy működik. Amikor fájdalmat és szenvedést
érzünk, és vagy fel tudjuk oldani, vagy türelmesen kivárni, akkor
tudjuk, hogy ez nem csupán imitáció volt, és a gyakorlásba vetett
bizalmunk erősödni fog. Ebből tanultunk valamit, és nem ragaszkodunk az
első, bemutató példa részleteihez.

Remek lenne, ha a meditációnk egyenes vonalban fejlődne, a gyakorlással
töltött percek és órák számával egyenes arányban. Úgy képzeljük, hogy le
fogunk ülni, az elején talán kissé szétszórtan, és egy órával később,
\emph{ha jól tudunk meditálni}, nyugalmat fogunk érezni, az elménk
tiszta és összeszedett lesz. Legalábbis erre számítunk.

Később, ha vissza emlékezünk mi történt a meditáció alatt, azt látjuk,
hogy nem ez szokott történni. A tapasztalat nem egyenes vonalban
fejlődik a sekélytől a mélyig, vagy a szétszórttól az összeszedettig.
Azt gondolhatjuk, ez a mi hibánk, mert nem vagyunk jók a meditációban,
vagy nem helyesen követjük a lépéseket.

Amit megpróbálunk terv szerint, lépésről lépésre haladni, minden
különbözik az elvárásainktól. Lehet talán, hogy nem próbáltuk elég
erősen? Egyre jobban neki feszülünk, és csak egyre fájdalmasabb lesz.
Ilyen az az érzés, amikor a véleményünket rá akarjuk erőltetni a
tapasztalatunkra.

Ha felidézzük, hogy a tapasztalatunk hogyan változik idő közben, más
mintát látunk. Egy tapasztalat megjelenik, változások megy keresztül,
elmúlik, és egy újabb tapasztalat jelenik meg. Az elme ilyen ciklusokban
fejlődik, és ezek a ciklusok nem vesznek tudomást a céljainkról, hogy a
meditációnkat úgy akarjuk fejleszteni mint egy rang létrát.

Vehetjük a tapasztalatot alapigazságnak, és onnan indulunk. Milyen fajta
tapasztalat ez itt? Észrevehetjük, hogyan mozog a figyelem, mint a
tudatosság egy folyamata, hogyan megy keresztül különféle ciklusokon.

Eleinte az elme elégedett az üléssel, és a figyelem ellazult, pihen a
ülésben. Gondolatok jelennek meg és követjük őket. Megállunk, újra
nyugodtak és csendesek vagyunk, a gondolkodás lehet, hogy meg is áll
anélkül, hogy mi észrevennénk, hogy nem gondolkodunk. A figyelem elkezd
mozogni, és megint észrevesszük magunkat, hogy gondolkodunk. Emlékek,
vágyak, nyugtalanság jelenik meg és észrevesszük, hogy ezen dolgoznunk
kell. Ezután az elme újra nyugodt, és visszatér a csendesség érzéséghez.

Némi ismeretre szükség van, de egy kevés is elég. A Buddha tanítására
emlékezni olyan kincs, ami nem fogy ki. Ez a tudás viszont nem válik a
miénkké, nem tehetjük egy dobozba, hogy eltárazzuk a következő
alkalomra. Bármit is tanultunk, minden alkalommal újból az elejénél
kezdjük, és onnan, bízunk a jelen megismerésében.

A gondolkodó elme vonzódik a tényekhez és megállapításokhoz, egyfajta
biztonságot érzünk abban, ha tényeket tudunk felmondani. Szeretnénk
megállapítani, hogy ez egy jó meditáció, vagy rossz meditáció,
különbséget akarunk tenni és nevet adni neki.

Ilyen az elégedetlen elme. Valamivé válni akar, meg akar érkezni egy
állapotba és nevet akar magának. De sehol nem szeret megállni. Megy
tovább és tovább, amíg csak észre nem vesszük, hogy ez a folytonos futás
kész őrület.

Amikor a tudatban láthatóvá válik, hogy mi magunk tesszük ezt, a
névkeresés megáll. Megáll, mert a tudatlanságot, a nem-látást,
felváltotta a látás. A tudatos látás elegendő, hogy megtörje a kényszert
a folytatásban.

A jelenben minden változik, semmi sem állandó. Minden mozog, a
tapasztalat erre-arra fordul és folyik, nem áll meg egy fotóra és vár
amíg nevet adunk neki.

Ennyi elég, így ismerve az elmét megállunk és megérkezünk egy helyre,
ahol hálásak tudunk lenni a létezésért. Nem egy bizonyos dolog miatt,
csak hálásnak lenni, hogy van tapasztalat, megismerés, tisztánlátás, és
a szabadság, ami engedi, hogy megálljunk és nem kell több és több felé
mennünk.

Kiegyensúlyozott testtartásban a test kifinomult belső érzéseit könnyebb
megfigyelni. Befelé irányítjuk a figyelmet, kíváncsi módon. Nem tudjuk
előre mit fogunk találuni.

Ezek az érzések gyakran nem tisztán kivehetőek. Megtapasztaljuk őket, de
nincsenek tiszta határvonalaik. Nincsenek éleik, vagy határozott
formájuk. Próbálunk szavakat találni rájuk, de ezek nem illeszkednek
jól, nem vagyunk biztosak abban, hogy minek nevezzük őket.

Minden szimbólum, ami név lehetne, hiányos. A nyugati kultúránkban ahhoz
vagyunk szokva, hogy a tényekben bízzunk, és szeretünk visszatérni ahhoz
a biztonsághoz, amit a nevekben és terminológiában érzünk. Nem vagyunk
ismerősek azzal a tudati folyamattal, ami nem használ neveket és
rögzített szimbólumokat. Az érzések, a tapasztalat maga nem tisztán
meghatározott, csak az a tény, hogy tudjuk, hogy jelen van ez a
tapasztalat.

Így meg tudjuk különböztetni a nevet adó folyamatot magától a
tapasztalattól. A test érzései kifejezetten ködszerűek, nincsenek éles
határaik. Belégzés és kilégzés közben, megtapasztalhatjuk milyen ez az
érzés az egész testben mindenhol egyszerre. Az egész test lélegzik, van
érzés és tapasztalat, de nincsenek nevek és éles határok.

A nevet adó folyamatot elhagyjuk, és észre vesszük, hogy képesek vagyunk
egyszerűen ismerni ezeket az érzéseket, ahogy jelen vannak. A megismerő
elme örömét találja abban, hogy szűrők nélkül bevonja a tapasztalatot.
Tudjuk, hogy milyen a tapasztalat, anélkül, hogy nevet kellene találnunk
rá.

A kártékony elme állapotok érzetében észrevehetünk egyfajta hőt,
nyugtalanságot, elégedetlenséget és szorongást. Emlékezünk, hogy
türelemmel forduljuk felé, és fenntartsuk a kitartást az állapot
érzéseinek jelenlétében. Ez is meg fog változni, ez is el fog múlni, és
meg tudjuk ezt várni. Amikor tudjuk hol állunk, a legtöbb esetben ennyi
elég. Az elme folyamatai maguktól meg fognak változni. Ha nem teszünk
tüzelőt a tűzre, az el fogja égetni amije van és magától kialszik.

Ezt nem erőltethetjük akarattal, bíznunk kell a folyamatban. Ami marad,
az a jótékony elme ami érti mi történik. Nem sürgeti a kényszer, nincs
erőltetve. A nehézség után van terünk ahhoz, hogy megjelenjen a hála, a
könnyedség hűs, kényelmes érzésével.

A tanítóinkra nézünk fel példaként. Nem azért meditáltak, hogy elérjenek
egy különleges állapotot és azután keressenek valami más tennivalót. A
meditáció nem kiválik, hanem beépül az életükbe. A buddhista hagyomány
megőrzött szövegeinek példáiban, a Tiszteletreméltó Száriputta azt
mondta, elméje az üresség szemléletével marad, a Buddha azt mondta, a
jeltelen koncentrációban tölti az idejét, így folytatják a meditációt.
